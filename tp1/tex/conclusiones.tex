% TODO
% Esta seccion debe contener las conclusiones generales del trabajo.  Se deben mencionar
% las  relaciones  de  la  discusion  sobre  las  que  se  tiene  certeza,  junto  con  comentarios
% y  observaciones  generales  aplicables  a  todo  el  proceso.   Mencionar  tambien  posibles
% extensiones a los metodos, experimentos que hayan quedado pendientes, etc.

En este trabajo buscamos comparar la eficiencia temporal de dos algoritmos de resolución de sistemas lineales. Consideramos confirmada la hipótesis inicial de que, si bien Eliminación Gaussiana tiene una eficiencia similar a Factorización LU para un único sistema, cuando consideramos una familia de sistemas que varían solo en el vector y no en la matriz, la Factorización LU es considerablemente más rápida. \\

Sobre el problema que motivó esta comparación, el cálculo de una isoterma en una corona circular con temperaturas conocidas solamente en su circunferencia interna y externa, pudimos verificar que usando una discretización lo suficientemente fina, el mismo tiende a estabilizarse rápidamente, dándonos cierta confianza sobre el método elegido (interpolación lineal).\\

\section{Trabajo Futuro}
 Como posible trabajo futuro resta comparar este con otros métodos de interpolación, por ejemplo logarítmica. Asimismo, podrian estudiarse otras formas de interpolacion de la isoterma, por ejemplo interpolacion respecto a la dimension angular del problema, donde dados $k$ y $k + \Delta_\theta$ se puede aproximar $k + \frac{\Delta_\theta}{2}$ con alguna tecnica mas avanzada para dar mas suavidad a la curva sin aumentar la discretizacion. Queda pendiente tambien la implementacion de estructuras de datos y algoritmos que aprovechen la condicion de banda de la matriz, esta mejora cambiaria drasticamente el orden de complejidad espacial y variaria tambien el orden de complejidad temporal. Por otro lado, de estos algoritmos no serian triviales y requeririan un analisis mas fino.\\
 
 Lo que si pudimos experimentar un poco pero no nos alcanzo el tiempo para poner la informacion en el informe fue, al poner una condicion en los algoritmos de LU y eliminacion gaussiana que saltee el procesamiento de una fila si ya hay un cero bajo la diagonal en esa fila, produce que el tiempo empirico de computo se reduzca a tiempo cuadratico. Creemos que esto se debe a la condicion banda de la matriz, y que en muchos casos esta condicion de corte evita que el algoritmo ingrese en la tercera iteracion anidada.