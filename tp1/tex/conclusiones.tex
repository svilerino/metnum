% TODO
% Esta seccion debe contener las conclusiones generales del trabajo.  Se deben mencionar
% las  relaciones  de  la  discusion  sobre  las  que  se  tiene  certeza,  junto  con  comentarios
% y  observaciones  generales  aplicables  a  todo  el  proceso.   Mencionar  tambien  posibles
% extensiones a los metodos, experimentos que hayan quedado pendientes, etc.

En este trabajo buscamos comparar la eficiencia temporal de dos algoritmos de resolución de sistemas lineales. Consideramos confirmada la hipótesis inicial de que, si bien Eliminación Gaussiana tiene una eficiencia similar a Factorización LU para un único sistema, cuando consideramos una familia de sistemas que varían solo en el vector y no en la matriz, la Factorización LU es considerablemente más rápida. \\

Por otro lado, numéricamente, no hallamos diferencia entre EG y LU para los experimentos planteados.\\

Respecto el problema que motivó este trabajo práctico, el cálculo de una isoterma en una corona circular con temperaturas conocidas solamente en su circunferencia interna y externa, pudimos verificar que usando una discretización lo suficientemente fina, la posicion de la isoterma tiende a estabilizarse, dándonos cierta confianza sobre los métodos propuestos de seguridad. Sin embargo, si la discretización no es lo suficientemente fina, la estimacion de colapso de la pared puede volverse inconsistente, una propiedad para nada deseable.\\

\section{Trabajo Futuro}
 Como posible trabajo futuro resta comparar este con otros métodos de interpolación, por ejemplo logarítmica. Asimismo, podrian estudiarse otras formas de interpolación de la isoterma, por ejemplo interpolación respecto a la dimensión angular del problema, donde dados $k$ y $k + \Delta_\theta$ se puede aproximar $k + \frac{\Delta_\theta}{2}$ con alguna técnica mas avanzada para dar mas suavidad a la curva sin aumentar la discretización. Queda pendiente tambien la implementación de estructuras de datos y algoritmos que aprovechen la condición de banda de la matriz, esta mejora cambiaría drasticamente el orden de complejidad espacial y variaría tambien el orden de complejidad temporal. Por otro lado, de estos algoritmos no serían triviales y requeririan un analisis más fino.\\
 
 Lo que sí pudimos experimentar un poco pero no nos alcanzó el tiempo para poner la información en el informe fue, al poner una condición en los algoritmos de LU y eliminacion gaussiana que saltee el procesamiento de una fila si ya hay un cero bajo la diagonal en esa fila, produce que el tiempo empírico de computo se reduzca a tiempo cuadratico. Creemos que esto se debe a la condición banda de la matriz, y que en muchos casos esta condición de corte evita que el algoritmo ingrese en la tercera iteracion anidada.