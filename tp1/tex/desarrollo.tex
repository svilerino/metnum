% TODO

% Deben  explicarse  los  metodos  numericos  que  utilizaron  y  su  aplicacion  al  problema
% concreto  involucrado  en  el  trabajo  practico.   Se  deben  mencionar  los  pasos  que  si-
% guieron para implementar los algoritmos, las dicultades que fueron encontrando y la
% descripcion de como las fueron resolviendo.  Explicar tambien como fueron planteadas
% y realizadas las mediciones experimentales.  Los ensayos fallidos, hipotesis y conjeturas
% equivocadas,  experimentos  y  metodos  malogrados  deben  gurar  en  esta  seccion,  con
% una breve explicacion de los motivos de estas fallas (en caso de ser conocidas).

De la discretización de la ecuación del calor provista por el informe resulta una nueva ecuación que nos va a servir para armar nuestro sistema discreto:

\begin{equation}\label{calor}
\frac{t_{j-1,k}-2t_{jk}+t_{j+1,k}}{(\Delta r)^2}+\frac{1}{r}\frac{t_{j,k}-t_{j-1,k}}{\Delta r}+\frac{1}{r^2}\frac{t_{j,k-1}-2t_{jk}+t_{j,k+1}}{(\Delta \theta)^2} = 0 
\end{equation}

Para poder armar el sistema $Ax=b$ equivalente, es necesario:
\begin{itemize}
 \item
    Extraer los factores que multiplican a cada una de las cinco incógnitas: $t_{j-1,k}$; $t_{j,k}$; $t_{j+1,k}$; $t_{j,k-1}$ y $t_{j,k+1}$.

    Estos se obtienen de la ecuación \ref{calor}.
    \begin{align*}
        &t_{j-1, k}(\frac{1}{(\Delta r)^2} - \frac{1}{r_j \Delta r}) \\
        &t_{j, k}(\frac{-2}{(\Delta r)^2} + \frac{1}{r_j \Delta r} - \frac{2}{{r_j}^2 (\Delta r)^2}) \\
        &t_{j+1, k}(\frac{1}{(\Delta r)^2}) \\
        &t_{j, k-1}(\frac{1}{{r_j}^2(\Delta \theta)^2}) \\
        &t_{j, k+1}(\frac{1}{{r_j}^2(\Delta \theta)^2})
    \end{align*}
    Para simplificar, en adelante llamaremos $F_{j,k}$ al factor que multiplica a la incógnita $t_{j,k}$ y $Ft_{j,k}$ a $F_{j,k}*t_{j,k}$.
 \item
    Analizar los ``casos borde'': aquellos puntos donde la ecuación \ref{calor} no vale.
    
    Para evitar confusiones de variables, tomaremos $\theta_0 = 0$ como el menor valor posible de $\theta$ y $\theta_{n-1}$ como el mayor, pues vale $(r_j, \theta_n) = (r_j, \theta_0)$ para cualquier $j$. Los casos interesantes para valores de $j, k$ entonces son:
    \begin{enumerate}
     \item La pared interior del horno ($j = 0$; $k = 0, ..., n-1$). La ecuación en esos casos es $t_{0, k} = T_i(\theta_k)$.
     \item La pared exterior del horno ($j = m$; $k = 0, ..., n-1$). La ecuación en esos casos es $t_{m, k} = T_e(\theta_k)$.
     \item El valor mínimo de $\theta$ ($j = 0, ..., m$; $k = 0$). Se debe reemplazar $t_{j, k-1}$ por $t_{j, n-1}$ en todas las ecuaciones correspondientes.
     \item El valor máximo de $\theta$ ($j = 0, ..., m$; $k = n-1$). Se debe reemplazar $t_{j, k+1}$ por $t_{j, 0}$ en todas las ecuaciones correspondientes.
    \end{enumerate}
    Estos últimos reemplazos se pueden resumir en que todo punto $(j, k)$ sea $(j, k$ mod $n)$.
 \item
    Combinar los puntos anteriores para plantear sistema de ecuaciones a resolver:
    \begin{align*}
    &t_{0, k} = T_i(\theta_k)                                           &\forall k = 0, ..., n-1  \\
    &t_{m, k} = T_e(\theta_k)                                           &\forall k = 0, ..., n-1  \\
    &Ft_{j-1,k} + Ft_{j,k} + Ft_{j+1,k} + Ft_{j,k-1} + Ft_{j,k+1} = 0  &\forall j=1, ..., m-1; k = 1, ... , n-2 \\
    &Ft_{j-1,0} + Ft_{j,0} + Ft_{j+1,0} + Ft_{j,n-1} + Ft_{j,1} = 0    &\forall j=1, ..., m-1 \\
    &Ft_{j-1,n-1} + Ft_{j,n-1} + Ft_{j+1,n-1} + Ft_{j,n-2} + Ft_{j,0} = 0    &\forall j=1, ..., m-1 \\
    \end{align*}

    Del mismo podemos obtener fácilmente la matriz $A$ (que tendrá 5 valores no nulos por fila a lo sumo) y el vector $b$ (que será nulo en todas sus componentes salvo aquellas correspondientes a $j=0$ y $j=m$).
  \item
    Resta pensar un orden para las incógnitas que permita asegurar que la matriz resultante sea $banda$. Sobre el proceso llevado adelante para su elección hablaremos en la sección \ref{banda}. El orden elegido fue:
    
    $$ (0,0); (0,1); ... ; (j,n-1); (j+1,0); (j,1); ... ; (m, n-1)$$
\end{itemize}
