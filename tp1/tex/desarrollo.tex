% TODO

% Deben  explicarse  los  metodos  numericos  que  utilizaron  y  su  aplicacion  al  problema
% concreto  involucrado  en  el  trabajo  practico.   Se  deben  mencionar  los  pasos  que  si-
% guieron para implementar los algoritmos, las dicultades que fueron encontrando y la
% descripcion de como las fueron resolviendo.  Explicar tambien como fueron planteadas
% y realizadas las mediciones experimentales.  Los ensayos fallidos, hipotesis y conjeturas
% equivocadas,  experimentos  y  metodos  malogrados  deben  gurar  en  esta  seccion,  con
% una breve explicacion de los motivos de estas fallas (en caso de ser conocidas).

\subsection{Armado del sistema de ecuaciones}
De la discretización de la ecuación del calor provista por el informe resulta una nueva ecuación que nos va a servir para armar nuestro sistema discreto:

\begin{equation}\label{calor}
\frac{t_{j-1,k}-2t_{jk}+t_{j+1,k}}{(\Delta r)^2}+\frac{1}{r}\frac{t_{j,k}-t_{j-1,k}}{\Delta r}+\frac{1}{r^2}\frac{t_{j,k-1}-2t_{jk}+t_{j,k+1}}{(\Delta \theta)^2} = 0 
\end{equation}

Esta ecuación vale para cada punto del modelo salvo los límites, sobre los cuales hablaremos en breve.

Para poder armar el sistema $Ax=b$ equivalente, es necesario:
\begin{itemize}
 \item
    Extraer los factores que multiplican a cada una de las cinco incógnitas: $t_{j-1,k}$; $t_{j,k}$; $t_{j+1,k}$; $t_{j,k-1}$ y $t_{j,k+1}$.

    Estos se obtienen de la ecuación \ref{calor}.
    \begin{align*}
        t_{j-1, k}&*(\frac{1}{(\Delta r)^2} - \frac{1}{r_j \Delta r}) \\
        t_{j, k}  &*(\frac{-2}{(\Delta r)^2} + \frac{1}{r_j \Delta r} - \frac{2}{{r_j}^2 (\Delta r)^2}) \\
        t_{j+1, k}&*(\frac{1}{(\Delta r)^2}) \\
        t_{j, k-1}&*(\frac{1}{{r_j}^2(\Delta \theta)^2}) \\
        t_{j, k+1}&*(\frac{1}{{r_j}^2(\Delta \theta)^2})
    \end{align*}
    Por cuestiones de espacio, en adelante llamaremos $M_{j,k}$ al factor que multiplica a la incógnita $t_{j,k}$, $M_{j-1,k}$ al que multiplica a $t_{j-1,k}$ y así sucesivamente. Y resumiremos $Mt_{j,k} = M_{j,k}*t_{j,k}$, $Mt_{j-1,k}=M_{j-1,k}*t_{j-1,k}$.
 \item
    Analizar los ``casos borde'': aquellos puntos donde la ecuación \ref{calor} no vale.
    
    Para evitar confusiones de variables, tomaremos $\theta_0 = 0$ como el menor valor posible de $\theta$ y $\theta_{n-1}$ como el mayor, pues vale $(r_j, \theta_n) = (r_j, \theta_0)$ para cualquier $j$. 
    
    Los casos interesantes para valores de $j, k$ entonces son:
    \begin{enumerate}
     \item La pared interior del horno ($j = 0$; $k = 0, ..., n-1$). La ecuación en esos casos es $t_{0, k} = T_i(\theta_k)$.
     \item La pared exterior del horno ($j = m$; $k = 0, ..., n-1$). La ecuación en esos casos es $t_{m, k} = T_e(\theta_k)$.
     \item El valor mínimo de $\theta$ ($j = 0, ..., m$; $k = 0$). Se debe reemplazar $t_{j, k-1}$ por $t_{j, n-1}$ en todas las ecuaciones correspondientes.
     \item El valor máximo de $\theta$ ($j = 0, ..., m$; $k = n-1$). Se debe reemplazar $t_{j, k+1}$ por $t_{j, 0}$ en todas las ecuaciones correspondientes.
    \end{enumerate}
    Estos últimos reemplazos se pueden resumir en $$(j, k) \Rightarrow (j, k \text{ mod } n)$$
 \item
    Combinar los puntos anteriores para plantear el sistema de ecuaciones a resolver:
    \begin{align*}\label{sistema}
    &t_{0, k} = T_i(\theta_k)                                           &\forall k = 0, ..., n-1  \\
    &t_{m, k} = T_e(\theta_k)                                           &\forall k = 0, ..., n-1  \\
    &Mt_{j-1,k} + Mt_{j,k} + Mt_{j+1,k} + Mt_{j,k-1} + Mt_{j,k+1} = 0  &\forall j=1, ..., m-1; k = 1, ... , n-2 \\
    &Mt_{j-1,0} + Mt_{j,0} + Mt_{j+1,0} + Mt_{j,n-1} + Mt_{j,1} = 0    &\forall j=1, ..., m-1 \\
    &Mt_{j-1,n-1} + Mt_{j,n-1} + Mt_{j+1,n-1} + Mt_{j,n-2} + Mt_{j,0} = 0    &\forall j=1, ..., m-1
    \end{align*}

    Del mismo podemos obtener la matriz $A$ (que tendrá 5 valores no nulos por fila a lo sumo) y el vector $b$ (que será nulo en todas sus componentes salvo aquellas correspondientes a $j=0$ y $j=m$).
  \item
    Pensar un orden para las incógnitas que permita asegurar que la matriz resultante sea $banda$. El mismo es:
    
    $$ (0,0); (0,1); ... ; (j,n-1); (j+1,0); (j,1); ... ; (m, n-1)$$ % TODO: estaría bueno hacer una imagen representativa, que muestre un toque la espiral rara esta. No sé usar mucho ninguna herramienta como para hacerlo :(.
    
    tanto para las filas como para las columnas. Sobre el proceso llevado adelante para su elección hablaremos en la sección \ref{banda}.
\end{itemize}

Una vez realizados estos pasos estamos en condiciones de plantear el sistema de ecuaciones $Ax=b$:

Lo primero que debemos notar es que como hay $n*m$ puntos diferentes tendremos $n*m$ incógnitas diferentes. Luego, $A \in \mathbb{R}^{nm*nm}$: cada columna y cada fila de $A$ corresponden a un punto $t_{j,k}$ del sistema. Asimismo, $x \in \mathbb{R}^{nm}$ y $b \in \mathbb{R}^{nm}$. 

Lo segundo que debemos notar es que, por coincidir el orden elegido para filas y para columnas, el índice de la fila correspondiente al punto $t_{j,k}$ coincide con el de la columna correspondiente a ese punto. Llamaremos a este índice $i(j,k)$. Notar que podemos computar $i$ fácilmente como $i(j,k)=j*m+k$ (suponiendo que indexamos por 0 tanto filas como columnas).

Por el orden elegido, las primeras $n$ filas corresponden a los puntos $t_{0,k}$. Mirando el sistema de ecuaciones, las primeras $n$ filas de $A$ coinciden con la identidad (1 en la diagonal y 0 en el resto) y las primeras $n$ filas de $b$ coinciden con $T_i(\theta_k)$.

Lo mismo vale para las últimas $n$ filas: corresponden a los puntos $t_{m,k}$, las filas correspondientes de $A$ coinciden con la identidad y las componentes de $b$ con $T_e(\theta_k)$.

Llegado este punto podemos definir completamente $b$: todas sus demás componentes son nulas (por ser $0$ la solución al resto de las ecuaciones del sistema), por lo que resulta:

$$b = (T_i(0), T_i(1), ..., T_i(n-1), 0, ..., 0, T_e(0), T_e(1), ..., T_e(n-1)) $$

Para $j \not = 0, j \not = m$, las filas $i(j,k)$ de $A$ tendrán cinco componentes no-nulas (que corresponden a los vecinos de $t_{j,k}$ en el modelo). Fijados $j$ y $k$ ($0\not=j\not=m, 0\not=k\not=n-1$), estas componentes serán $i(j-1,k); i(j,k); i(j+1,k); i(j,k-1)$ e $i(j,k+1)$ y coincidirán con lo que anteriormente llamamos $M(j-1,k); M(j,k); M(j+1,k); M(j,k-1)$ y $M(j,k+1)$ respectivamente. 

Resta simplemente considerar los casos $k=0$ y $k=n-1$, pero no reviste mayor complejidad que tomar módulo $n$ después de las operaciones que involucren $k$.

\subsection{Resolución del sistema de ecuaciones}
