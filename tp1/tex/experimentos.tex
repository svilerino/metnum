\subsection{Métodología de experimentación}
En esta sección desarrollaremos los experimentos planteados, desde la generació de instancias de prueba hasta los criterios de medición utilizados y los detalles de los experimentos.
\subsubsection{Consideraciones iniciales}
\begin{itemize}
    \item La pared interna del horno en teoría debería ser de 1500 grados constantes, nosotros consideramos más realista que al variar los ángulos de la pared interna, la temperatura varíe uniformemente entre $1500 \pm \epsilon$, con $\epsilon$ pequeño.
\end{itemize}

\subsubsection{Generación de casos de prueba}
Los casos de prueba son generados mediante un script de python llamado testgen.py, dentro de la carpeta tools. El mismo posee 3 modos de ejecuci\'on: 
\begin{enumerate}
    \item \textbf{Modo 1:} Dados por par\'ametro un radio interno, externo e isoterma fijas, toma adem\'as  una cantidad de radios inicial, una final, una cantidad de \'angulos inicial, una final, y para cada combinaci\'on de cantidades de ellos genera un n\'umero de instancias tambi\'en provisto por par\'ametro. En cada instancia, las temperaturas internas y externas son generadas pseudoaleatoriamente, ejecutando el comando \textbf{random.uniform(1450, 1550)} para las temperaturas internas y \textbf{random.uniform(50, 200)}.
    \item \textbf{Modo 2:} Dados un radio interno, externo, isoterma, n\'umero de \'angulos y radios fijos por par\'ametro, genera un \'unico archivo en el que var\'ia las temperatura de esa discretizaci\'on particular (usando los comandos descriptos anteriormente), generando un n\'umero de instancias provisto por par\'ametro.
    \item \textbf{Modo 3:} Dados un radio interno, externo, isoterma, y n\'umero de radios fijos por par\'ametro, se le especifica una cantidad de radios inicial y una final.  Para cada cantidad de radios genera un \'unico vector de temperaturas internas y externas, y genera tantas instancias como se le especifique.

\end{enumerate}

\subsubsection{Métricas de performance}

Se medir\'a y comparar\'a la performance de ambos m\'etodos en la resoluci\'on de diversas instancias del problema. El tiempo se mide en microsegundos, y corresponde al tiempo de procesamiento de CPU, dejando fuera de consideraci\'on el tiempo de entrada/salida. El objetivo \'ultimo es encontrar una relaci\'on entre la complejidad te\'orica de ambos algoritmos y su tiempo de c\'omputo, y realizar la comparaci\'on de ambos m\'etodos en distintos escenarios. Todos los casos considerados est\'an especificados en la secci\'on \textb{Esquema de experimentaci\'on}

\subsubsection{Método de posiciónamiento estimado de la isoterma}
El problema consiste en estimar, para cada ángulo, la posición radial de la isoterma \texttt{$\alpha$}.\\
Consideremos la función de temperatura $T(r,\theta)$. Si fijamos $\theta = \theta_i$ podemos definir la función $g_{\theta_i}(r) = T(r,\theta_i)$ como la función de temperatura sobre todos los radios del ángulo $\theta_i$. Dado que nosotros conocemos $m+1$ puntos aproximados de dicha función, esto reduce el problema a aproximar el elemento $ z_\alpha \in Dom(g_{\theta_i}) $ tal que $g_{\theta_i}(z_\alpha) = \alpha$. El método propuesto consiste en aplicar el siguiente algoritmo a las aproximaciones discretas conocidas de las funciones $g_{\theta_i}$, para todos los ángulos.
\begin{theorem}[Método de estimación de la isoterma k]
    Sea $\hat{g_{\theta_i}}$ la función \textbf{discreta de aproximaciones} de temperatura de un ángulo i.
    \begin{enumerate}
        \item Buscar $ x_1, x_2 \in Dom(\hat{g_{\theta_i}}) $ tales que $ \hat{g_{\theta_i}}(x_1) \leq \alpha \leq \hat{g_{\theta_i}}(x_2)$ y esta cota sea ajustada.
        
        \item $z_\alpha = x_1 + \left(\frac{x_2 - x_1}{\hat{g_{\theta_i}}(x_2) - \hat{g_{\theta_i}}(x_1)}\right) * (\alpha - \hat{g_{\theta_i}}(x_1))$

        \item Devolver $z_\alpha$.
    \end{enumerate}
    $z_\alpha$ es la posición estimada de la isoterma tal que $g_{\theta_i}(z_\alpha) = \alpha$.
\end{theorem}

\begin{proposition}[Dominio de correctitud de la estimación]
    El algoritmo anterior estima linealmente la isoterma $\alpha$ entre $x_1$ y $x_2$ si:
    \begin{itemize}
        \item $\displaystyle\min_{x \in Dom(\hat{g_{\theta_i}})}{\hat{g_{\theta_i}}(x)} \leq \alpha \leq \displaystyle\max_{x \in Dom(\hat{g_{\theta_i}})}{\hat{g_{\theta_i}}(x)}$.
        \item $\alpha \notin Im(\hat{g_{\theta_i}})$.
    \end{itemize}
    \begin{itemize}
        \item En caso de que $\alpha$ esté fuera del rango $[min..max]$ por convención se establece que la isoterma se encuentra en una posición radial $R_i - \epsilon$ o $R_e + \epsilon$ según corresponda.
        \item En caso de que $\alpha \in Im(\hat{g_{\theta_i}})$ entonces $z = x_1 = x_2$ y no es necesario ajuste lineal.
    \end{itemize}
\end{proposition}
\begin{proof}
    Si se cumplen las hipótesis de la proposición anterior entonces la cota existe y el algoritmo procederá a hacer el cálculo aproximado del paso siguiente, que no es más que un ajuste lineal entre los dos puntos de las cotas. Es decir, estamos asumiendo que el calor se disipa linealmente entre 2 puntos cualesquiera de la función.\\
    \vspace{0.3cm}
    \textbf{Nota:} Asumir esto no es necesariamente correcto, se podría hacer un analisis más fino graficando las funciones $g_{\theta_i}$ y aplicando métodos más avanzados de estimación para que la curva quede más suave. Por simplicidad solo consideramos el ajuste lineal en este trabajo.
\end{proof}

\subsubsection{Métricas de seguridad de la isoterma}
Plantemos una métrica que estima la \texttt{estabilidad} o \texttt{seguridad} de la pared del horno estableciendo una relacion relativa entre la posición de la isoterma y el radio externo. 
\begin{proposition}[Métrica de seguridad del horno basada en la posición relativa de la isoterma]
    Consideremos $\Delta_{iso_\alpha} = \left( \frac{f(iso_\alpha) - R_i}{R_e - R_i} \right)$.\\
    Donde $f(iso_\alpha)$ es una función de la isoterma, en nuestro caso, consideramos que el máximo o el promedio son buenas métricas.\\
    Salvo casos patológicos, vale que $0 \leq \Delta_{iso_\alpha}\leq 1$. Luego basta establecer un \textbf{límite de seguridad} $0 \leq \gamma_0 \leq 1$ tal que si vale $\Delta_{iso_\alpha} > \gamma_0$ se considera \texttt{inestable} o \texttt{insegura} la pared del horno.
\end{proposition}

\subsection{Esquema de experimentación}
\subsubsection{Medida de la performance}

Los experimentos de performance se dividen en tres secciones:

\begin{enumerate}

\item \textbf {Performance de los m\'etodos en funci\'on de la discretizaci\'on para una instancia:}  Se generaron diversas discretizaciones del problema, fijando la cantidad de \'angulos en 30, y variando la cantidad de radios de 5 a 30, con una instancia para cada discretizaci\'on. El objetivo es analizar la evoluci\'on del tiempo de c\'omputo a mayor granulaidad de la discretizaci\'on, y comprobar emp\'iricamente que la complejidad de ambos algoritmos es del orden c\'ubico.
\item \textbf {Performance de los m\'etodos en funci\'on de la cantidad de instancias:} Para una misma discretizaci\'on elegida de 30 \'angulos y 30 radios, se generaron 3 archivos con distinto n\'umero de instancias, el primero de una instancia, el segundo de 5 instancias, y el tercero de 10. En cada instancia cambian las condiciones de borde. Se medir\'a el tiempo total de ejecuci\'on de cada uno de los dos m\'etodos para cada archivo, de forma de poder comparar la perfomance de ambos m\'etodos bajo una misma discretizaci\'on, variando la cantidad de instancias a resolver.
\item \textbf {Performance de los m\'etodos en funci\'on de la discretizaci\'on para varias instancias:}
Se compara la perfomance de ambos m\'etodos en un caso combinado de los anteriores. Dado un n\'umero fijo de 50 instancias, se medir\'a el tiempo total de c\'omputo en tres discretizaciones: 20 \'angulos y 20 radios, 40 \'angulos y radios, y 60 \'angulos y radios. 
\end{enumerate}

\subsubsection{Evolución de la temperatura y posición de la isoterma con distintas discretizaciones}
Para evaluar la calidad de nuestros estimadores y la evolucion de la posición de la isoterma planteamos los siguientes experimentos:\\
Experimentos de estimación de isoterma con distintas discretizaciones:
\begin{enumerate}
    \item \textbf{Variación de cantidad de radios.} La cantidad de ángulos, las temperaturas (internas y externas) y los demás parametros se mantienen invariantes entre los tests de este experimento
    \begin{itemize}
        \item Se plantean los radios internos, externos y la isoterma buscada y se mantienen fijos durante todo el experimento.
        \item Se plantean un conjunto de temperaturas internas y externas \textbf{aleatorias uniformes} iniciales y se mantienen fijas durante todo el experimento.
        \item Se plantea un numero de ángulos de la discretización y se mantiene fijo durante todo el experimento. 
        \item Se plantea un rango $[r_{min}, r_{max}]$ que denota la cobertura de discretizaciones distintas del experimento.
        \item Se generan archivos de entrada $test_i$ variando unicamente la cantidad de radios de la discretización a utilizar con una instancia por archivo.
        \item Se ejecutan todos los archivos de entrada con el método de resolución más conveniente.
        \item Se grafica, para cada archivo de test, la posición de la isoterma y el mapa de temperaturas del horno.
        \item Se considera un video que tiene por frames los graficos ordenados en el rango $[r_{min}..r_{max}]$ del ítem anterior relacionado con la posición de la isoterma.
        \item Se considera un video que tiene por frames los graficos ordenados en el rango $[r_{min}..r_{max}]$ del ítem anterior relacionado con la temperatura de la pared del horno.
        \item Se grafica una función en el plano que denota la \textbf{máxima} posición relativa de la isoterma en la pared del horno a medida que varía ordenadamente el rango $[r{min}..r_{max}]$.
        \item Se grafica una función en el plano que denota la posición relativa \textbf{promedio} de la isoterma en la pared del horno a medida que varía ordenadamente el rango $[r{min}..r_{max}]$.
    \end{itemize}  Estos ultimos 2 ítems corresponden a cuantificar el error de la estimación lineal de las funciones $g_{\theta_i}$ a medida que aumentan los radios, y en consecuencia teniendo más puntos discretos en las funciones $\hat{g_{\theta_i}}$.

    \item \textbf{Variación de cantidad de ángulos.} La cantidad de radios, las temperaturas (internas y externas) y los demás parametros, se mantienen invariantes entre los tests de este experimento\begin{itemize}
        \item Se plantean los radios internos, externos y la isoterma buscada y se mantienen fijos durante todo el experimento.
        \item Se plantean un conjunto de temperaturas internas y externas \textbf{constantes} (si son aleatorias no asegurar misma solución entre tests del experimento) y se mantienen fijas durante todo el experimento, usándose en la creación de cada archivo de input con diferente cantidad de ángulos.
        \item Se plantea un numero de radios de la discretización y se mantiene fijo durante todo el experimento. 
        \item Se plantea un rango $[\theta_{min}, \theta_{max}]$ que denota la cobertura de discretizaciones distintas del experimento.
        \item Se generan archivos de entrada $test_i$ variando unicamente la cantidad de ángulos de la discretización a utilizar con una instancia por archivo.
        \item Se ejecutan todos los archivos de entrada con el método de resolución más conveniente.
        \item Se grafica, para cada archivo de test, la posición de la isoterma y el mapa de temperaturas del horno.
        \item Se considera un video que tiene por frames los graficos ordenados en el rango $[\theta_{min}, \theta_{max}]$ del ítem anterior relacionado con la posición de la isoterma.
        \item Se considera un video que tiene por frames los graficos ordenados en el rango $[\theta_{min}, \theta_{max}]$ del ítem anterior relacionado con la temperatura de la pared del horno.
    \end{itemize}    

    \item \textbf{Variación conjunta de cantidad de radios y ángulos.} Las temperaturas (internas y externas) y los demás parametros, se mantienen invariantes entre los tests de este experimento\begin{itemize}
        \item Se plantean los radios internos, externos y la isoterma buscada y se mantienen fijos durante todo el experimento.
        \item Se plantean un conjunto de temperaturas internas y externas \textbf{constantes} (si son aleatorias no asegurar misma solución entre tests del experimento) y se mantienen fijas durante todo el experimento.
        \item Se plantea un rango $[r_{min}, r_{max}]$ que denota la cobertura de discretizaciones distintas del experimento.        
        \item Se plantea un rango $[\theta_{min}, \theta_{max}]$ que denota la cobertura de discretizaciones distintas del experimento.
        \item Se generan archivos de entrada $test_i$ variando ambos rangos mencionados anteriormente en la discretización a utilizar con una instancia por archivo.
        \item Se ejecutan todos los archivos de entrada con el método de resolución más conveniente.
        \item Se grafica, para cada archivo de test, la posición de la isoterma y el mapa de temperaturas del horno.
        \item Se considera un video que tiene por frames los graficos ordenados convenientemente según granularidad, del ítem anterior relacionado con la posición de la isoterma.
        \item Se considera un video que tiene por frames los graficos ordenados convenientemente según granularidad, del ítem anterior relacionado con la temperatura de la pared del horno.
    \end{itemize}    
\end{enumerate}

Los experimentos aquí planteados, tienen la misma solución entre tests, variando unicamente las discretizaciones, esto nos debería dar la pauta de como cambia la posición estimada de la isoterma al cambiar la discretización.

Se espera poder obtener conclusiones acerca de la suavidad de la curva estimada de la isoterma a medida que disminuye la granularidad de la discretización. Asimismo en las funciones que grafican el máximo y el promedio, se espera poder obtener conclusiones similares respecto a la variacion radial de la curva polar de la isoterma.\\

\subsubsection{Evolución de la posición de la isoterma con distintas discretizaciones y temperaturas externas}
Experimentos de estimación de \texttt{estabilidad} de la pared del horno:
\begin{enumerate}
    \item Para distintas discretizaciones, se plantea una variacion gradual en las temperaturas externas entre 50 y 1350 grados. Esto modifica la posición estimada de la isoterma 500, asimismo cambiando el coeficiente de seguridad mencionado más arriba.
\end{enumerate}