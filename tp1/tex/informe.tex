\documentclass[10pt, a4paper, english, spanish]{article}

\usepackage[spanish]{babel}
\parindent = 0 pt
\parskip = 5 pt
\usepackage[width=15.5cm, left=3cm, top=2.5cm, height= 24.5cm]{geometry}
\usepackage{amsmath}
\usepackage{amsfonts}
\usepackage{amssymb}
\usepackage{amsthm}
\usepackage[utf8]{inputenc}
\usepackage{graphicx}
\usepackage{verbatim}
\usepackage{color}
\usepackage[colorinlistoftodos]{todonotes}
\usepackage{multicol}
\usepackage{makeidx}
\usepackage{hyperref}
\usepackage{caption}
\usepackage{listings}
\usepackage{algpseudocode}
\usepackage{courier}
\usepackage{enumitem}
\usepackage{placeins}
\usepackage{booktabs}
\usepackage{color}

% \usepackage[lined, ruled, linesnumbered]{algorithm2e}
% \usepackage[margin=1in]{geometry}

\newtheorem{theorem}{Teorema}[section]
\newtheorem{corollary}{Corolario}[theorem]
\newtheorem{definition}[theorem]{Definicion}

\newcommand{\norm}[1]{\left\lVert#1\right\rVert}

\newcommand{\BigO}[1]{\ensuremath{\operatorname{O}\bigl(#1\bigr)}}
\newtheorem{proposition}{Proposici\'on}
\newtheorem{hipotesis}{Hip\'otesis}
\newtheorem{algoritmo}{Algoritmo}
\newtheorem{lemma}{Lema}
\lstset{language=C++, showstringspaces=false, tabsize=2, breaklines=true, title=\lstname}

\makeindex

\begin{document}
\newgeometry{margin=2cm}
\pagenumbering{gobble}
\raggedleft
\includegraphics[width=8cm]{caratula/logo1.jpg}\\

\raggedright
\vspace{3cm}
{\Huge \bfseries Trabajo Práctico 1 \\ Con 15 $\theta$s discretizo alto horno\ldots}
\rule{\textwidth}{0.02in}
\large Jueves 3 de septiembre de 2015 \hfill Métodos Numéricos
\vspace{1.5cm}

\normalsize
\begin{tabular}{|l@{\hspace{5ex}}c@{\hspace{5ex}}l|}
        \hline
        \rule{0pt}{1.2em}Integrante & LU & Correo electrónico\\[0.2em]
        \hline
        \rule{0pt}{1.2em} Lascano, Nahuel  & 476/11 &\tt laski.nahuel@gmail.com\\[0.2em]
        \rule{0pt}{1.2em} Vileriño, Silvio & 106/12 &\tt svilerino@gmail.com\\[0.2em]
        \hline
\end{tabular}

\medskip
En este trabajo aplicamos dos métodos de resolución de sistemas de ecuaciones lineales (Factorización LU y Eliminación Gaussiana) para el cálculo de isotermas de una corona circular, dadas las temperaturas de las circunferencias interior y exterior.

Pudimos verificar experimentalmente que el método de factorización LU resulta más eficiente si se tienen varias posibles soluciones para una misma matriz, pero que para una única instancia es conveniente usar la eliminación gaussiana.

Para el cálculo de la isoterma, concluimos que para una discretización lo suficientemente fina alcanza con hacer una extrapolación lineal de las temperaturas obtenidas, pero que si la discretización es pobre la estimación no presenta confianza.

\medskip
Palabras clave: factorización LU, eliminación gaussiana, sistemas de ecuaciones lineales, matriz banda

\raggedright

\begin{multicols}{2}
\includegraphics[width=8cm]{caratula/logo-uba.png}

\columnbreak
\vspace*{3.5cm}
\raggedleft
\textbf{Facultad de Ciencias Exactas y Naturales}\\
Universidad de Buenos Aires\\
\small
Ciudad Universitaria - (Pabellon I/Planta Baja)\\
Intendente G\"uiraldes 2160 - C1428EGA\\
Ciudad Autonoma de Buenos Aires - Rep. Argentina\\
Tel/Fax: (54 11) 4576-3359\\
http://www.fcen.uba.ar
\end{multicols}

\restoregeometry

\clearpage

\pagenumbering{arabic}

\tableofcontents

\vspace{3cm}

\clearpage

\setlength{\parindent}{10pt}

\section{Introducción teórica}
En el presente trabajo se intentó evaluar computacionalmente la seguridad térmica de un horno circular. El problema presentado consistía en estimar el riesgo que corre el mismo de fracturarse por efecto de la elevada temperatura. Dicho de otro modo, dadas las temperaturas de las paredes internas y externas del horno (obtenidas a través de sensores) se quiere estimar la ubicación de la isoterma de 500$^{o}$C, cuya cercanía a la pared externa es un indicador de la peligrosidad de la estructura.

Dado un corte transversal del horno podemos definir $r_i$ y $r_e$ como los radios de la pared interna y externa, ambos circualres. Nos interesa analizar la temperatura de los puntos entre ellas. Para referirnos a los puntos de dicha corona circular utilizaremos coordenadas polares, por lo que cada punto $P$ quedará definido por un radio $r$ y un ángulo $\theta$. Si llamamos $T(r,\theta)$ a la temperatura del punto $P_{r, \theta}$, podemos utilizar la ecuación del calor de Laplace para encontrar el estado de equilibrio del sistema:

\begin{equation}\label{calor-continuo}
\frac{\partial^2T(r,\theta)}{\partial r^2}+\frac{1}{r}\frac{\partial T(r,\theta)}{\partial r}+\frac{1}{r^2}\frac{\partial^2T(r,\theta)}{\partial \theta^2} = 0 
\end{equation}

la cual debe cumplirse para todos los puntos internos del horno.

Por estar trabajando con aritmética finita, usamos una discretización de los puntos que nos interesa analizar (todos los pertenecientes a la pared del horno, que forman una corona circular) y discretizamos asimismo la ecuación \ref{calor-continuo}. De esa manera arribamos a un sistema de ecuaciones lineal que podemos representar en su versión matricial como $Ax=b$ y que intentaremos resolver usando dos métodos: la Eliminación Gaussiana y Factorización LU.

Nos interesa comparar estos dos métodos por el tiempo que les lleva resolver un único sistema $Ax=b$ en función de la granularidad de la discretización. Posteriormente podemos complejizar el problema suponiendo que tenemos múltiples mediciones para las temperaturas de las paredes (a lo largo del tiempo), por lo que debemos comparar su performance a la hora de resolver mútiples sistemas $Ax=b_i$ con diferentes vectores $b_i$.

La solución del sistema de ecuaciones nos permitirá conocer el valor de la funcion $T$ en los puntos de la discretización elegida, pero es posible que ninguno de ellos coincida con el valor de la isoterma buscada. Otro objetivo del trabajo será evaluar diferentes formas de estimar esa isoterma y compararlas variando la granularidad de la discretización.


\clearpage

\section{Desarrollo}
% TODO

% Deben  explicarse  los  metodos  numericos  que  utilizaron  y  su  aplicacion  al  problema
% concreto  involucrado  en  el  trabajo  practico.   Se  deben  mencionar  los  pasos  que  si-
% guieron para implementar los algoritmos, las dicultades que fueron encontrando y la
% descripcion de como las fueron resolviendo.  Explicar tambien como fueron planteadas
% y realizadas las mediciones experimentales.  Los ensayos fallidos, hipotesis y conjeturas
% equivocadas,  experimentos  y  metodos  malogrados  deben  gurar  en  esta  seccion,  con
% una breve explicacion de los motivos de estas fallas (en caso de ser conocidas).

De la discretización de la ecuación del calor provista por el informe resulta una nueva ecuación que nos va a servir para armar nuestro sistema discreto:

\begin{equation}\label{calor}
\frac{t_{j-1,k}-2t_{jk}+t_{j+1,k}}{(\Delta r)^2}+\frac{1}{r}\frac{t_{j,k}-t_{j-1,k}}{\Delta r}+\frac{1}{r^2}\frac{t_{j,k-1}-2t_{jk}+t_{j,k+1}}{(\Delta \theta)^2} = 0 
\end{equation}

Para poder armar el sistema $Ax=b$ equivalente, es necesario:
\begin{itemize}
 \item
    Extraer los factores que multiplican a cada una de las cinco incógnitas: $t_{j-1,k}$; $t_{j,k}$; $t_{j+1,k}$; $t_{j,k-1}$ y $t_{j,k+1}$.

    Estos se obtienen de la ecuación \ref{calor}.
    \begin{align*}
        &t_{j-1, k}(\frac{1}{(\Delta r)^2} - \frac{1}{r_j \Delta r}) \\
        &t_{j, k}(\frac{-2}{(\Delta r)^2} + \frac{1}{r_j \Delta r} - \frac{2}{{r_j}^2 (\Delta r)^2}) \\
        &t_{j+1, k}(\frac{1}{(\Delta r)^2}) \\
        &t_{j, k-1}(\frac{1}{{r_j}^2(\Delta \theta)^2}) \\
        &t_{j, k+1}(\frac{1}{{r_j}^2(\Delta \theta)^2})
    \end{align*}
    Para simplificar, en adelante llamaremos $F_{j,k}$ al factor que multiplica a la incógnita $t_{j,k}$ y $Ft_{j,k}$ a $F_{j,k}*t_{j,k}$.
 \item
    Analizar los ``casos borde'': aquellos puntos donde la ecuación \ref{calor} no vale.
    
    Para evitar confusiones de variables, tomaremos $\theta_0 = 0$ como el menor valor posible de $\theta$ y $\theta_{n-1}$ como el mayor, pues vale $(r_j, \theta_n) = (r_j, \theta_0)$ para cualquier $j$. Los casos interesantes para valores de $j, k$ entonces son:
    \begin{enumerate}
     \item La pared interior del horno ($j = 0$; $k = 0, ..., n-1$). La ecuación en esos casos es $t_{0, k} = T_i(\theta_k)$.
     \item La pared exterior del horno ($j = m$; $k = 0, ..., n-1$). La ecuación en esos casos es $t_{m, k} = T_e(\theta_k)$.
     \item El valor mínimo de $\theta$ ($j = 0, ..., m$; $k = 0$). Se debe reemplazar $t_{j, k-1}$ por $t_{j, n-1}$ en todas las ecuaciones correspondientes.
     \item El valor máximo de $\theta$ ($j = 0, ..., m$; $k = n-1$). Se debe reemplazar $t_{j, k+1}$ por $t_{j, 0}$ en todas las ecuaciones correspondientes.
    \end{enumerate}
    Estos últimos reemplazos se pueden resumir en que todo punto $(j, k)$ sea $(j, k$ mod $n)$.
 \item
    Combinar los puntos anteriores para plantear sistema de ecuaciones a resolver:
    \begin{align*}
    &t_{0, k} = T_i(\theta_k)                                           &\forall k = 0, ..., n-1  \\
    &t_{m, k} = T_e(\theta_k)                                           &\forall k = 0, ..., n-1  \\
    &Ft_{j-1,k} + Ft_{j,k} + Ft_{j+1,k} + Ft_{j,k-1} + Ft_{j,k+1} = 0  &\forall j=1, ..., m-1; k = 1, ... , n-2 \\
    &Ft_{j-1,0} + Ft_{j,0} + Ft_{j+1,0} + Ft_{j,n-1} + Ft_{j,1} = 0    &\forall j=1, ..., m-1 \\
    &Ft_{j-1,n-1} + Ft_{j,n-1} + Ft_{j+1,n-1} + Ft_{j,n-2} + Ft_{j,0} = 0    &\forall j=1, ..., m-1 \\
    \end{align*}

    Del mismo podemos obtener fácilmente la matriz $A$ (que tendrá 5 valores no nulos por fila a lo sumo) y el vector $b$ (que será nulo en todas sus componentes salvo aquellas correspondientes a $j=0$ y $j=m$).
  \item
    Resta pensar un orden para las incógnitas que permita asegurar que la matriz resultante sea $banda$. Sobre el proceso llevado adelante para su elección hablaremos en la sección \ref{banda}. El orden elegido fue:
    
    $$ (0,0); (0,1); ... ; (j,n-1); (j+1,0); (j,1); ... ; (m, n-1)$$
\end{itemize}

\par A continuaci\'on se detallan todos los experimentos realizados en este
trabajo y sus resultados. Se detalla no s\'olo el experimento en si, sino que
tambi\'en se explican los resultados que se esperan comprobar y sus
motivaciones.

%---------------------------------------------------------------
%\subsection{PageRank}
%\subsection{Experimento 1}
\label{subsec:exp1}
\begin{LaTeXdescription}
    \item[Tesis]

    \item[Proposici\'on] 

    \item[M\'etodo de Experimentaci\'on]

    \item[Resultados, an\'alisis y discusi\'on]
\end{LaTeXdescription}


%\newpage
%\subsection{\'Ordenes: Google vs PageRank vs In-Deg}
\label{subsec:exp2}
\begin{LaTeXdescription}
    \item[Objetivo] Analizar el orden obtenido respecto de otros \'ordenes
        disponibles. ¿Es igual? ¿Hay coincidencias? ¿Cu\'antas? ¿Tienen
        sentido?\\

    \item[Proposici\'on] Cualitativamente hablando, ¿c\'omo es el orden obtenido
        por PageRank? ¿Bueno? ¿Malo?. Obviamente que estas categorizaciones son
        intr\'insecas a la realidad: s\'olo nosotros podemos decir que al
        realizar una b\'usqueda web, los resultados vinieron en un orden
        correcto o deseable (es decir, lo que se buscaba en las primeras
        posiciones).  Utilizar nuestro criterio personal para hablar de la
        calidad del orden obtenido no ser\'ia muy correcto, ya que cualquier
        otra persona con criterios distintos podr\'ia disentir y ninguno de los
        criterios ser\'ia \textit{a priori} m\'as correcto que el
        otro\footnote{Se podr\'ia ver muestralmente que opina la gente de
        distintos \'ambitos, pero esto escapa al objeto de estudio de este
        trabajo.}. Pero lo que si podemos hacer es comparar el resultado
        obtenido con otros resultados disponibles, de los cuales proponemos
        In-Deg (que se basa en el grafo de conectividad) y los resultados de un
        \textit{search engine}: Google.\\

    \item[M\'etodo de Experimentaci\'on] Utilizamos la misma instancia que en el
        experimento anterior. Sobre esta instancia s\'olo nos falta calcular el
        orden In-Deg (el cual no es otra cosa que ordenar a los nodos en orden
        descendiente seg\'un su grado de entrada, es decir seg\'un la cantidad
        de ejes que los apuntan). Luego utilizamos el orden provisto por Google
        en la b\'usqueda inicial m\'as los \'ordenes obtenidos en el experimento
        previo.\\

    \item[Resultados, an\'alisis y discusi\'on]
\end{LaTeXdescription}

\begin{table}[H]
    \centering
    \caption{\'Ordenes comparativos entre los resultados de Google, PageRank e
        In-Deg}
    \label{tbl:google_pagerank_vs_indeg_siteubaar} 
    \setlength{\tabcolsep}{3pt}
    \begin{tabular}{|l|l|l|}
        \hline
        Google & PageRank & In-Deg\\
        \hline\hline
        www.derecho.uba.ar & www.agro.uba.ar & www.agro.uba.ar\\
        orga2.exp.dc.uba.ar & www.uba.ar & www.uba.ar\\
        www.agro.uba.ar & videos.agro.uba.ar & videos.agro.uba.ar\\
        www.ffyb.uba.ar & www.agro.uba.ar/cursos & www.agro.uba.ar/cursos\\
        www.uba.ar & www.agro.uba.ar/ced & www.agro.uba.ar/ced\\
        www.fvet.uba.ar & www.derecho.uba.ar & www.derecho.com.ar\\
        videos.agro.uba.ar & www.ffyb.uba.ar & www.ffyb.uba.ar\\
        iigg.sociales.uba.ar & www.fvet.uba.ar & www.fvet.uba.ar\\
        www.agro.uba.ar/cursos & orga2.exp.dc.uba.ar & orga2.exp.dc.uba.ar\\
        www.agro.uba.ar/ced & iigg.sociales.uba.ar & iigg.sociales.uba.ar\\
        \hline
    \end{tabular}
\end{table}

\par Los resultados obtenidos en \ref{tbl:google_pagerank_vs_indeg_siteubaar}
dieron resultados idénticos en cuanto a In-Deg vs PageRank, no así la búsqueda
en google, que debe utilizar otras heurísticas que no consideramos en este
trabajo. Asimismo, el orden de las búsquedas en google para un mismo término van
cambiando a lo largo del tiempo\footnote{True Story.} .\\

\par Respecto a In-Deg y PageRank, sus \'ordenes resultaron idénticos. Los
motivos por lo cual esto ocurre fueron desarrollados en el experimento anterior,
pero intuyendo que esto no tiene que ser siempre
as\'i (sino claramente PageRank no tendr\'ia sentido), realizamos un nuevo
experimento. Creamos una nueva instancia de prueba basada en los resultados de
buscar en wikipedia\cite{wikipedia} distintos t\'erminos relacionados con los
temas vistos en la materia. El grafo de conectividad resultante se puede
observar en la figura \ref{fig:wiki_graph}.

\begin{figure}[H]
    \centering
    \includegraphics[width=0.75\textwidth]{exp2_conn_graph_metodos.png}
    \caption{Grafo de conectividad de p\'aginas de Wikipedia relacionadas con
        m\'etodos num\'ericos}
    \label{fig:wiki_graph}
\end{figure}

\par Sobre estas p\'aginas, calculamos los \'ordenes de PageRank e In-Deg, cuya
comparativa se encuentra detallada en el cuadro
\ref{tbl:pagerank_vs_indeg_wikipedia}. Nuevamente, In-Deg y PageRank dieron
resultados muy similares, pero no idénticos. M\'as aún, en In-Deg quedaron
varios nodos empatados, teniendo la misma cantidad de ejes entrantes. Pero esto
en PageRank no ocurri\'o, quedando definido un órden total.

\begin{table}[H]
    \centering
    \caption{\'Ordenes comparativos entre PageRank e In-Deg para Wikipedia}
    \label{tbl:pagerank_vs_indeg_wikipedia} 
    \setlength{\tabcolsep}{3pt}
    \begin{tabular}{|l|l||l|l|}
        \hline
        \multicolumn{2}{|c||}{PageRank} &\multicolumn{2}{c|}{In-Deg}\\
        \hline
        Puntaje & Nodo & Puntaje & Nodo\\
        \hline\hline
        0.196 & Matrix\_decomposition & 8 & Matrix\_decomposition\\
        0.165 & Numerical\_linear\_algebra & 7 & Numerical\_linear\_algebra\\
        0.125 & QR\_decomposition & 5 & QR\_decomposition\\
        0.106 & Numerical\_analysis & 4 & Numerical\_analysis\\
        0.105 & Singular\_value\_decomposition & 4 & LU\_decomposition\\
        0.098 & LU\_decomposition & 4 & Cholesky\_decomposition\\
        0.093 & Cholesky\_decomposition & 4 & Singular\_value\_decomposition\\
        0.057 & Eigendecomposition\_of\_a\_matrix & 2 & Eigendecomposition\_of\_a\_matrix\\
        0.050 & Matrix\_splitting & 2 & Matrix\_splitting\\
        \hline
    \end{tabular}
\end{table}

\par La diferencia entre \'ambos \'ordenes est\'a en la ubicaci\'on de
\emph{Singular value decomposition}. Mientras que en PageRank se encuentra en la
posici\'on 5, In-Deg lo lista en la 7ma ubicaci\'on. En el caso de In-Deg, la
determinaci\'on de la ubicaci\'on es determin\'istica, dependiendo del grado de
entrada del nodo que lo representa. Pero en nuestro ejemplo, vemos que existe un
empate con otros 3 nodos (como ya se ha explicado), con lo cual aqu\'i el orden
tambi\'en depende de la estabilidad y/o criterio de desempate que implemente
In-Deg. En nuestro caso, la implementaci\'on es estable, con lo cual se respeta
el orden inicial (o numeraci\'on) de los nodos. Por el otro lado, observando el
grafo podemos entender porque PageRank lo ubica en una posici\'on m\'as alta que
In-Deg: El nodo \emph{Numerical Linear Algebra}, uno de los de mayor puntaje (y
grado de entrada) tiene un link al nodo en cuesti\'on y a \emph{LU
decomposition}, pero no al resto de los valores que empatan en In-Deg. Por lo
visto en el experimento \ref{subsec:exp1}, sabemos que esto tiene un efecto de
subir el puntaje notoriamente en los nodos ''linkeados''. Luego, utilizando el
mismo razonamiento, observamos que \emph{Single value decomposition} queda por
encima de \emph{LU decomposition} por el voto/link de \emph{QR decomposition}.

\par Semánticamente, podemos observar que \textbf{en general}\footnote{\emph{QR}
queda por encima de \emph{Numerical Analysis}, creemos que esto se debe a la
morfología del grafo de referencias entre artículos en Wikipedia.} quedan
primeros en el ranking terminos mas \emph{generales} o \emph{abarcativos}; y a
medida que avanza el ranking se encuentran términos mas particulares.
Claramente, esta jerarqu\'ia de \emph{generalidad} es \'arbitraria para los
autores de este trabajo\footnote{¿Suma puntos hablar en tercera persona de
nosotros mismos? Very Scientific!}, aunque estimamos que habr\'a muy pocas
posibilidades de disenso al respecto para los temas representados por los nodos
del ejemplo.

\par Para evidenciar aún mas la diferencia entre los algoritmos de In-Deg y
PageRank decidimos alterar expl\'icitamente el grafo de conectividad de este
\'ultimo ejemplo, agregando aristas desde los 5 nodos peor puntuados hacia uno
de los últimos en el ranking In-Deg (\emph{Eigendecomposition of a matrix}).
Esto deberia aumentar drásticamente el rankeo del ultimo elemento en In-Deg ya
que aumentamos su grado de entrada en 5, pero no tanto asi en Pagerank donde la
''calidad'' de los votantes tiene una mayor injerencia. Los resultados de este
último experimento pueden verse en el cuadro
\ref{tbl:pagerank_vs_indeg_wikipedia_modificado}. 

\begin{table}[H]
    \centering
    \caption{\'Ordenes comparativos entre PageRank e In-Deg para Wikipedia con grafo alterado explícitamente}
    \label{tbl:pagerank_vs_indeg_wikipedia_modificado}
    \setlength{\tabcolsep}{3pt}
    \begin{tabular}{|l|l||l|l|}
        \hline
        \multicolumn{2}{|c||}{PageRank} &\multicolumn{2}{c|}{In-Deg}\\
        \hline
        Puntaje & Nodo & Puntaje & Nodo\\
        \hline\hline
        0.189 & Matrix\_decomposition & 8 & Matrix\_decomposition\\
        0.137 & Numerical\_linear\_algebra & 7 & Numerical\_linear\_algebra\\
        0.123 & Numerical\_analysis & 7 & Eigendecomposition\_of\_a\_matrix\\
        0.114 & Eigendecomposition\_of\_a\_matrix & 5 & QR\_decomposition\\
        0.108 & QR\_decomposition & 4 & Numerical\_analysis\\
        0.097 & Singular\_value\_decomposition & 4 & LU\_decomposition\\
        0.089 & LU\_decomposition & 4 & Cholesky\_decomposition\\
        0.087 & Cholesky\_decomposition & 4 & Singular\_value\_decomposition\\
        0.051 & Matrix\_splitting & 2 & Matrix\_splitting\\
        \hline
    \end{tabular}
\end{table}

\par Puede observarse que respecto a In-Deg el elemento con ejes entrantes
artificiales paso a ser el
tercero por su nuevo grado de entrada aumentado. En Pagerank, el elemento subió
de ranking pero quedó por debajo de Numerical\_analysis, lo cual en principio es
raro, ya que este item quedó con puntaje 4 en In-Deg. Si miramos el grafo,
veremos que, efectivamente, el hecho de que \emph{Numerical analysis} sea apuntado por
\emph{Matrix decomposition} y \emph{Numerical linear algebra}, que son los términos mas
importantes, le da mas potencia al elemento en cuestión que al elemento
\emph{Eigendecomposition of a matrix}, apuntado por los últimos 5 de la lista.

%\par Por último, vemos que en el último experimento se acentúa el orden
%\texttt{abarcativo} de los resultados respecto al dominio de los elementos.

\medskip
\par A lo largo de este experimento pudimos evidenciar las diferencias que
existen entre dos algoritmos de elaboraci\'on de rankings distintos: In-Deg y
PageRank. Esto, que no lo hab\'iamos podido exponer en el experimento previo,
nos demostr\'o el peso que PageRank le otorga a los links provenientes de
p\'aginas web con mejores puntajes, diferenciaci\'on que In-Deg no realiza.
M\'as a\'un, nos encontramos conque In-Deg parece ser tener m\'as posibilidades
de empate en su forma de ''rankear'' que PageRank, volvi\'endose dependiente del
criterio de desempate que se utilice y convirti\'endose en otra
faceta/problema a resolver, situaci\'on que puede ser despreciada en el caso de
PageRank (las probabilidades de empate ya son chicas para pocos nodos, y las
mismas son cada vez menores a mayor cantidad). Como dato de color, observamos
que los resultados devueltos por \emph{Google} son bastante distintos de los
obtenidos con PageRank, con lo cual queda claro que a pesar de haber sido este
m\'etodo un hito en la historia del motor de b\'usqueda, el mismo ya a
evolucionado much\'isimo en pocos a\~nos, lo que demuestra la importancia del
problema estudiado. 


%---------------------------------------------------------------

%\newpage
%\subsection*{Experimentos a Futuro}

\clearpage

\section{Resultados}
% TODO
% Deben incluir los resultados de los experimentos, utilizando el formato mas adecuado
% para  su  presentacion.   Deberan  especicar  claramente  a  que  experiencia  corresponde
% cada resultado.  No se incluiran aqu corridas de maquina.
\subsection{Performance}
\subsubsection{Gauss vs LU para una sola instancia variando discretizaciones}
\subsubsection{Gauss vs LU para muchas instancias variando discretizaciones}



\subsubsection{Evolución estimación de la isoterma y temperatura}
Se presentarán los resultados de los experimentos en el mismo orden en que fueron planteados en la sección de desarrollo. Se realizará el análisis de los mismos en este mismo apartado.
\begin{enumerate}
	\item \begin{itemize}
				\item \textbf{Temperaturas internas y externas:} aleatorias uniformes entre $[50\dots200]$ y $[1450\dots1550]$, pero fijas entre tests.
				\item \textbf{Radio interno:} 200
				\item \textbf{Radio externo:} 400
				\item \textbf{Cantidad radios:} $[5\dots100]$
				\item \textbf{Cantidad ángulos:} 100
				\item \textbf{Isoterma buscada:} 500
			\end{itemize}
Se adjunta con el trabajo práctico un video que expone la evolución del sistema mientras se incrementa la cantidad de radios. Expondremos estáticamente algunos frames, pero es conveniente ver el video primero. Se encuentra en la misma carpeta que el pdf. (variación\_radial\_isomap.mp4, variación\_radial\_heatmap.mp4).

\vspace{0.5cm}

  	\textbf{Variación de la estimación de la isoterma entre 5 y 6 radios de discretización}\\
	\includegraphics[scale=0.35]{experimentos1a_1b/evolucion_posicion_isoterma_temperatura/test2/test6_006_radios_inst_001_isomap.png}
	\includegraphics[scale=0.35]{experimentos1a_1b/evolucion_posicion_isoterma_temperatura/test2/test6_007_radios_inst_001_isomap.png}
	
  	\textbf{Variación de la temperatura entre 6 y 7 radios de discretización}\\
	\includegraphics[scale=0.35]{experimentos1a_1b/evolucion_posicion_isoterma_temperatura/test2/test6_006_radios_inst_001_heatmap.png}
	\includegraphics[scale=0.35]{experimentos1a_1b/evolucion_posicion_isoterma_temperatura/test2/test6_007_radios_inst_001_heatmap.png}

 	\textbf{Variación de la estimación de la isoterma entre 99 y 100 radios de discretización}\\
	\includegraphics[scale=0.35]{experimentos1a_1b/evolucion_posicion_isoterma_temperatura/test2/test6_099_radios_inst_001_isomap.png}
	\includegraphics[scale=0.35]{experimentos1a_1b/evolucion_posicion_isoterma_temperatura/test2/test6_100_radios_inst_001_isomap.png}
	
	\textbf{Variación de la temperatura entre 99 y 100 radios de discretización}\\
	\includegraphics[scale=0.35]{experimentos1a_1b/evolucion_posicion_isoterma_temperatura/test2/test6_099_radios_inst_001_heatmap.png}
	\includegraphics[scale=0.35]{experimentos1a_1b/evolucion_posicion_isoterma_temperatura/test2/test6_100_radios_inst_001_heatmap.png}

\vspace{0.5cm}

Se observa es que a medida que se aumenta la cantidad de radios de la discretización, la variación radial de la curva de la isoterma disminuye entre tests, es decir, se hace más fina la estimación, de forma tal que entre $i$ e $i+1$ radios la diferencia de la posición de la isoterma es menor a medida que $i$ crece. Para ver mejor esto se graficaron, para cada test de $i$ cantidad de radios de la discretización, el máximo y el promedio radial de la isoterma.

	\textbf{Evolución de la variación radial de la isoterma con cantidad creciente de radios}\\
	\includegraphics[scale=0.5]{experimentos1a_1b/evolucion_estimacion_seguridad_isoterma/100ang_5to100radios.png}\\

	\item \begin{itemize}
					\item \textbf{Temperaturas internas y externas:} constantes, 100 y 1500. Esto es para que tenga la misma solución cada test del experimento.
					\item \textbf{Radio interno:} 200
					\item \textbf{Radio externo:} 400
					\item \textbf{Cantidad radios:} 50
					\item \textbf{Cantidad ángulos:} $[5\dots50]$
					\item \textbf{Isoterma buscada:} 500
				\end{itemize}
	Se adjunta con el trabajo práctico un video que expone la evolución del sistema mientras se incrementa la cantidad de radios. Expondremos estáticamente algunos frames, pero es conveniente ver el video primero. Se encuentra en la misma carpeta que el pdf. (variación\_angular\_isomap.mp4, variación\_angular\_heatmap.mp4).

	\vspace{0.5cm}
	  	\textbf{Variación de la estimación de la isoterma entre 5 y 6 ángulos de discretización}\\
		\includegraphics[scale=0.35]{experimentos1a_1b/evolucion_posicion_isoterma_temperatura/variacion_angulos_radio_fijo_se_suaviza_isoterma/test10_050_radios_005_angulos_inst_001_isomap.png}
		\includegraphics[scale=0.35]{experimentos1a_1b/evolucion_posicion_isoterma_temperatura/variacion_angulos_radio_fijo_se_suaviza_isoterma/test10_050_radios_006_angulos_inst_001_isomap.png}

	  	\textbf{Variación de la temperatura entre 5 y 6 ángulos de discretización}\\
	  	\includegraphics[scale=0.35]{experimentos1a_1b/evolucion_posicion_isoterma_temperatura/variacion_angulos_radio_fijo_se_suaviza_isoterma/test10_050_radios_005_angulos_inst_001_heatmap.png}
		\includegraphics[scale=0.35]{experimentos1a_1b/evolucion_posicion_isoterma_temperatura/variacion_angulos_radio_fijo_se_suaviza_isoterma/test10_050_radios_006_angulos_inst_001_heatmap.png}	  	

	  	\textbf{Variación de la estimación de la isoterma entre 49 y 50 ángulos de discretización}\\
		\includegraphics[scale=0.35]{experimentos1a_1b/evolucion_posicion_isoterma_temperatura/variacion_angulos_radio_fijo_se_suaviza_isoterma/test10_050_radios_049_angulos_inst_001_isomap.png}
		\includegraphics[scale=0.35]{experimentos1a_1b/evolucion_posicion_isoterma_temperatura/variacion_angulos_radio_fijo_se_suaviza_isoterma/test10_050_radios_050_angulos_inst_001_isomap.png}

		\textbf{Variación de la temperatura entre 49 y 50 ángulos de discretización}\\
	  	\includegraphics[scale=0.35]{experimentos1a_1b/evolucion_posicion_isoterma_temperatura/variacion_angulos_radio_fijo_se_suaviza_isoterma/test10_050_radios_049_angulos_inst_001_heatmap.png}
		\includegraphics[scale=0.35]{experimentos1a_1b/evolucion_posicion_isoterma_temperatura/variacion_angulos_radio_fijo_se_suaviza_isoterma/test10_050_radios_050_angulos_inst_001_heatmap.png}

\vspace{0.5cm}

Aquí el radio es el mismo, pero se gana en precisión al tener más ángulos por no tener que linealizar la posición de la isoterma angularmente. Nuevamente, la posición entre dos tests consecutivos se estabiliza al aumentar la cantidad de ángulos. Tambien se observa que al cambiar el $\Delta_\theta$ los ángulos entre tests consecutivos no son los mismos.

	\item \begin{itemize}
						\item \textbf{Temperaturas internas y externas:} constantes, 100 y 1500. Esto es para que tenga la misma solución cada test del experimento.
						\item \textbf{Radio interno:} 200
						\item \textbf{Radio externo:} 400
						\item \textbf{Cantidad radios:} $[15\dots60]$
						\item \textbf{Cantidad ángulos:} $[15\dots60]$
						\item \textbf{Isoterma buscada:} 500
					\end{itemize}
	Se adjunta con el trabajo práctico un video que expone la evolución del sistema mientras se incrementa la cantidad de radios. Expondremos estáticamente algunos frames, pero es conveniente ver el video primero. Se encuentra en la misma carpeta que el pdf. (variación\_doble\_isomap.mp4, variación\_doble\_heatmap.mp4).

	\vspace{0.5cm}
	  	\textbf{Variación de la estimación de la isoterma entre 15 y 16 radios, ángulos de discretización}\\
		\includegraphics[scale=0.35]{experimentos1a_1b/evolucion_posicion_isoterma_temperatura/variacion_radios_angulos_se_reduce_diferencia_radial/test11_testord_001_inst_001_isomap.png}
		\includegraphics[scale=0.35]{experimentos1a_1b/evolucion_posicion_isoterma_temperatura/variacion_radios_angulos_se_reduce_diferencia_radial/test11_testord_002_inst_001_isomap.png}

	  	\textbf{Variación de la temperatura entre 59 y 60 radios, ángulos de discretización}\\
	  	\includegraphics[scale=0.35]{experimentos1a_1b/evolucion_posicion_isoterma_temperatura/variacion_radios_angulos_se_reduce_diferencia_radial/test11_testord_001_inst_001_heatmap.png}
		\includegraphics[scale=0.35]{experimentos1a_1b/evolucion_posicion_isoterma_temperatura/variacion_radios_angulos_se_reduce_diferencia_radial/test11_testord_002_inst_001_heatmap.png}

	  	\textbf{Variación de la estimación de la isoterma entre 15 y 16 radios, ángulos de discretización}\\
		\includegraphics[scale=0.35]{experimentos1a_1b/evolucion_posicion_isoterma_temperatura/variacion_radios_angulos_se_reduce_diferencia_radial/test11_testord_045_inst_001_isomap.png}
		\includegraphics[scale=0.35]{experimentos1a_1b/evolucion_posicion_isoterma_temperatura/variacion_radios_angulos_se_reduce_diferencia_radial/test11_testord_046_inst_001_isomap.png}

		\textbf{Variación de la temperatura entre 59 y 60 radios, ángulos de discretización}\\
	  	\includegraphics[scale=0.35]{experimentos1a_1b/evolucion_posicion_isoterma_temperatura/variacion_radios_angulos_se_reduce_diferencia_radial/test11_testord_045_inst_001_heatmap.png}
		\includegraphics[scale=0.35]{experimentos1a_1b/evolucion_posicion_isoterma_temperatura/variacion_radios_angulos_se_reduce_diferencia_radial/test11_testord_046_inst_001_heatmap.png}

\vspace{0.5cm}

En este último ejemplo ocurren ambos fenomenos al mismo tiempo, hay una variación radial menor a medida que crecen los radios y la curva se suaviza al aumentar los ángulos.

\end{enumerate}

\vspace{0.5cm}

Efectivamente, podemos concluir que mientras más fina sea la discretización, se obtendrán resultados más \texttt{estables y confiables} acerca de la estimación. Uno de los motivos es porque habrá menos puntos para interpolar en la posición de la isoterma y el otro porque se tiene más informacion de la temperatura de la pared del horno.

\subsubsection{Estimación de estabilidad de la pared del horno}
Para este experimento utilizaremos discretizaciones finas, ya vimos en los experimentos anteriores que esto provee mayor confiabilidad en las estimaciones, para discretizaciones más gruesas las estimaciones de seguridad serán menos exactas.
\vspace{0.5cm}

\begin{itemize}
	\item \textbf{Temperaturas internas:} $[200, 500, 750]$
	\item \textbf{Temperaturas externas:} $[1450\dots1550]$ aleatorias uniformes.
	\item \textbf{Radio interno:} 200
	\item \textbf{Radio externo:} 400
	\item \textbf{Cantidad radios:} 30
	\item \textbf{Cantidad ángulos:} 30
	\item \textbf{Isoterma buscada:} 500
\end{itemize}

	\textbf{Temperaturas externas: } 200\\
	\includegraphics[scale=0.35]{experimentos1a_1b/evolucion_isoterma_cambios_temperatura_varias_discretizaciones/test21_030_radios_030_angulos_inst_001_heatmap.png}
	\includegraphics[scale=0.35]{experimentos1a_1b/evolucion_isoterma_cambios_temperatura_varias_discretizaciones/test21_030_radios_030_angulos_inst_001_isomap.png}

	\textbf{Temperaturas externas: } 500\\
	\includegraphics[scale=0.35]{experimentos1a_1b/evolucion_isoterma_cambios_temperatura_varias_discretizaciones/test22_030_radios_030_angulos_inst_001_heatmap.png}
	\includegraphics[scale=0.35]{experimentos1a_1b/evolucion_isoterma_cambios_temperatura_varias_discretizaciones/test22_030_radios_030_angulos_inst_001_isomap.png}

	\textbf{Temperaturas externas: } 750\\
	\includegraphics[scale=0.35]{experimentos1a_1b/evolucion_isoterma_cambios_temperatura_varias_discretizaciones/test23_030_radios_030_angulos_inst_001_heatmap.png}
	\includegraphics[scale=0.35]{experimentos1a_1b/evolucion_isoterma_cambios_temperatura_varias_discretizaciones/test23_030_radios_030_angulos_inst_001_isomap.png}

\vspace{0.5cm}

La escala de color de los mapas de temperaturas se hace en base al mínimo y máximo de la muestra, es por esto que la variación de temperaturas externas no provee una variación en los colores de los radios del borde. \\
Asimismo, se ve que la posición de la isoterma en los casos $200, 500$ se posiciona según lo esperado dentro de la pared del horno. Mientras que en el caso $750$, al ser $750 > 500$, por convención posicionamos la isoterma en $R_i - \epsilon$.

\vspace{0.5cm}

En la siguiente tabla se puede observar que las métricas de seguridad nos dan una pauta acerca de la posición promedio y maxima de la isoterma dentro de la pared del horno. Si utilizamos $\gamma_0 = 0.75$ como limite de seguridad, podemos establecer conclusiones acerca de la seguridad: en el caso donde hay 200 grados en el exterior es seguro, mientras que 500 grados, no lo es.

\begin{center}
	\begin{tabular}{| c | c | c | c |}
	 	\hline
	 	$T_e$ & $\Delta_{max_{iso500}}$ & $\Delta_{prom_{iso500}}$ & Seguro bajo $\gamma_0 = 0.75$\\
	 	\hline			
		200 & 0.709145 & 0.71037 & Si\\
		\hline
		500 & 0.965517 & 0.965517 & No\\
		\hline  
	\end{tabular}
\end{center}

\clearpage

%\section{Discusión}
%% TODO
% Se incluira aqu un analisis de los resultados obtenidos en la seccion anterior (se analizara
% su  validez,  coherencia,  etc.).   Deben  analizarse  como  mnimo  los tems  pedidos  en  el
% enunciado.  No es aceptable decir que los resultados fueron los esperados", sin hacer
% clara referencia a la teora a la cual se ajustan.  Ademas, se deben mencionar los resul-
% tados interesantes y los casos patologicos" encontrados.


%\clearpage

\section{Conclusiones}
% TODO
% Esta seccion debe contener las conclusiones generales del trabajo.  Se deben mencionar
% las  relaciones  de  la  discusion  sobre  las  que  se  tiene  certeza,  junto  con  comentarios
% y  observaciones  generales  aplicables  a  todo  el  proceso.   Mencionar  tambien  posibles
% extensiones a los metodos, experimentos que hayan quedado pendientes, etc.

\clearpage

\section{Apéndice}
\subsection{Apéndice A: Enunciado}\label{enunciado}
  \input{tp1.tex}
\clearpage
\subsection{Apéndice B: Código relevante}
  \lstinputlisting[language=C++, caption=Eliminacion gaussiana]{gaussiana.ctex}
\lstinputlisting[language=C++, caption=Factorizacion LU]{lu.ctex}

%\section{Referencias}
\end{document}
