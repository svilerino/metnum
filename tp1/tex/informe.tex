\documentclass[10pt, a4paper,english,spanish]{article}
\usepackage[spanish]{babel}
\parindent = 0 pt
\parskip = 11 pt
\usepackage[width=15.5cm, left=3cm, top=2.5cm, height= 24.5cm]{geometry}

\usepackage{amsmath}
\usepackage{amsfonts}
\usepackage{amssymb}
\usepackage[utf8]{inputenc}
\usepackage{graphicx}
\usepackage{verbatim}
\usepackage{color}
\usepackage{amsmath}
\usepackage{graphicx}
\usepackage[colorinlistoftodos]{todonotes}
\usepackage{multicol}
\usepackage{makeidx}
\usepackage{hyperref}
\usepackage{caption}
\usepackage{amsfonts}
\usepackage{amssymb}
\usepackage{amsmath}
\usepackage[utf8]{inputenc}
\usepackage{verbatim}
\usepackage{listings}
\usepackage{algpseudocode}
\usepackage{courier}
\usepackage{enumitem}
\usepackage{placeins}
\usepackage{booktabs}
% \usepackage[margin=1in]{geometry}


\newcommand{\BigO}[1]{\ensuremath{\operatorname{O}\bigl(#1\bigr)}}
\newtheorem{proposition}{Proposici\'on}
\lstset{language=C++, showstringspaces=false, tabsize=2, breaklines=true, title=\lstname}

\makeindex

\begin{document}
\newgeometry{margin=2cm}
\pagenumbering{gobble}
\raggedleft
\includegraphics[width=8cm]{caratula/logo1.jpg}\\

\raggedright
\vspace{3cm}
{\Huge \bfseries Trabajo Práctico 1 \\ Cálculo de isotermas en sectores circulares}
\rule{\textwidth}{0.02in}
\large Jueves 3 de septiembre de 2015 \hfill Métodos Numéricos
\vspace{1.5cm}

\normalsize
\begin{tabular}{|l@{\hspace{5ex}}c@{\hspace{5ex}}l|}
        \hline
        \rule{0pt}{1.2em}Integrante & LU & Correo electrónico\\[0.2em]
        \hline
        \rule{0pt}{1.2em} Nahuel Lascano  & 476/11 &\tt laski.nahuel@gmail.com\\[0.2em]
        \rule{0pt}{1.2em} XXXX & XXXX &\tt XXXX\\[0.2em]
        \rule{0pt}{1.2em} XXXX & XXXX &\tt XXXX\\[0.2em]
        \hline
\end{tabular}

\medskip
En este trabajo aplicamos dos métodos de resolución de sistemas de ecuaciones lineales (factorización LU y eliminación gaussiana) para el cálculo de isotermas de sectores circulares, dadas las temperaturas de las circunferencias interior y exterior. % TODO: agregar algo breve con conclusiones

\medskip
Palabras clave: factorización LU, eliminación gaussiana, sistemas de ecuaciones lineales, matriz banda

\vspace{1.0cm}
\raggedright

\begin{multicols}{2}
\includegraphics[width=8cm]{caratula/logo-uba.png}

\columnbreak
\vspace*{4.5cm}
\raggedleft
\textbf{Facultad de Ciencias Exactas y Naturales}\\
Universidad de Buenos Aires\\
\small
Ciudad Universitaria - (Pabellon I/Planta Baja)\\
Intendente G\"uiraldes 2160 - C1428EGA\\
Ciudad Autonoma de Buenos Aires - Rep. Argentina\\
Tel/Fax: (54 11) 4576-3359\\
http://www.fcen.uba.ar
\end{multicols}

\restoregeometry

\clearpage

\pagenumbering{arabic}

\tableofcontents

\vspace{3cm}

\clearpage

\setlength{\parindent}{10pt}

\section{Introducción teórica}
En el presente trabajo se intentó evaluar computacionalmente la seguridad térmica de un horno circular. El problema presentado consistía en estimar el riesgo que corre el mismo de fracturarse por efecto de la elevada temperatura. Dicho de otro modo, dadas las temperaturas de las paredes internas y externas del horno (obtenidas a través de sensores) se quiere estimar la ubicación de la isoterma de 500$^{o}$C, cuya cercanía a la pared externa es un indicador de la peligrosidad de la estructura.

Dado un corte transversal del horno podemos definir $r_i$ y $r_e$ como los radios de la pared interna y externa, ambos circualres. Nos interesa analizar la temperatura de los puntos entre ellas. Para referirnos a los puntos de dicha corona circular utilizaremos coordenadas polares, por lo que cada punto $P$ quedará definido por un radio $r$ y un ángulo $\theta$. Si llamamos $T(r,\theta)$ a la temperatura del punto $P_{r, \theta}$, podemos utilizar la ecuación del calor de Laplace para encontrar el estado de equilibrio del sistema:

\begin{equation}\label{calor-continuo}
\frac{\partial^2T(r,\theta)}{\partial r^2}+\frac{1}{r}\frac{\partial T(r,\theta)}{\partial r}+\frac{1}{r^2}\frac{\partial^2T(r,\theta)}{\partial \theta^2} = 0 
\end{equation}

la cual debe cumplirse para todos los puntos internos del horno.

Por estar trabajando con aritmética finita, usamos una discretización de los puntos que nos interesa analizar (todos los pertenecientes a la pared del horno, que forman una corona circular) y discretizamos asimismo la ecuación \ref{calor-continuo}. De esa manera arribamos a un sistema de ecuaciones lineal que podemos representar en su versión matricial como $Ax=b$ y que intentaremos resolver usando dos métodos: la Eliminación Gaussiana y Factorización LU.

Nos interesa comparar estos dos métodos por el tiempo que les lleva resolver un único sistema $Ax=b$ en función de la granularidad de la discretización. Posteriormente podemos complejizar el problema suponiendo que tenemos múltiples mediciones para las temperaturas de las paredes (a lo largo del tiempo), por lo que debemos comparar su performance a la hora de resolver mútiples sistemas $Ax=b_i$ con diferentes vectores $b_i$.

La solución del sistema de ecuaciones nos permitirá conocer el valor de la funcion $T$ en los puntos de la discretización elegida, pero es posible que ninguno de ellos coincida con el valor de la isoterma buscada. Otro objetivo del trabajo será evaluar diferentes formas de estimar esa isoterma y compararlas variando la granularidad de la discretización.


\clearpage

\section{Desarrollo}
% TODO

% Deben  explicarse  los  metodos  numericos  que  utilizaron  y  su  aplicacion  al  problema
% concreto  involucrado  en  el  trabajo  practico.   Se  deben  mencionar  los  pasos  que  si-
% guieron para implementar los algoritmos, las dicultades que fueron encontrando y la
% descripcion de como las fueron resolviendo.  Explicar tambien como fueron planteadas
% y realizadas las mediciones experimentales.  Los ensayos fallidos, hipotesis y conjeturas
% equivocadas,  experimentos  y  metodos  malogrados  deben  gurar  en  esta  seccion,  con
% una breve explicacion de los motivos de estas fallas (en caso de ser conocidas).

De la discretización de la ecuación del calor provista por el informe resulta una nueva ecuación que nos va a servir para armar nuestro sistema discreto:

\begin{equation}\label{calor}
\frac{t_{j-1,k}-2t_{jk}+t_{j+1,k}}{(\Delta r)^2}+\frac{1}{r}\frac{t_{j,k}-t_{j-1,k}}{\Delta r}+\frac{1}{r^2}\frac{t_{j,k-1}-2t_{jk}+t_{j,k+1}}{(\Delta \theta)^2} = 0 
\end{equation}

Para poder armar el sistema $Ax=b$ equivalente, es necesario:
\begin{itemize}
 \item
    Extraer los factores que multiplican a cada una de las cinco incógnitas: $t_{j-1,k}$; $t_{j,k}$; $t_{j+1,k}$; $t_{j,k-1}$ y $t_{j,k+1}$.

    Estos se obtienen de la ecuación \ref{calor}.
    \begin{align*}
        &t_{j-1, k}(\frac{1}{(\Delta r)^2} - \frac{1}{r_j \Delta r}) \\
        &t_{j, k}(\frac{-2}{(\Delta r)^2} + \frac{1}{r_j \Delta r} - \frac{2}{{r_j}^2 (\Delta r)^2}) \\
        &t_{j+1, k}(\frac{1}{(\Delta r)^2}) \\
        &t_{j, k-1}(\frac{1}{{r_j}^2(\Delta \theta)^2}) \\
        &t_{j, k+1}(\frac{1}{{r_j}^2(\Delta \theta)^2})
    \end{align*}
    Para simplificar, en adelante llamaremos $F_{j,k}$ al factor que multiplica a la incógnita $t_{j,k}$ y $Ft_{j,k}$ a $F_{j,k}*t_{j,k}$.
 \item
    Analizar los ``casos borde'': aquellos puntos donde la ecuación \ref{calor} no vale.
    
    Para evitar confusiones de variables, tomaremos $\theta_0 = 0$ como el menor valor posible de $\theta$ y $\theta_{n-1}$ como el mayor, pues vale $(r_j, \theta_n) = (r_j, \theta_0)$ para cualquier $j$. Los casos interesantes para valores de $j, k$ entonces son:
    \begin{enumerate}
     \item La pared interior del horno ($j = 0$; $k = 0, ..., n-1$). La ecuación en esos casos es $t_{0, k} = T_i(\theta_k)$.
     \item La pared exterior del horno ($j = m$; $k = 0, ..., n-1$). La ecuación en esos casos es $t_{m, k} = T_e(\theta_k)$.
     \item El valor mínimo de $\theta$ ($j = 0, ..., m$; $k = 0$). Se debe reemplazar $t_{j, k-1}$ por $t_{j, n-1}$ en todas las ecuaciones correspondientes.
     \item El valor máximo de $\theta$ ($j = 0, ..., m$; $k = n-1$). Se debe reemplazar $t_{j, k+1}$ por $t_{j, 0}$ en todas las ecuaciones correspondientes.
    \end{enumerate}
    Estos últimos reemplazos se pueden resumir en que todo punto $(j, k)$ sea $(j, k$ mod $n)$.
 \item
    Combinar los puntos anteriores para plantear sistema de ecuaciones a resolver:
    \begin{align*}
    &t_{0, k} = T_i(\theta_k)                                           &\forall k = 0, ..., n-1  \\
    &t_{m, k} = T_e(\theta_k)                                           &\forall k = 0, ..., n-1  \\
    &Ft_{j-1,k} + Ft_{j,k} + Ft_{j+1,k} + Ft_{j,k-1} + Ft_{j,k+1} = 0  &\forall j=1, ..., m-1; k = 1, ... , n-2 \\
    &Ft_{j-1,0} + Ft_{j,0} + Ft_{j+1,0} + Ft_{j,n-1} + Ft_{j,1} = 0    &\forall j=1, ..., m-1 \\
    &Ft_{j-1,n-1} + Ft_{j,n-1} + Ft_{j+1,n-1} + Ft_{j,n-2} + Ft_{j,0} = 0    &\forall j=1, ..., m-1 \\
    \end{align*}

    Del mismo podemos obtener fácilmente la matriz $A$ (que tendrá 5 valores no nulos por fila a lo sumo) y el vector $b$ (que será nulo en todas sus componentes salvo aquellas correspondientes a $j=0$ y $j=m$).
  \item
    Resta pensar un orden para las incógnitas que permita asegurar que la matriz resultante sea $banda$. Sobre el proceso llevado adelante para su elección hablaremos en la sección \ref{banda}. El orden elegido fue:
    
    $$ (0,0); (0,1); ... ; (j,n-1); (j+1,0); (j,1); ... ; (m, n-1)$$
\end{itemize}

\clearpage

\section{Resultados}
% TODO
% Deben incluir los resultados de los experimentos, utilizando el formato mas adecuado
% para  su  presentacion.   Deberan  especicar  claramente  a  que  experiencia  corresponde
% cada resultado.  No se incluiran aqu corridas de maquina.
\subsection{Performance}
\subsubsection{Gauss vs LU para una sola instancia variando discretizaciones}
\subsubsection{Gauss vs LU para muchas instancias variando discretizaciones}



\subsubsection{Evolución estimación de la isoterma y temperatura}
Se presentarán los resultados de los experimentos en el mismo orden en que fueron planteados en la sección de desarrollo. Se realizará el análisis de los mismos en este mismo apartado.
\begin{enumerate}
	\item \begin{itemize}
				\item \textbf{Temperaturas internas y externas:} aleatorias uniformes entre $[50\dots200]$ y $[1450\dots1550]$, pero fijas entre tests.
				\item \textbf{Radio interno:} 200
				\item \textbf{Radio externo:} 400
				\item \textbf{Cantidad radios:} $[5\dots100]$
				\item \textbf{Cantidad ángulos:} 100
				\item \textbf{Isoterma buscada:} 500
			\end{itemize}
Se adjunta con el trabajo práctico un video que expone la evolución del sistema mientras se incrementa la cantidad de radios. Expondremos estáticamente algunos frames, pero es conveniente ver el video primero. Se encuentra en la misma carpeta que el pdf. (variación\_radial\_isomap.mp4, variación\_radial\_heatmap.mp4).

\vspace{0.5cm}

  	\textbf{Variación de la estimación de la isoterma entre 5 y 6 radios de discretización}\\
	\includegraphics[scale=0.35]{experimentos1a_1b/evolucion_posicion_isoterma_temperatura/test2/test6_006_radios_inst_001_isomap.png}
	\includegraphics[scale=0.35]{experimentos1a_1b/evolucion_posicion_isoterma_temperatura/test2/test6_007_radios_inst_001_isomap.png}
	
  	\textbf{Variación de la temperatura entre 6 y 7 radios de discretización}\\
	\includegraphics[scale=0.35]{experimentos1a_1b/evolucion_posicion_isoterma_temperatura/test2/test6_006_radios_inst_001_heatmap.png}
	\includegraphics[scale=0.35]{experimentos1a_1b/evolucion_posicion_isoterma_temperatura/test2/test6_007_radios_inst_001_heatmap.png}

 	\textbf{Variación de la estimación de la isoterma entre 99 y 100 radios de discretización}\\
	\includegraphics[scale=0.35]{experimentos1a_1b/evolucion_posicion_isoterma_temperatura/test2/test6_099_radios_inst_001_isomap.png}
	\includegraphics[scale=0.35]{experimentos1a_1b/evolucion_posicion_isoterma_temperatura/test2/test6_100_radios_inst_001_isomap.png}
	
	\textbf{Variación de la temperatura entre 99 y 100 radios de discretización}\\
	\includegraphics[scale=0.35]{experimentos1a_1b/evolucion_posicion_isoterma_temperatura/test2/test6_099_radios_inst_001_heatmap.png}
	\includegraphics[scale=0.35]{experimentos1a_1b/evolucion_posicion_isoterma_temperatura/test2/test6_100_radios_inst_001_heatmap.png}

\vspace{0.5cm}

Se observa es que a medida que se aumenta la cantidad de radios de la discretización, la variación radial de la curva de la isoterma disminuye entre tests, es decir, se hace más fina la estimación, de forma tal que entre $i$ e $i+1$ radios la diferencia de la posición de la isoterma es menor a medida que $i$ crece. Para ver mejor esto se graficaron, para cada test de $i$ cantidad de radios de la discretización, el máximo y el promedio radial de la isoterma.

	\textbf{Evolución de la variación radial de la isoterma con cantidad creciente de radios}\\
	\includegraphics[scale=0.5]{experimentos1a_1b/evolucion_estimacion_seguridad_isoterma/100ang_5to100radios.png}\\

	\item \begin{itemize}
					\item \textbf{Temperaturas internas y externas:} constantes, 100 y 1500. Esto es para que tenga la misma solución cada test del experimento.
					\item \textbf{Radio interno:} 200
					\item \textbf{Radio externo:} 400
					\item \textbf{Cantidad radios:} 50
					\item \textbf{Cantidad ángulos:} $[5\dots50]$
					\item \textbf{Isoterma buscada:} 500
				\end{itemize}
	Se adjunta con el trabajo práctico un video que expone la evolución del sistema mientras se incrementa la cantidad de radios. Expondremos estáticamente algunos frames, pero es conveniente ver el video primero. Se encuentra en la misma carpeta que el pdf. (variación\_angular\_isomap.mp4, variación\_angular\_heatmap.mp4).

	\vspace{0.5cm}
	  	\textbf{Variación de la estimación de la isoterma entre 5 y 6 ángulos de discretización}\\
		\includegraphics[scale=0.35]{experimentos1a_1b/evolucion_posicion_isoterma_temperatura/variacion_angulos_radio_fijo_se_suaviza_isoterma/test10_050_radios_005_angulos_inst_001_isomap.png}
		\includegraphics[scale=0.35]{experimentos1a_1b/evolucion_posicion_isoterma_temperatura/variacion_angulos_radio_fijo_se_suaviza_isoterma/test10_050_radios_006_angulos_inst_001_isomap.png}

	  	\textbf{Variación de la temperatura entre 5 y 6 ángulos de discretización}\\
	  	\includegraphics[scale=0.35]{experimentos1a_1b/evolucion_posicion_isoterma_temperatura/variacion_angulos_radio_fijo_se_suaviza_isoterma/test10_050_radios_005_angulos_inst_001_heatmap.png}
		\includegraphics[scale=0.35]{experimentos1a_1b/evolucion_posicion_isoterma_temperatura/variacion_angulos_radio_fijo_se_suaviza_isoterma/test10_050_radios_006_angulos_inst_001_heatmap.png}	  	

	  	\textbf{Variación de la estimación de la isoterma entre 49 y 50 ángulos de discretización}\\
		\includegraphics[scale=0.35]{experimentos1a_1b/evolucion_posicion_isoterma_temperatura/variacion_angulos_radio_fijo_se_suaviza_isoterma/test10_050_radios_049_angulos_inst_001_isomap.png}
		\includegraphics[scale=0.35]{experimentos1a_1b/evolucion_posicion_isoterma_temperatura/variacion_angulos_radio_fijo_se_suaviza_isoterma/test10_050_radios_050_angulos_inst_001_isomap.png}

		\textbf{Variación de la temperatura entre 49 y 50 ángulos de discretización}\\
	  	\includegraphics[scale=0.35]{experimentos1a_1b/evolucion_posicion_isoterma_temperatura/variacion_angulos_radio_fijo_se_suaviza_isoterma/test10_050_radios_049_angulos_inst_001_heatmap.png}
		\includegraphics[scale=0.35]{experimentos1a_1b/evolucion_posicion_isoterma_temperatura/variacion_angulos_radio_fijo_se_suaviza_isoterma/test10_050_radios_050_angulos_inst_001_heatmap.png}

\vspace{0.5cm}

Aquí el radio es el mismo, pero se gana en precisión al tener más ángulos por no tener que linealizar la posición de la isoterma angularmente. Nuevamente, la posición entre dos tests consecutivos se estabiliza al aumentar la cantidad de ángulos. Tambien se observa que al cambiar el $\Delta_\theta$ los ángulos entre tests consecutivos no son los mismos.

	\item \begin{itemize}
						\item \textbf{Temperaturas internas y externas:} constantes, 100 y 1500. Esto es para que tenga la misma solución cada test del experimento.
						\item \textbf{Radio interno:} 200
						\item \textbf{Radio externo:} 400
						\item \textbf{Cantidad radios:} $[15\dots60]$
						\item \textbf{Cantidad ángulos:} $[15\dots60]$
						\item \textbf{Isoterma buscada:} 500
					\end{itemize}
	Se adjunta con el trabajo práctico un video que expone la evolución del sistema mientras se incrementa la cantidad de radios. Expondremos estáticamente algunos frames, pero es conveniente ver el video primero. Se encuentra en la misma carpeta que el pdf. (variación\_doble\_isomap.mp4, variación\_doble\_heatmap.mp4).

	\vspace{0.5cm}
	  	\textbf{Variación de la estimación de la isoterma entre 15 y 16 radios, ángulos de discretización}\\
		\includegraphics[scale=0.35]{experimentos1a_1b/evolucion_posicion_isoterma_temperatura/variacion_radios_angulos_se_reduce_diferencia_radial/test11_testord_001_inst_001_isomap.png}
		\includegraphics[scale=0.35]{experimentos1a_1b/evolucion_posicion_isoterma_temperatura/variacion_radios_angulos_se_reduce_diferencia_radial/test11_testord_002_inst_001_isomap.png}

	  	\textbf{Variación de la temperatura entre 59 y 60 radios, ángulos de discretización}\\
	  	\includegraphics[scale=0.35]{experimentos1a_1b/evolucion_posicion_isoterma_temperatura/variacion_radios_angulos_se_reduce_diferencia_radial/test11_testord_001_inst_001_heatmap.png}
		\includegraphics[scale=0.35]{experimentos1a_1b/evolucion_posicion_isoterma_temperatura/variacion_radios_angulos_se_reduce_diferencia_radial/test11_testord_002_inst_001_heatmap.png}

	  	\textbf{Variación de la estimación de la isoterma entre 15 y 16 radios, ángulos de discretización}\\
		\includegraphics[scale=0.35]{experimentos1a_1b/evolucion_posicion_isoterma_temperatura/variacion_radios_angulos_se_reduce_diferencia_radial/test11_testord_045_inst_001_isomap.png}
		\includegraphics[scale=0.35]{experimentos1a_1b/evolucion_posicion_isoterma_temperatura/variacion_radios_angulos_se_reduce_diferencia_radial/test11_testord_046_inst_001_isomap.png}

		\textbf{Variación de la temperatura entre 59 y 60 radios, ángulos de discretización}\\
	  	\includegraphics[scale=0.35]{experimentos1a_1b/evolucion_posicion_isoterma_temperatura/variacion_radios_angulos_se_reduce_diferencia_radial/test11_testord_045_inst_001_heatmap.png}
		\includegraphics[scale=0.35]{experimentos1a_1b/evolucion_posicion_isoterma_temperatura/variacion_radios_angulos_se_reduce_diferencia_radial/test11_testord_046_inst_001_heatmap.png}

\vspace{0.5cm}

En este último ejemplo ocurren ambos fenomenos al mismo tiempo, hay una variación radial menor a medida que crecen los radios y la curva se suaviza al aumentar los ángulos.

\end{enumerate}

\vspace{0.5cm}

Efectivamente, podemos concluir que mientras más fina sea la discretización, se obtendrán resultados más \texttt{estables y confiables} acerca de la estimación. Uno de los motivos es porque habrá menos puntos para interpolar en la posición de la isoterma y el otro porque se tiene más informacion de la temperatura de la pared del horno.

\subsubsection{Estimación de estabilidad de la pared del horno}
Para este experimento utilizaremos discretizaciones finas, ya vimos en los experimentos anteriores que esto provee mayor confiabilidad en las estimaciones, para discretizaciones más gruesas las estimaciones de seguridad serán menos exactas.
\vspace{0.5cm}

\begin{itemize}
	\item \textbf{Temperaturas internas:} $[200, 500, 750]$
	\item \textbf{Temperaturas externas:} $[1450\dots1550]$ aleatorias uniformes.
	\item \textbf{Radio interno:} 200
	\item \textbf{Radio externo:} 400
	\item \textbf{Cantidad radios:} 30
	\item \textbf{Cantidad ángulos:} 30
	\item \textbf{Isoterma buscada:} 500
\end{itemize}

	\textbf{Temperaturas externas: } 200\\
	\includegraphics[scale=0.35]{experimentos1a_1b/evolucion_isoterma_cambios_temperatura_varias_discretizaciones/test21_030_radios_030_angulos_inst_001_heatmap.png}
	\includegraphics[scale=0.35]{experimentos1a_1b/evolucion_isoterma_cambios_temperatura_varias_discretizaciones/test21_030_radios_030_angulos_inst_001_isomap.png}

	\textbf{Temperaturas externas: } 500\\
	\includegraphics[scale=0.35]{experimentos1a_1b/evolucion_isoterma_cambios_temperatura_varias_discretizaciones/test22_030_radios_030_angulos_inst_001_heatmap.png}
	\includegraphics[scale=0.35]{experimentos1a_1b/evolucion_isoterma_cambios_temperatura_varias_discretizaciones/test22_030_radios_030_angulos_inst_001_isomap.png}

	\textbf{Temperaturas externas: } 750\\
	\includegraphics[scale=0.35]{experimentos1a_1b/evolucion_isoterma_cambios_temperatura_varias_discretizaciones/test23_030_radios_030_angulos_inst_001_heatmap.png}
	\includegraphics[scale=0.35]{experimentos1a_1b/evolucion_isoterma_cambios_temperatura_varias_discretizaciones/test23_030_radios_030_angulos_inst_001_isomap.png}

\vspace{0.5cm}

La escala de color de los mapas de temperaturas se hace en base al mínimo y máximo de la muestra, es por esto que la variación de temperaturas externas no provee una variación en los colores de los radios del borde. \\
Asimismo, se ve que la posición de la isoterma en los casos $200, 500$ se posiciona según lo esperado dentro de la pared del horno. Mientras que en el caso $750$, al ser $750 > 500$, por convención posicionamos la isoterma en $R_i - \epsilon$.

\vspace{0.5cm}

En la siguiente tabla se puede observar que las métricas de seguridad nos dan una pauta acerca de la posición promedio y maxima de la isoterma dentro de la pared del horno. Si utilizamos $\gamma_0 = 0.75$ como limite de seguridad, podemos establecer conclusiones acerca de la seguridad: en el caso donde hay 200 grados en el exterior es seguro, mientras que 500 grados, no lo es.

\begin{center}
	\begin{tabular}{| c | c | c | c |}
	 	\hline
	 	$T_e$ & $\Delta_{max_{iso500}}$ & $\Delta_{prom_{iso500}}$ & Seguro bajo $\gamma_0 = 0.75$\\
	 	\hline			
		200 & 0.709145 & 0.71037 & Si\\
		\hline
		500 & 0.965517 & 0.965517 & No\\
		\hline  
	\end{tabular}
\end{center}

\clearpage

\section{Discusión}
% TODO
% Se incluira aqu un analisis de los resultados obtenidos en la seccion anterior (se analizara
% su  validez,  coherencia,  etc.).   Deben  analizarse  como  mnimo  los tems  pedidos  en  el
% enunciado.  No es aceptable decir que los resultados fueron los esperados", sin hacer
% clara referencia a la teora a la cual se ajustan.  Ademas, se deben mencionar los resul-
% tados interesantes y los casos patologicos" encontrados.


\clearpage

\section{Conclusiones}
% TODO
% Esta seccion debe contener las conclusiones generales del trabajo.  Se deben mencionar
% las  relaciones  de  la  discusion  sobre  las  que  se  tiene  certeza,  junto  con  comentarios
% y  observaciones  generales  aplicables  a  todo  el  proceso.   Mencionar  tambien  posibles
% extensiones a los metodos, experimentos que hayan quedado pendientes, etc.


\section{Apéndice}
\subsection{Apéndice A: Enunciado}
% \input{tp1.tex}

\end{document}
