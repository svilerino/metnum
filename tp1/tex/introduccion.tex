En el presente trabajo se intentó evaluar computacionalmente la seguridad térmica de un horno circular. El problema presentado consistía en estimar el riesgo que corre el mismo de fracturarse por efecto de la elevada temperatura. Dicho de otro modo, dadas las temperaturas de las paredes internas y externas del horno (obtenidas a través de sensores) se quiere estimar la ubicación de la isoterma de 500$^{o}$C, cuya cercanía a la pared externa es un indicador de la peligrosidad de la estructura.

Dado un corte transversal del horno podemos definir $r_i$ y $r_e$ como los radios de la pared interna y externa, ambos circualres. Nos interesa analizar la temperatura de los puntos entre ellas. Para referirnos a los puntos de dicha corona circular utilizaremos coordenadas polares, por lo que cada punto $P$ quedará definido por un radio $r$ y un ángulo $\theta$. Si llamamos $T(r,\theta)$ a la temperatura del punto $P_{r, \theta}$, podemos utilizar la ecuación del calor de Laplace para encontrar el estado de equilibrio del sistema:

\begin{equation}\label{calor-continuo}
\frac{\partial^2T(r,\theta)}{\partial r^2}+\frac{1}{r}\frac{\partial T(r,\theta)}{\partial r}+\frac{1}{r^2}\frac{\partial^2T(r,\theta)}{\partial \theta^2} = 0 
\end{equation}

la cual debe cumplirse para todos los puntos internos del horno.

Por estar trabajando con aritmética finita, usamos una discretización de los puntos que nos interesa analizar (todos los pertenecientes a la pared del horno, que forman una corona circular) y discretizamos asimismo la ecuación \ref{calor-continuo}. De esa manera arribamos a un sistema de ecuaciones lineal que podemos representar en su versión matricial como $Ax=b$ y que intentaremos resolver usando dos métodos: la Eliminación Gaussiana y Factorización LU.

Nos interesa comparar estos dos métodos por el tiempo que les lleva resolver un único sistema $Ax=b$ en función de la granularidad de la discretización. Posteriormente podemos complejizar el problema suponiendo que tenemos múltiples mediciones para las temperaturas de las paredes (a lo largo del tiempo), por lo que debemos comparar su performance a la hora de resolver mútiples sistemas $Ax=b_i$ con diferentes vectores $b_i$.

La solución del sistema de ecuaciones nos permitirá conocer el valor de la funcion $T$ en los puntos de la discretización elegida, pero es posible que ninguno de ellos coincida con el valor de la isoterma buscada. Otro objetivo del trabajo será evaluar diferentes formas de estimar esa isoterma y compararlas variando la granularidad de la discretización.

