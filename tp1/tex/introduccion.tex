En el presente trabajo se intentó evaluar computacionalmente la seguridad térmica de un horno circular. El problema presentado consistía en estimar el riesgo que corre el mismo de fracturarse por efecto de la elevada temperatura. Dicho de otro modo, dadas las temperaturas de las paredes internas y externas del horno (obtenidas a través de sensores) se quiere estimar la ubicación de la isoterma de 500$^{o}$C, cuya cercanía a la pared externa es un indicador de la peligrosidad de la estructura.

Dado un corte transversal del horno podemos definir $r_i$ y $r_e$ como los radios de la pared interna y externa, ambos circualres. Para referirnos a los puntos de dicha corona circular utilizaremos coordenadas polares, por lo que cada punto $p$ quedará definido por un radio $r$ y un ángulo $\theta$. Si llamamos $T(r,\theta)$ a la temperatura del punto $p_{r, \theta}$, podemos utilizar la ecuación del calor de Laplace para encontrar el estado de equilibrio del sistema:

\begin{equation}\label{calor-continuo}
\frac{\partial^2T(r,\theta)}{\partial r^2}+\frac{1}{r}\frac{\partial T(r,\theta)}{\partial r}+\frac{1}{r^2}\frac{\partial^2T(r,\theta)}{\partial \theta^2} = 0 
\end{equation}

la cual debe cumplirse para todos los puntos internos del horno.

Por estar trabajando con aritmética finita, usamos una discretización de los puntos que nos interesa analizar (todos los pertenecientes a la pared del horno, que forman una sección circular) y discretizamos asimismo la ecuación \ref{calor-continuo}. De esa manera arribamos a un sistema de ecuaciones lineal que podemos representar en su versión matricial como $Ax=b$ y que intentaremos resolver usando dos métodos: la Eliminación Gaussiana y Factorización LU.

Nos interesa comparar estos dos métodos por el tiempo que les lleva resolver un único sistema $Ax=b$ en función de la granularidad de la discretización. Posteriormente podemos complejizar el problema suponiendo que tenemos múltiples mediciones para las temperaturas de las paredes (a lo largo del tiempo), por lo que debemos comparar su performance a la hora de resolver mútiples sistemas $Ax=b_i$ con diferentes vectores $b_i$.

La solución del sistema de ecuaciones nos permitirá conocer el valor de la funcion $T$ en los puntos de la discretización elegida, pero es posible que ninguno de ellos coincida con el valor de la isoterma buscada. Otro objetivo del trabajo será evaluar diferentes formas de estimar esa isoterma y compararlas variando la granularidad de la discretización.

\subsection{Eliminación Gaussiana}
La Eliminación Gaussiana es un método que consiste en resolver el sistema triangulando la matriz (aplicando restas entre filas) y aplicando las mismas operaciones al vector solución, para obtener un sistema equivalente pero con una matriz triangular superior que se puede resolver de manera sencilla mediante \emph{backward substitution}.

El algoritmo de trabaja con la matriz ampliada del sistema:

\[\left(\begin{array}{ccc|c}
a_{11} & \ldots & a_{1n} & b_1\\
\vdots & \ddots & \vdots & \vdots\\
a_{n1} & \ldots & a_{nn} & b_n
\end{array}\right)\]

En cada paso $k$ el algoritmo produce, mediante operaciones de filas, la aparición de ceros debajo de la diagonal en la columna $k$:

\begin{align*}
&F_i \leftarrow F_i - \frac{a_{i1}^{(0)}}{a_{11}^{(0)}}F_1 &\forall i = 2, \ldots, n \\
&F_i \leftarrow F_i - \frac{a_{i2}^{(1)}}{a_{22}^{(1)}}F_2 &\forall i = 3, \ldots, n \\
&\ldots \\
&F_i \leftarrow F_i - \frac{a_{ik}^{(k - 1)}}{a_{kk}^{(k - 1)}}F_k &\forall i = k + 1, \ldots, n \\
&\ldots \\
&F_n \leftarrow F_n - \frac{a_{n(n-1)}^{(n-2)}}{a_{(n-1)(n-1)}^{(n-2)}}F_{n - 1} &
\end{align*}

Notemos que es necesario pedir que ningún valor de la diagonal se anule en ningún paso del algoritmo. Más adelante veremos que esto se puede garantizar gracias a la estructura de nuestra matriz.

El algoritmo tiene complejidad $\mathcal{O}(n^3)$ para una matriz $A \in \mathbb{R}^{n \times n}$ ya que realiza $\mathcal{O}(n)$ operaciones escalares por fila por paso; hay $n$ filas y realiza $n-1$ pasos.

\subsection{Backward substitution}
Dada una matriz triangular superior $A$ sin elementos nulos en la diagonal, entonces el vector solución $x$ es único, y se obtiene haciendo

\begin{align*}
x_n &= \frac{b_n}{a_{nn}}	\\
x_{n - 1} &= \frac{b_{n - 1} - a_{(n-1)n} x_n}{a_{(n-1)(n-1)}}	\\
&\vdots \\
x_1 &= \frac{b_1 - a_{12}x_2 - a_{13}x_3 - \ldots - a_{1n} x_n}{a_{11}}
\end{align*}

Notemos nuevamente la importancia de que los elementos de la diagonal sean no nulos para que las operaciones estén bien definidas.

Este algoritmo tiene complejidad $\mathcal{O}(n^2)$ pues para despejar cada incógnita realiza $\mathcal{O}(n)$ operaciones y hay $n$ incógnitas.

\subsection{Factorización LU}

El método de Factorización LU consiste en aplicar el método de Eliminación Gaussiana pero ``almacenando'' las operaciones realizadas para obtener una factorización $A = LU$ con $L$ triangular inferior (con unos en la diagonal) y $U$ triangular superior. Más formalmente, cada paso $k$ de la Eliminación Gaussiana es equivalente a multiplicar a la matriz $A^{(k-1)}$ a izquierda por la matriz

\[
M_k = 
\begin{pmatrix} 
1 		& \ldots 	& 0 				& \ldots 	& 0 \\
0 		& \ddots 	& 0 				& \ldots 	& 0 \\
\vdots 	& 			& 1 				& 			& \vdots\\
0		& \ldots		& -\frac{a_{(k+1)k}^{(k - 1)}}{a_{kk}^{(k - 1)}} 	& \ldots		& 0\\
\vdots	& 			& \vdots		 	& \ddots		& \vdots\\
0		& \ldots		& -\frac{a_{(n)k}^{(k - 1)}}{a_{kk}^{(k - 1)}}	& \ldots		& 1\\
\end{pmatrix}
\]

por lo que el resultado de la Eliminación Gaussiana es una matriz triangular superior $U$ que coincide con $M_{n-1} \ldots M_1 A$. Pero como todas las matrices $M_k$ son inversibles por ser triangulares inferiores con unos en la diagonal, vale $A = M_1^{-1}\ldots M_{n - 1}^{-1} U = LU$.

Notemos que es posible ir armando $L$ a cada paso de la Eliminación Gaussiana por lo que la complejidad computacional se mantiene: computar la factorización $LU$ toma también $\mathcal{O}(n^3)$.

Luego, dado cualquier vector $b$, para resolver el sistema $Ax=b$ alcanza con calcular $y$ tal que $Ly=b$ y luego $x$ tal que $Ux=y$. El primer sistema se puede resolver en $\mathcal{O}(n^2)$ por ser $L$ triangular inferior (usamos \emph{forward substitution}, que es igual que \emph{backward} pero a la inversa) y el segundo también, por ser $U$ triangular superior (usamos \emph{backward substitution} normalmente).

De este modo vemos que, si para una misma matriz $A$ y mútliples vectores $b_1, \ldots, b_k$ queremos resolver todos los sistemas $Ax=b_i$, en lugar de aplicar $k$ veces la Eliminación Gaussiana (que toma $\mathcal{O}(n^3)$) parecería conveniente calcular una única vez la factorización LU en $\mathcal{O}(n^3)$ y luego usarla para resolver los $k$ sistemas en $\mathcal{O}(n^2)$. Este trabajo práctico tiene como objetivo principal contrastar dicha conjetura experimentalmente.