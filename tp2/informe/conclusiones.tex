\IEEEPARstart{A}{} lo largo de este trabajo pudimos vislumbrar las complejidades propias del problema de ordenar una colección en principio desordenada de elementos. Tomamos el ejemplo de las páginas web por un lado y el de las competiciones deportivas por otro, enfocándonos particularmente en el fútbol. Usamos el algoritmo de PageRank y su adaptación GeM para resolver ambos problemas respectivamente, y estudiamos su comportamiento al variar diferentes parámetros de entrada, particularmente el factor de teletransportación $\alpha$.

Respecto al contexto de páginas Web, como conclusión general acerca de los experimentos de performance pudimos observar que la variación del factor de navegacion $\alpha$ no altera el tiempo consumido por iteración del método de cálculo del ranking (método de la potencia). Otras condiciones sobre el grafo de entrada (ie. densidad) sí producen una variación en el tiempo de cómputo por iteración. \\
Sobre los experimentos acerca de la convergencia, pudimos observar que a medida que el factor de navegación crece -para un vector inicial equiprobable- la cantidad de iteraciones requeridas hasta cumplir la condición de corte del algoritmo se incrementan, o equivalentemente, la velocidad de convergencia decrece. Estos experimentos nos dieron la pauta de cuales condiciones o parámetros afectan la performance del método implementado y de qué manera.

Respecto a los experimentos cualitativos acerca del orden, corroboramos que para cualquier factor de navegación $\alpha$ dentro de un rango válido el ranking obtenido por PageRank no varía a pesar de cambiar el autovector resultante. Para poner a prueba las bondades cualitativas de PageRank contrastamos el ranking para estas instancias mencionadas anteriormente contra otros algoritmos de rankeo, entre ellos In-Deg y el ranking de Google mismo al dia de hoy. Pudimos comprobar que PageRank genera rankings más ``interesantes'' o útiles que In-Deg y que, por ejemplo, los términos más importantes de una lista de temas quedan en los primeros puestos del ranking.

Como conclusión general para el caso de las competencias deportivas pudimos concluir que, si bien el algoritmo se comporta ``mejor'' que la punutación oficial para casos de comportamiento ``ideal'' donde todos los resultados están determinados de entrada y las victorias son transitivas (una liga de lo más aburrida), presenta comportamientos que podemos considerar ``injustos'' o ``indeseables'' para cierta definición de dichas palabras. Por ejemplo, la posibilidad de que un único encuentro altere fuertemente el orden final, y mejore significativamente la posición de un equipo que en nuestra opinión no ``merecía'' el lugar que GeM le asignaba. O también la posibilidad de que una mala racha de un equipo afecte negativamente la posición de todos los que le ganaron a él, dando lugar a especulaciones de todo tipo. Para estos casos, encontramos que valores pequeños de $\alpha$ se acercan más a nuestra noción de ``justicia'', dado que representan una mayor probabilidad de que ``cualquiera le gane a cualquiera'' lo cual es cierto en todos los deportes. Y ocurre al revés con los casos ideales: el algoritmo (en promedio) se comporta mejor para valores grandes de $\alpha$ cuando las victorias sí son transitivas.

A pesar de estos problemas, encontramos que, para casos estables y sin alterar, el algoritmo GeM puede acercarse muy bien a los órdenes establecidos por los ránkings oficiales, sobretodo para los primeros y últimos lugares. Inclusive logra esto en ciertos casos en donde ha habido resultados ``extraños'', como que un subcampeón le haya ganado al campeón en una fecha anterior a la final. E independientemente de la exactitud de posiciones, GeM parecería hacer un buen trabajo a la hora de distinguir, mirándolo como conjuntos, a los ``buenos'' de los ``malos'', obteniendo coincidencias fuertes (módulo el orden específico establecido por cada uno) con los sistemas de puntuación oficial.

En última instancia, queda a criterio del lector decidir si los órdenes devueltos por GeM son ``mejores'' o ``peores'' que los resultados oficiales, un terreno en el que decidimos conscientemente no meternos por estar repleto de subjetividades\footnote{Si, según GeM, Argentina le ganaba a Alemania en el Mundial lo habríamos calificado como el mejor método del mundo}, limitándonos a estudiar casos precisos donde estuviese más o menos claro qué orden era ``correcto''. En cualquier caso, siempre sobrevive la \emph{mayor objeción a la aplicación de GeM en la vida real}: prácticamente nadie entendería por qué su equipo está donde está.