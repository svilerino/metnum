\IEEEPARstart{A}{} lo largo de este trabajo pudimos vislumbrar las complejidades propias del problema de ordenar una colección en principio desordenada de elementos. Tomamos el ejemplo de las páginas web por un lado y el de las competiciones deportivas por otro, enfocándonos particularmente en el fútbol. Usamos el algoritmo de PageRank y su adaptación GeM para resolver ambos problemas respectivamente, y estudiamos su comportamiento al variar diferentes parámetros de entrada, particularmente el factor de teletransportación $\alpha$.

Como conclusión general para el caso de las competencias deportivas pudimos concluir que, si bien el algoritmo se comporta ``mejor'' que la punutación oficial para casos de comportamiento ``ideal'' donde todos los resultados están determinados de entrada y las victorias son transitivas (una liga de lo más aburrida), presenta comportamientos que podemos considerar ``injustos'' o ``indeseables'' para cierta definición de dichas palabras. Por ejemplo, la posibilidad de que un único encuentro altere fuertemente el orden final, y mejore significativamente la posición de un equipo que en nuestra opinión no ``merecía'' el lugar que GeM le asignaba. O también la posibilidad de que una mala racha de un equipo afecte negativamente la posición de todos los que le ganaron a él, dando lugar a especulaciones de todo tipo. Para estos casos, encontramos que valores pequeños de $\alpha$ se acercan más a nuestra noción de ``justicia'', dado que representan una mayor probabilidad de que ``cualquiera le gane a cualquiera'' lo cual es cierto en todos los deportes. Y ocurre al revés con los casos ideales: el algoritmo (en promedio) se comporta mejor para valores grandes de $\alpha$ cuando las victorias sí son transitivas.

A pesar de estos problemas, encontramos que, para casos estables y sin alterar, el algoritmo GeM puede acercarse muy bien a los órdenes establecidos por los ránkings oficiales, sobretodo para los primeros y últimos lugares. Inclusive logra esto en ciertos casos en donde ha habido resultados ``extraños'', como que un subcampeón le haya ganado al campeón en una fecha anterior a la final. E independientemente de la exactitud de posiciones, GeM parecería hacer un buen trabajo a la hora de distinguir, mirándolo como conjuntos, a los ``buenos'' de los ``malos'', obteniendo coincidencias fuertes (módulo el orden específico establecido por cada uno) con los sistemas de puntuación oficial.

En última instancia, queda a criterio del lector decidir si los órdenes devueltos por GeM son ``mejores'' o ``peores'' que los resultados oficiales, un terreno en el que decidimos conscientemente no meternos por estar repleto de subjetividades\footnote{Si, según GeM, Argentina le ganaba a Alemania en el Mundial lo habríamos calificado como el mejor método del mundo}, limitándonos a estudiar casos precisos donde estuviese más o menos claro qué orden era ``correcto''. En cualquier caso, siempre sobrevive la \emph{mayor objeción a la aplicación de GeM en la vida real}: prácticamente nadie entendería por qué su equipo está donde está.