\par A continuaci\'on se detallan todos los experimentos realizados en este
trabajo y sus resultados. Se detalla no s\'olo el experimento en si, sino que
tambi\'en se explican los resultados que se esperan comprobar y sus
motivaciones.

%---------------------------------------------------------------
\subsection{PageRank}
\subsection{Experimento 1}
\label{subsec:exp1}
\begin{LaTeXdescription}
    \item[Tesis]

    \item[Proposici\'on] 

    \item[M\'etodo de Experimentaci\'on]

    \item[Resultados, an\'alisis y discusi\'on]
\end{LaTeXdescription}


\newpage
\subsection{\'Ordenes: Google vs PageRank vs In-Deg}
\label{subsec:exp2}
\begin{LaTeXdescription}
    \item[Objetivo] Analizar el orden obtenido respecto de otros \'ordenes
        disponibles. ¿Es igual? ¿Hay coincidencias? ¿Cu\'antas? ¿Tienen
        sentido?\\

    \item[Proposici\'on] Cualitativamente hablando, ¿c\'omo es el orden obtenido
        por PageRank? ¿Bueno? ¿Malo?. Obviamente que estas categorizaciones son
        intr\'insecas a la realidad: s\'olo nosotros podemos decir que al
        realizar una b\'usqueda web, los resultados vinieron en un orden
        correcto o deseable (es decir, lo que se buscaba en las primeras
        posiciones).  Utilizar nuestro criterio personal para hablar de la
        calidad del orden obtenido no ser\'ia muy correcto, ya que cualquier
        otra persona con criterios distintos podr\'ia disentir y ninguno de los
        criterios ser\'ia \textit{a priori} m\'as correcto que el
        otro\footnote{Se podr\'ia ver muestralmente que opina la gente de
        distintos \'ambitos, pero esto escapa al objeto de estudio de este
        trabajo.}. Pero lo que si podemos hacer es comparar el resultado
        obtenido con otros resultados disponibles, de los cuales proponemos
        In-Deg (que se basa en el grafo de conectividad) y los resultados de un
        \textit{search engine}: Google.\\

    \item[M\'etodo de Experimentaci\'on] Utilizamos la misma instancia que en el
        experimento anterior. Sobre esta instancia s\'olo nos falta calcular el
        orden In-Deg (el cual no es otra cosa que ordenar a los nodos en orden
        descendiente seg\'un su grado de entrada, es decir seg\'un la cantidad
        de ejes que los apuntan). Luego utilizamos el orden provisto por Google
        en la b\'usqueda inicial m\'as los \'ordenes obtenidos en el experimento
        previo.\\

    \item[Resultados, an\'alisis y discusi\'on]
\end{LaTeXdescription}

\begin{table}[H]
    \centering
    \caption{\'Ordenes comparativos entre los resultados de Google, PageRank e
        In-Deg}
    \label{tbl:google_pagerank_vs_indeg_siteubaar} 
    \setlength{\tabcolsep}{3pt}
    \begin{tabular}{|l|l|l|}
        \hline
        Google & PageRank & In-Deg\\
        \hline\hline
        www.derecho.uba.ar & www.agro.uba.ar & www.agro.uba.ar\\
        orga2.exp.dc.uba.ar & www.uba.ar & www.uba.ar\\
        www.agro.uba.ar & videos.agro.uba.ar & videos.agro.uba.ar\\
        www.ffyb.uba.ar & www.agro.uba.ar/cursos & www.agro.uba.ar/cursos\\
        www.uba.ar & www.agro.uba.ar/ced & www.agro.uba.ar/ced\\
        www.fvet.uba.ar & www.derecho.uba.ar & www.derecho.com.ar\\
        videos.agro.uba.ar & www.ffyb.uba.ar & www.ffyb.uba.ar\\
        iigg.sociales.uba.ar & www.fvet.uba.ar & www.fvet.uba.ar\\
        www.agro.uba.ar/cursos & orga2.exp.dc.uba.ar & orga2.exp.dc.uba.ar\\
        www.agro.uba.ar/ced & iigg.sociales.uba.ar & iigg.sociales.uba.ar\\
        \hline
    \end{tabular}
\end{table}

\par Los resultados obtenidos en \ref{tbl:google_pagerank_vs_indeg_siteubaar}
dieron resultados idénticos en cuanto a In-Deg vs PageRank, no así la búsqueda
en google, que debe utilizar otras heurísticas que no consideramos en este
trabajo. Asimismo, el orden de las búsquedas en google para un mismo término van
cambiando a lo largo del tiempo\footnote{True Story.} .\\

\par Respecto a In-Deg y PageRank, sus \'ordenes resultaron idénticos. Los
motivos por lo cual esto ocurre fueron desarrollados en el experimento anterior,
pero intuyendo que esto no tiene que ser siempre
as\'i (sino claramente PageRank no tendr\'ia sentido), realizamos un nuevo
experimento. Creamos una nueva instancia de prueba basada en los resultados de
buscar en wikipedia\cite{wikipedia} distintos t\'erminos relacionados con los
temas vistos en la materia. El grafo de conectividad resultante se puede
observar en la figura \ref{fig:wiki_graph}.

\begin{figure}[H]
    \centering
    \includegraphics[width=0.75\textwidth]{exp2_conn_graph_metodos.png}
    \caption{Grafo de conectividad de p\'aginas de Wikipedia relacionadas con
        m\'etodos num\'ericos}
    \label{fig:wiki_graph}
\end{figure}

\par Sobre estas p\'aginas, calculamos los \'ordenes de PageRank e In-Deg, cuya
comparativa se encuentra detallada en el cuadro
\ref{tbl:pagerank_vs_indeg_wikipedia}. Nuevamente, In-Deg y PageRank dieron
resultados muy similares, pero no idénticos. M\'as aún, en In-Deg quedaron
varios nodos empatados, teniendo la misma cantidad de ejes entrantes. Pero esto
en PageRank no ocurri\'o, quedando definido un órden total.

\begin{table}[H]
    \centering
    \caption{\'Ordenes comparativos entre PageRank e In-Deg para Wikipedia}
    \label{tbl:pagerank_vs_indeg_wikipedia} 
    \setlength{\tabcolsep}{3pt}
    \begin{tabular}{|l|l||l|l|}
        \hline
        \multicolumn{2}{|c||}{PageRank} &\multicolumn{2}{c|}{In-Deg}\\
        \hline
        Puntaje & Nodo & Puntaje & Nodo\\
        \hline\hline
        0.196 & Matrix\_decomposition & 8 & Matrix\_decomposition\\
        0.165 & Numerical\_linear\_algebra & 7 & Numerical\_linear\_algebra\\
        0.125 & QR\_decomposition & 5 & QR\_decomposition\\
        0.106 & Numerical\_analysis & 4 & Numerical\_analysis\\
        0.105 & Singular\_value\_decomposition & 4 & LU\_decomposition\\
        0.098 & LU\_decomposition & 4 & Cholesky\_decomposition\\
        0.093 & Cholesky\_decomposition & 4 & Singular\_value\_decomposition\\
        0.057 & Eigendecomposition\_of\_a\_matrix & 2 & Eigendecomposition\_of\_a\_matrix\\
        0.050 & Matrix\_splitting & 2 & Matrix\_splitting\\
        \hline
    \end{tabular}
\end{table}

\par La diferencia entre \'ambos \'ordenes est\'a en la ubicaci\'on de
\emph{Singular value decomposition}. Mientras que en PageRank se encuentra en la
posici\'on 5, In-Deg lo lista en la 7ma ubicaci\'on. En el caso de In-Deg, la
determinaci\'on de la ubicaci\'on es determin\'istica, dependiendo del grado de
entrada del nodo que lo representa. Pero en nuestro ejemplo, vemos que existe un
empate con otros 3 nodos (como ya se ha explicado), con lo cual aqu\'i el orden
tambi\'en depende de la estabilidad y/o criterio de desempate que implemente
In-Deg. En nuestro caso, la implementaci\'on es estable, con lo cual se respeta
el orden inicial (o numeraci\'on) de los nodos. Por el otro lado, observando el
grafo podemos entender porque PageRank lo ubica en una posici\'on m\'as alta que
In-Deg: El nodo \emph{Numerical Linear Algebra}, uno de los de mayor puntaje (y
grado de entrada) tiene un link al nodo en cuesti\'on y a \emph{LU
decomposition}, pero no al resto de los valores que empatan en In-Deg. Por lo
visto en el experimento \ref{subsec:exp1}, sabemos que esto tiene un efecto de
subir el puntaje notoriamente en los nodos ''linkeados''. Luego, utilizando el
mismo razonamiento, observamos que \emph{Single value decomposition} queda por
encima de \emph{LU decomposition} por el voto/link de \emph{QR decomposition}.

\par Semánticamente, podemos observar que \textbf{en general}\footnote{\emph{QR}
queda por encima de \emph{Numerical Analysis}, creemos que esto se debe a la
morfología del grafo de referencias entre artículos en Wikipedia.} quedan
primeros en el ranking terminos mas \emph{generales} o \emph{abarcativos}; y a
medida que avanza el ranking se encuentran términos mas particulares.
Claramente, esta jerarqu\'ia de \emph{generalidad} es \'arbitraria para los
autores de este trabajo\footnote{¿Suma puntos hablar en tercera persona de
nosotros mismos? Very Scientific!}, aunque estimamos que habr\'a muy pocas
posibilidades de disenso al respecto para los temas representados por los nodos
del ejemplo.

\par Para evidenciar aún mas la diferencia entre los algoritmos de In-Deg y
PageRank decidimos alterar expl\'icitamente el grafo de conectividad de este
\'ultimo ejemplo, agregando aristas desde los 5 nodos peor puntuados hacia uno
de los últimos en el ranking In-Deg (\emph{Eigendecomposition of a matrix}).
Esto deberia aumentar drásticamente el rankeo del ultimo elemento en In-Deg ya
que aumentamos su grado de entrada en 5, pero no tanto asi en Pagerank donde la
''calidad'' de los votantes tiene una mayor injerencia. Los resultados de este
último experimento pueden verse en el cuadro
\ref{tbl:pagerank_vs_indeg_wikipedia_modificado}. 

\begin{table}[H]
    \centering
    \caption{\'Ordenes comparativos entre PageRank e In-Deg para Wikipedia con grafo alterado explícitamente}
    \label{tbl:pagerank_vs_indeg_wikipedia_modificado}
    \setlength{\tabcolsep}{3pt}
    \begin{tabular}{|l|l||l|l|}
        \hline
        \multicolumn{2}{|c||}{PageRank} &\multicolumn{2}{c|}{In-Deg}\\
        \hline
        Puntaje & Nodo & Puntaje & Nodo\\
        \hline\hline
        0.189 & Matrix\_decomposition & 8 & Matrix\_decomposition\\
        0.137 & Numerical\_linear\_algebra & 7 & Numerical\_linear\_algebra\\
        0.123 & Numerical\_analysis & 7 & Eigendecomposition\_of\_a\_matrix\\
        0.114 & Eigendecomposition\_of\_a\_matrix & 5 & QR\_decomposition\\
        0.108 & QR\_decomposition & 4 & Numerical\_analysis\\
        0.097 & Singular\_value\_decomposition & 4 & LU\_decomposition\\
        0.089 & LU\_decomposition & 4 & Cholesky\_decomposition\\
        0.087 & Cholesky\_decomposition & 4 & Singular\_value\_decomposition\\
        0.051 & Matrix\_splitting & 2 & Matrix\_splitting\\
        \hline
    \end{tabular}
\end{table}

\par Puede observarse que respecto a In-Deg el elemento con ejes entrantes
artificiales paso a ser el
tercero por su nuevo grado de entrada aumentado. En Pagerank, el elemento subió
de ranking pero quedó por debajo de Numerical\_analysis, lo cual en principio es
raro, ya que este item quedó con puntaje 4 en In-Deg. Si miramos el grafo,
veremos que, efectivamente, el hecho de que \emph{Numerical analysis} sea apuntado por
\emph{Matrix decomposition} y \emph{Numerical linear algebra}, que son los términos mas
importantes, le da mas potencia al elemento en cuestión que al elemento
\emph{Eigendecomposition of a matrix}, apuntado por los últimos 5 de la lista.

%\par Por último, vemos que en el último experimento se acentúa el orden
%\texttt{abarcativo} de los resultados respecto al dominio de los elementos.

\medskip
\par A lo largo de este experimento pudimos evidenciar las diferencias que
existen entre dos algoritmos de elaboraci\'on de rankings distintos: In-Deg y
PageRank. Esto, que no lo hab\'iamos podido exponer en el experimento previo,
nos demostr\'o el peso que PageRank le otorga a los links provenientes de
p\'aginas web con mejores puntajes, diferenciaci\'on que In-Deg no realiza.
M\'as a\'un, nos encontramos conque In-Deg parece ser tener m\'as posibilidades
de empate en su forma de ''rankear'' que PageRank, volvi\'endose dependiente del
criterio de desempate que se utilice y convirti\'endose en otra
faceta/problema a resolver, situaci\'on que puede ser despreciada en el caso de
PageRank (las probabilidades de empate ya son chicas para pocos nodos, y las
mismas son cada vez menores a mayor cantidad). Como dato de color, observamos
que los resultados devueltos por \emph{Google} son bastante distintos de los
obtenidos con PageRank, con lo cual queda claro que a pesar de haber sido este
m\'etodo un hito en la historia del motor de b\'usqueda, el mismo ya a
evolucionado much\'isimo en pocos a\~nos, lo que demuestra la importancia del
problema estudiado. 


\newpage
\subsection{Convergencia}
\label{subsec:exp3}
\begin{LaTeXdescription}
    \item[Objetivo] Estudiar como se comporta la convergencia en funci\'on de
        el factor de navegaci\'on $\alpha$.

    \item[Proposici\'on] PageRank propone una soluci\'on donde se observa a un
        conjunto de p\'aginas web como un grafo dirigido. Luego pasa este modelo
        a uno  matem\'atico, utilizando una matriz para computar una soluci\'on.
        Esta matriz, por construcci\'on, termina representando a un grafo
        completo (el grafo original no necesariamente lo era), y los valores del
        mismo dependen principalmente de 2 variables: $\epsilon$ y $\alpha$
        (algoritmo \ref{alg:power_method2}, p\'agina \pageref{alg:power_method2}
        y ecuaci\'on \ref{eq:M_def}, p\'agina \pageref{eq:M_def}). Si se observa
        el algoritmo, se ver\'a que claramente modificar el $\epsilon$ tendr\'a
        como resultado hacer que cadda corrida converga en m\'as o menos
        iteraciones, ya que el criterio de parada depende de \'el. As\'i pues,
        no vemos como algo rico experimentar con este valor. En cambio, $\alpha$
        es una variable que modifica en gran medida a nuestra matriz $M$, con lo
        cual es dificil saber cual ser\'a su injerencia en la convergencia (en
        principio). El objetivo, entonces, es ver como $\alpha$ afecta al
        m\'etodo matem\'atico iterativo de la potencia en cuanto a su
        convergencia.

    \item[M\'etodo de Experimentaci\'on] Tomaremos 3 instancias de grafos de
        conectividad de p\'aginas web de tama\~no mediano-grande, obtenidas en
        \url{http://snap.stanford.edu/data/\#web}. Luego, correremos el
        algoritmo de PageRank para $\alpha=0 0.1 0.2 \dots 0.9$ y $\epsilon$
        fijo en $0.00001$

    \item[Resultados, an\'alisis y discusi\'on]
\end{LaTeXdescription}

\begin{figure}[!h]
    \centering
    \subfloat[][Iteraciones hasta Converger en funci\'on de $\alpha$]{
        \label{subfig:exp3_comp}
        \includegraphics[width=.5\textwidth]{exp3_iteraciones_funcion_alpha.png}
    }
    \subfloat[][Velocidad de convergencia en funci\'on de $\alpha$]{
        \includegraphics[width=.5\textwidth]{exp3_diff_funcion_iteraciones_standford.png}
        \label{subfig:exp3_diff}
    }
    \caption{An\'alisis de Convergencia en funci\'on de $\alpha$}
\end{figure}


\newpage
\subsubsection{An\'alisis de Tiempo de C\'omputo}
\label{subsec:exp4}
\begin{LaTeXdescription}
    \item[Objetivo] Analizar la complejidad temporal del m\'etodo.\\

    \item[Hip\'otesis] Proponemos que el tiempo de c\'omputo por iteraci\'on
        para una instancia y $\epsilon$ fijos ser\'a siempre el mismo para todo
        $\alpha$. Tambi\'en proponemos que el tiempo de c\'omputo por interación
        para $\epsilon$ fijo ser\'a el mismo para \textbf{toda instancia} (sin
        importar tama\~no, cantidad de ejes, etc).\\

    \item[Proposici\'on] De los experimentos previos hemos llegado a la
        conclusi\'on de que a mayor $\alpha$, mayor cantidad de iteraciones
        ser\'an necesarias para converger, lo cual implica inmediatamente un
        mayor tiempo de c\'omputo requerido. La pregunta ideal a responder
        ser\'ia ''cu\'anto tiempo'', pero tambi\'en concluimos previamente que
        la cantidad de iteraciones es dependiente (entre otros factores) de la
        instancia de entrada. Con lo cual, responder a esta pregunta en el
        contexto de este trabajo es imposible sin una cota te\'orica para la
        cantidad de iteraciones (que ni siquiera sabemos si existe). En cambio,
        lo que s\'i podemos analizar es si el tiempo de c\'omputo \textbf{por
        iteraci\'on} es el mismo para toda instancia, sin importar el
        $\epsilon$\footnote{M\'as all\'a de que para nuestros experimentos lo
        estemos dejando fijo.} o $\alpha$. Analizamos entonces si el tiempo por
        iteraci\'on var\'ia seg\'un la densidad de la instancia inicial, y a su
        vez si varía para distintos valores de $\alpha$.\\

    \item[M\'etodo de Experimentaci\'on] Tomamos 3 instancias de tama\~no
        mediano-grande, con una diferencia relativa de densidad entre ellas
        significativa. Particularmente, tomamos las mismas del experimento
        anterior. Luego resolvemos cada instancia con PageRank 10 veces para
        $\alpha=0.0$; $0.1$; $0.2$; $\dots$; $0.9$\footnote{Totalizando un total
        de 300 corridas del m\'etodo (10 corridas para 10 valores de $\alpha$
        para 3 instancias distintas).}. Tomamos entonces para cada instancia y
        $\alpha$ fijos, el promedio de los tiempos de c\'omputo. Por \'ultimo,
        calculamos el tiempo de c\'omputo por iteraci\'on dividiendo este
        promedio por la cantidad de iteraciones totales que necesit\'o el
        m\'etodo para converger.\\

    \item[Resultados, an\'alisis y discusi\'on]
\end{LaTeXdescription}

\par Consideramos 2 enfoques para extraer conclusiones, el primero analiza el
tiempo consumido por iteración por el método de la potencia para diferentes
valores de $\alpha$ y el segundo se centra en el tiempo por iteraci\'on respecto
de la densidad de la instancia/grafo de entrada\footnote{Al referirnos a
\emph{densidad} de un grafo, nos referimos a la cantidad de ejes que tiene. A
mayor cantidad de ejes, m\'as denso es.}.

\begin{figure}[H]
    \centering
    \subfloat[][Tiempo por Iteraci\'on en funci\'on de $\alpha$.]{
        \label{subfig:exp4_tiempo_iteracion}
        \includegraphics[width=.55\textwidth]{exp4_tiempo_por_iteracion_notredame.png}
    }
        \subfloat[][Tiempo por Iteraci\'on promedio vs Densidad del Grafo]{
            \label{subfig:exp4_den}
            \includegraphics[width=.45\textwidth]{exp4_tiempo_vs_densidad.png}
        }
    \caption{An\'alisis de Tiempo de C\'omputo}
\end{figure}

\par Para el primer enfoque, observemos los resultados obtenidos en la figura
\ref{subfig:exp4_tiempo_iteracion}. En el mismo observamos dos curvas. La
primera, \emph{Tiempo}, que representa el tiempo por iteraci\'on promedio para
cada $\alpha$ (calculado como el tiempo promedio hasta converger de las 10
ejecuciones para un $\alpha$ fijo, dividido por la cantidad de iteraciones que
necesit\'o); y la segunda \emph{Promedio} que no es otra cosa que el promedio de
los tiempos por iteraci\'on calculados para la curva \emph{Tiempo}. Los
resultados expuestos en esta figura son similares para las 3 instancias de
prueba que se tomaron, con lo cual se decidi\'o exponer una sola.

\par Puede observarse en esta misma figura que el tiempo por iteraci\'on
calculado en funci\'on de $\alpha$ oscila en valores muy pequeños alrededor del
promedio. Para ilustrar esto, presentamos en el cuadro
\ref{tbl:exp4_data_notredame} las m\'etricas empíricas del experimento.

\begin{table}[H]
    \centering
    \caption{Métricas del tiempo por iteracion respecto de $\alpha$}
    \label{tbl:exp4_data_notredame} 
    \setlength{\tabcolsep}{3pt}
    \begin{tabular}{|l|l|}
        \hline
        Métrica & Segs\\
        \hline\hline
        Promedio & 0.059676\\
        Desv\'io Est\'andar & 0.000788\\
        M\'inimo & 0.058338\\
        M\'aximo & 0.060938\\
        Diferencia \%\footnotemark& 4.266231\\
        \hline
    \end{tabular}
\end{table}
\footnotetext{Diferencia \%: 100*(max-min)/max.}


\par Estas variaciones seguramente se deben al \emph{scheduler} del sistema
operativo y a fenómenos de bajo nivel de las corridas de los
experimentos. De hecho, ya vimos en el experimento \ref{subsec:exp3} que a
medida que aumentamos el $\alpha$, m\'as iteraciones ser\'an necesarias para
converger. Obviamente esto implica un mayor tiempo de c\'omputo, ergo, la
ejecuci\'on del experimento se vuelve a\'un m\'as sensible pues al estar m\'as
tiempo ejecut\'andose m\'as probabilidades hay de que el Sistema Operativo
decida darle el CPU a otro proceso de mayor prioridad que surja (o que al estar
tanto tiempo ejecut\'andose, su prioridad vaya bajando).

\par Finalmente concluimos que que el tiempo por iteraci\'on es constante, ya
que en las m\'etricas nos confirman, al observar los valores muy peque\~nos del
desv\'io est\'andard y diferencia porcentual, que estas diferencias para
distintos $\alpha$ sean muy probablemente errores de medici\'on (recordemos que
adem\'as estamos trabajando con instancias de entrada de tama\~no
mediano-grande).

\par Habiendo llegado a la conclusi\'on que la hip\'otesis sobre el tiempo de
c\'omputo por iteraci\'on es constante, debemos analizar por qu\'e. Repasando la
secci\'on \ref{sec:implementacion}, podemos decir que es razonable el
comportamiento presupuesto y observado, ya que el valor de $\alpha$ no modifica
la \emph{densidad} de la matriz original de conectividad $A$. Recordemos
entonces que nuestra implementaci\'on se basa en el algoritmo
\ref{alg:power_method3}, p\'agina \pageref{alg:power_method3}, que multiplica
usando justamente esta matriz $A$ y aprovechando que la misma es esparsa. Al no
modificar esta caracter\'istica la selecci\'on del par\'ametro $\alpha$, podemos
afirmar que la cantidad de $flops$ necesarios para el producto matriz-vector se
mantendr\'a constante en funci\'on de $\alpha$ (de hecho, $\alpha$ es un
par\'ametro que luego multiplica al vector resultante $A\vec{x}^{(k-1)}$ en
nuestro algoritmo, con lo cual no afecta en lo absoluto al producto
matriz-vector que involucra a $A$).

\par El segundo enfoque centra su atención en el tiempo por
iteracion respecto a la densidad del grafo asociado a la instancia de entrada.
Al contrario que lo esperado, la densidad (cantidad de ejes) del
grafo \textbf{s\'i} altera el tiempo de cómputo por iteración. Si observamos la
figura \ref{subfig:exp4_den}, observaremos que a mayor densidad, mayor tiempo
por iteraci\'on se necesita.

\par En esta figura se compara el tiempo promedio consumido\footnote{Se utiliza
el valor promedio del tiempo para todos los factores de navegación.} contra una
medida de densidad del grafo\footnote{Se toma como métrica un cociente entre la
cantidad de aristas del gráfo y la cantidad de aristas de una \emph{clique} de
su misma cantidad de nodos; multiplicado por una constante para igualar las
magnitudes con los valores de los tiempos de c\'omputo.}, y se ve, para las 3
instancias evaluadas, que el crecimiento del tiempo de c\'omputo parecer\'ia ser
l\'ineal en funci\'on de la densidad del grafo. Dado que s\'olo experimentamos
con estas 3 instancias, ser\'ia muy apresurado hacer dicha afirmaci\'on, pero sí
podemos decir que tenemos sospechas de que eso ocurra, dejando este aspecto para
un futuro posible experimento\footnote{Vale la pena aclarar, en este caso, que
podr\'iamos considerar a nuestras 3 instancias como poco densas, m\'as alla de
la diferencia notoria de densidad entre ellas.}.

\par Volviendo al hecho de que a mayor densidad, mayor ti\'empo de c\'omputo,
llegamos a la conclusi\'on de que esto se debe al mismo motivo que en el primer
enfoque, salvo que esta vez justifica el hecho de que \textbf{s\'i} se requiera
m\'as tiempo de c\'omputo. Ocurre que al ser la instancia de entrada m\'as
densa, menos esparsa ser\'a $A$ (m\'as ejes implica mayor cantidad de valores no
nulos en la matriz), y por lo tanto el producto $A\vec{x}$ de la
implementaci\'on del m\'etodo efectuar\'a m\'as $flops$.

\medskip
\par Finalizando, concluimos que una de nuestras hip\'otesis era correcta, y la
otra no. Peculiar es que las justificaciones que le encontramos a ambos
comportamientos (la relaci\'on tiempo de c\'omputo por iteraci\'on en funci\'on
de $\alpha$ y de la densidad de la instancia) es la misma: mientras que $\alpha$
no afecta a la densidad de la matriz de conectividad pesada $A$, cosa que si lo
hace la cantidad de ejes del grafo. Es decir, terminamos entendiendo que el
principal aspecto a tener en cuenta a la hora de ver como se ver\'a afectado el
tiempo de computo (para nuestra implementaci\'on basada en el algoritmo
propuesto por Kamvar et al.\cite{Kamvar2003}) ser\'a la
densidad/esparcidad\footnote{Coloquialismo.} de la matriz inicial del modelo.


\newpage
\subsection{GeM}
\subsection{PageRank vs Orden Total Conocido}
\label{subsec:exp5}
\begin{LaTeXdescription}
    \item[Objetivo] Analizar la convergencia del modelo GeM asumiendo que
        existe ranking ideal y correcto al cual converger.\\

    \item[Hip\'otesis] PageRank, utilizando el modelo GeM, converge finalmente a
        un ''Orden Real'' o ''Correcto'' de los competidores, si este existe.
        Adem\'as, no necesitar\'a de un grafo completo (los resultados de todos
        contra todos) para converger al mismo.\\

    \item[Proposici\'on] Como se coment\'o previamente en la experimentaci\'on
        sobre p\'aginas web, establecer cual es el ''mejor orden'' u ''orden
        correcto'' es completamente arbitrario. No hay una vara sobre la cual
        medir. En los deportes esto es a\'un más evidente, ya que se depende de
        los resultados deportivos, ¿y qui\'en es capaz de afirmar que la
        probabilidad de que Platense -el mejor equipo del mundo e insipirador
        del t\'itulo del enunciado de este trabajo- le gane al Barcelona es $0$?
        As\'i pues, en el caso de los deportes tampoco tenemos un orden total,
        conocido y determin\'istico para verificar que el resultado de PageRank
        es el correcto. Pero si existiese este orden, si fuese determin\'istico,
        ¿PageRank convergir\'ia al mismo?\\

    \item[M\'etodo de Experimentaci\'on] Generamos dos instancias ideales y
        completamente abstractas de los resultados de un torneo de f\'utbol con
        10 y 50 equipos respectivamente, que juegan todos contra todos una
        \'unica vez (45 partidos para la primera instancia, 1225 para la
        segunda). Las instancias son construidas de manera tal que $equipo_i$ le
        gana a $equipo_j$ si y s\'olo si $i<j$. Es decir, el ranking correcto es
        la numeraci\'on de los equipos de forma ascendente, y cada $equipo_i$
        ocupa el puesto $i$.

        \par Aprovechamos el hecho de que en los deportes hay una componente
        temporal y generamos los partidos por fecha (es decir, grupos de
        partidos que ocurren todos al mismo tiempo, o al menos as\'i ser\'a para
        la perspectiva del algoritmo, que recibir\'a todos los partidos de una
        fecha juntos). En cada fecha se enfrentan, de a dos, equipos que no se
        hayan enfrentado todav\'ia y que no jueguen otro partido esa misma
        fecha. Dentro de esas restricciones, los enfrentamientos de cada fecha
        se eligen al azar, pero con una semilla fija (5) para obtener
        reproducibilidad en los experimentos. \textbf{Notar que la cantidad de
        fechas necesarias para que se jueguen todos los partidos no est\'a
        determinada \'unicamente por la cantidad de equipos sino que depende
        tambi\'en de como se elijan los enfrentamientos de cada fecha. Esto se
        debe a que confeccionar un generador de instancias aleatorias que
        respete las restricciones y genere un fixture de $n-1$ equipos es
        complejo y escapa a los objetivos de este trabajo. As\'i pues, nuestra
        instancia de 10 equipos \underline{consta de 11 fechas} y la instancia
        de 50 equipos \underline{consta de 53 fechas}}.

        \par El resultado de todos los partidos es siempre 1 a 0. No hace falta
        considerar empates dado que siempre hay un ganador. Luego, ejecutamos
        GeM tantas veces como fechas haya, pas\'andole en cada instancia una
        fecha m\'as. Es decir, en la ejecuci\'on 1 le pasamos los resultados de
        la fecha 1, en la 2 los resultados de la fecha 1 y 2, y as\'i
        sucesivamente. Para cada resultado de GeM, comparamos el ranking
        obtenido con el correcto, para alguna medida de distancia entre
        rankings, que definiremos m\'as adelante.

        \par Hace falta considerar un detalle importante, que es qu\'e
        decisi\'on tomar ante empates del ''puntaje'' asignado por GeM. Si
        desempat\'aramos por n\'umero de equipo de manera ascendente caer\'iamos
        en el molesto caso de que ya desde antes de empezar el torneo la salida
        de GeM coincidir\'ia con el orden ideal. Lo mismo vale para cualquier
        subconjunto de competidores empatados en un momento dado: su orden
        relativo coincidir\'ia con el ideal, aunque por casualidad y no por
        virtud del algoritmo. Resolvimos entonces ''romper'' el caso y que el
        desempate se realice de manera descendente, asegur\'andonos as\'i de no
        tomar como correctos resultados en donde hay muchos empates.

        \par Repetimos el experimento variando los valores de $\alpha$ en
        factores de $0.2$, para estudiar la convergencia en cada caso.  De
        confirmarse nuestra hip\'otesis la diferencia/distancia del ranking
        respecto del orden total ideal (que sabemos que existe por
        construcci\'on) deber\'ia llegar a ser 0, eventualmente ''antes'' de que
        se hayan jugado todas las fechas.\\

    \item[Resultados, an\'alisis y discusi\'on]
\end{LaTeXdescription}

\par En primer lugar, como hemos adelantado, debemos decidir como determinar la
''distancia'' entre dos rankings. Es decir, determinar alguna manera de comparar
dos rankings dados, y poder tener una magnitud de cuan parecidos son. Para ello
consideramos algunas opciones para dados dos rankings A y B, de entre las cuales
mencionamos:

\begin{enumerate}
        \item La sumatoria de la distancia, para cada competidor, entre su
            posici\'on en el ranking A y la posici\'on en el ranking
            B.\label{itm:distancia_rankings}
\end{enumerate}

\par Luego de evaluar estas opciones (y algunas m\'as no tan concisas),
decidimos trabajar de aqu\'i en m\'as con la distancia basada en la suma de las
distancias de todos los competidores (definici\'on de distancia n\'umero
\ref{itm:distancia_rankings}). Consideramos que dicha forma de tomar la
distancia entre dos rankings, de alguna manera nos est\'a se\~nalando cuantas
permutaciones de elementos/equipos contiguos hay entre un ranking y el otro,
cosa que no queda tan claro con las dem\'as opciones. Y esa forma de medir la
distancia es la que consideramos que se ajusta a lo que queremos observar del
dominio del problema: cu\'antos equipos est\'an ubicados distinto y cuan lejos.

\begin{figure}[H]
    \caption{Distancia al orden total ideal/correcto ($c = \alpha$)}
    \label{fig:exp5_1}
    \centering
    \subfloat[][Torneo de 10 equipos]{
        \label{subfig:exp5_10equipos}
        \includegraphics[width=.5\textwidth]{exp5_10_equipos.png}
    }
    \subfloat[][Torneo de 50 equipos]{
        \includegraphics[width=.5\textwidth]{exp5_50_equipos.png}
        \label{subfig:exp5_50equipos}
    }
\end{figure}

\par La hip\'otesis fue confirmada. GeM converge al Orden Real para todos los
valores no nulos de $\alpha$ y no precisa todas las fechas para hacerlo, como se
puede observar en la figura \ref{fig:exp5_1}. Sin embargo, necesita una cantidad
significativa: para todos los valores no nulos de $\alpha$ fueron necesarias 10
de 11 fechas para el caso de 10 equipos y 50 de 53 fechas para el caso de 50
equipos.

\par El caso $\alpha=0$ no funciona porque la matriz termina descartando los
resultados de los equipos y usando simplemente la misma probabilidad para
cualquier equipo. En adelante dejaremos de lado este caso.

\par La variaci\'on $\alpha$ en el caso de 10 equipos no represent\'o
diferencias en el orden devuelto en cada fecha. es por eso que en la figura
\ref{subfig:exp5_10equipos} no se ven m\'as que dos curvas (una de las cuales
corresponde al siguiente experimento): estan todas superpuestas. En el caso de
50 equipos s\'i hubo leves diferencias en el orden devuelto en cada fecha.
Observamos que a mayor $\alpha$ el algoritmo en general converge ''mejor'' al
Orden Real (es decir, dada la misma cantidad de fechas, se acerca m\'as), aunque
siempre precisa 50 fechas para converger al orden real (figura
\ref{subfig:exp5_50equipos}).

\par La diferencia de todos modos es peque\~na y probablemente se deba a que, en
nuestro modelo ideal, las victorias son totalmente transitivas: la probabilidad
de que $C$ le gane a $A$ dado que $A$ le gan\'o a $B$ y $B$ le ganó a $C$ es
siempre $0$.  Como valores peque\~nos de $\alpha$ tienden a agregar una
probabilidad de que $C$ efectivamente le gane a $A$, la convergencia mejora
levemente para valores altos de $\alpha$ (donde esa probabilidad artificial es
cada vez menor). Sin embargo, variando la semilla usada para ordenar los
enfrentamientos encontramos casos en donde un valor mayor de $\alpha$ empeoraba
la convergencia en alg\'un punto (ver figura \ref{subfig:exp5_c_malo}), por lo
que no podemos generalizar esta conclusi\'on.

\par As\'i pues, motivados por estos resultados, realizamos la siguiente
experimentaci\'on.

%---------------------------------------------------------------
\subsubsection{PageRank vs Ranking FIFA en un caso de Orden Total Conocido}
\label{subsec:exp5_aux}
\begin{LaTeXdescription}
    \item[Objetivo] Comparar GeM contra el raking est\'andar del f\'utbol en un
        caso ideal.\\

    \item[Hip\'otesis] PageRank, utilizando el modelo GeM, converge m\'as
        r\'apido al ''Orden Real'', si lo hay, que el sistema oficial del
        f\'utbol asociado establecido por la FIFA\cite{fifa}.\\

    \item[Proposici\'on] Asumiendo que se confirm\'o la hip\'otesis del
        experimento anterior, nos interesa analizar si, asumiendo que existe un
        orden ideal/correcto, GeM se comporta peor, igual o mejor que la forma
        est\'andar de ordenar a los equipos (por puntos, 3/1/0 puntos para
        victoria/empate/derrota) donde ''comportarse'' mejor significa que con
        la misma información disponible (por ejemplo, la mitad de los partidos
        jugados) se acerca m\'as al orden ideal.\\

    \item[M\'etodo de Experimentaci\'on] Consideramos las mismas instancias de
        prueba que el experimento anterior. Comparamos la distancia entre GeM y
        el orden ideal contra la distancia entre el ordenamiento est\'andar y el
        orden ideal, usando la misma definici\'on de distancia de rankings que
        para el experimento anterior y el mismo criterio para ordenar equipos
        empatados en puntaje. De confirmarse nuestra hip\'otesis, deber\'iamos
        ver que GeM se acerca ''m\'as r\'apido'', es decir, con menos fechas, al
        orden ideal.\\

    \item[Resultados, an\'alisis y discusi\'on] 
\end{LaTeXdescription}

\begin{figure}[H]
    \caption{Casos Patol\'ogicos (10 equipos, $c = \alpha$)}
    \label{fig:exp5_2}
    \centering
    \subfloat[][Caso particular malo para $\alpha$ grande]{
        \label{subfig:exp5_c_malo}
        \includegraphics[width=.5\textwidth]{exp5_ejemplo_c_malo.png}
    }
    \subfloat[][Caso particular malo para GeM]{
        \includegraphics[width=.5\textwidth]{exp5_ejemplo_gem_malo.png}
        \label{subfig:exp5_gem_malo}
    }
\end{figure}

\par Comparamos GeM con el orden oficial, establecido por asignación de puntos
(3-1-0) y confirmamos tambi\'en nuestra hip\'otesis de que en el caso promedio,
GeM se comporta mejor que el orden oficial de asignaci\'on de puntajes,
acerc\'andose más al ideal para cualquier cantidad de fechas.

\par Sin embargo tambi\'en encontramos semillas para las cuales esto no era real
en alg\'un punto (ver figura \ref{subfig:exp5_gem_malo}) por lo que tampoco
podemos generalizar esta conclusi\'on.

\par Por \'ultimo, dado que los dos casos ''patol\'ogicos'' fueron encontrados
con la instancia de pocos equipos (10) y no logramos encontrar una semilla que
los genere en el caso de muchos (50), podr\'iamos argumentar que GeM se comporta
''mejor'' cuando la cantidad de equipos es grande. Queda fuera de los alcances
de este trabajo confirmar m\'as sistem\'aticamente esa hip\'otesis, pero es un
buen trabajo futuro.

%---------------------------------------------------------------
\medskip
\par En este experimento se trabaj\'o principalmente con instancias artificiales
''de juguete''. Las mismas lejos est\'an de representar la realidad, pero nos
sirvieron para poder asegurar que el modelo GeM ''descubrir\'ia'' el ranking
justo (o real, como dir\'ia Platon) si este existiese. A\'un as\'i, observamos
que para llegar a este ranking el modelo requerir\'a de mucha informaci\'on de
entrada, lo cual le quita quiz\'as utilidad como predictor de resultados de
torneos\footnote{Sin \'animo de menospreciar a GeM, si uno tiene los datos de 50
sobre 53 fechas, predecir con buena probabilidad de acierto el orden final no es
necesariamente una tarea extremadamente dif\'icil.}. Tambi\'en observamos que en
la progesi\'on fecha a fecha, el par\'ametro $\alpha$ pareciera tener de muy
poca a nula influencia sobre los rankings obtenidos. Por \'ultimo, decidimos
comprar la ''velocidad de aproximaci\'on al ranking ideal'' con casos reales (es
decir, con el resultado final de la competencia), y aqu\'i pudimos ver que si
bien considerabamos no tan bueno a GeM para aproximarse con poca informaci\'on
al resultado final, se comporta notoriamente mejor que el ranking oficial (para
los casos de f\'utbol que fueron considerados), con lo cual este modelo quiz\'as
podr\'ia ser un primer paso hacia un predictor de resultados de torneos de
f\'utbol profesional\footnote{Y la dominaci\'on mundial.}.


\newpage
\subsection{GeM vs Ranking Oficial}
\label{subsec:exp6}
\begin{LaTeXdescription}
    \item[Tesis]

    \item[Proposici\'on] 

    \item[M\'etodo de Experimentaci\'on]

    \item[Resultados, an\'alisis y discusi\'on]
\end{LaTeXdescription}


\newpage
\subsection{Estabilidad de PageRank/GeM}
\label{subsec:exp7}
\begin{LaTeXdescription}
    \item[Tesis]

    \item[Proposici\'on] 

    \item[M\'etodo de Experimentaci\'on]

    \item[Resultados, an\'alisis y discusi\'on]
\end{LaTeXdescription}


\newpage
\subsection{Caso Particular GeM}
\label{subsec:exp8}
\begin{LaTeXdescription}
    \item[Objetivo] Analizar cuan ''Justo'' es GeM, para un caso particular en
        el cual no haya dudas sobre lo que es justo y lo que no\footnote{O que
        haya muy poca probabilidad de que haya dudas al respecto.}.\\

    \item[Proposici\'on] Nos interesa analizar cu\'an ''justo'' es GeM para
        cierta definici\'on de justicia. Consideremos el caso de un torneo en
        que el equipo A le gana a todos los equipos salvo a B, y B pierde todos
        los partidos salvo el que le gana a A. Bajo nuestra definición de
        ''justicia'' o ''equidad'', o un aspecto de ella, A deber\'ia estar
        seguro por encima de B y B no deber\'ia estar por encima de muchos
        equipos (ya que perdi\'o contra todos). Observamos que en el caso del
        f\'utbol y su ránking 3-1-0 (o el esquema antiguo, 2-1-0) efectivamente
        B estar\'ia en la \'ultima posici\'on y A estar\'ia en la primera
        (eventualmente compartiendo dichas posiciones con alg\'un otro equipo).
        Entendemos entonces que este caso particular el ranking 3-1-0 es
        ''justo'' en este aspecto. Pero intu\'imos que esto no ser\'a lo que
        ocurra con GeM, ya que en el grafo de la instancia, A tiene un \'unico
        eje saliente (hacia B) y 18 entrantes, con lo cual su arista deber\'ia
        hacer subir mucho a B en el ranking.\\

    \item[Hip\'otesis] PageRank/GeM no es ''justo'' en cuanto al aspecto
        mencionado.\\

    \item[M\'etodo de Experimentaci\'on] Generamos una instancia de 20 equipos
        todos contra todos, donde existen A y B como se describieron. Entre los
        dem\'as equipos hacemos que los resultados sean aleatorios (con semilla =
        5). Ejecutamos GeM y observamos el ranking final para diferentes valores
        de $\alpha$ (el factor de navegaci\'on, como lo venimos llamando en este
        trabajo).\\

    \item[Resultados, an\'alisis y discusi\'on]
        
\end{LaTeXdescription}

\begin{wrapfigure}{l}{0.5\textwidth}
    \includegraphics[width=0.5\textwidth]{exp8_posicion_B.png}
    \caption{Posici\'on del equipo B en el ranking en funci\'on del factor
        $\alpha$ ($c=\alpha$)}
    \label{fig:exp8_posB}
\end{wrapfigure}
\noindent

%---------------------------------------------------------------

\newpage
\subsection*{Experimentos a Futuro}
\par Como experimentos futuros relacionados con estos temas, que por cuestiones
de tiempo y alcance de este trabajo no fueron realizados, quedaría evaluar la
\emph{sensibilidad} de PageRank y su convergencia al orden m\'as que al
(auto)vector de distribuci\'on estacionario.

\par En primer caso, la sensibilidad para PageRank, trata de exponer si cambios
peque\~nos en un grafo dado (algunos ejes menos, algunos ejes m\'as) modifican
radicalmente el orden obtenido. Y tambi\'en experimentar sobre como $\alpha$
afecta a esta sensibilidad.

\par En cuanto al caso de la convergencia al orden m\'as que al vector
resultante, lo que queremos explicar es que en realidad lo que se busca obtener
con PageRank es un ranking, m\'as all\'a de los puntajes de cada p\'agina web.
Quizás se podr\'ia observar si para toda instancia y $\alpha$ fijo, se puede
estimar estadist\'icamente que luego de $k$ iteraciones (que esperamos que sea
un n\'umero, o incluso una funci\'on que tome ciertos parametros y devuelva un
n\'umero, mucho menor que la cantidad de interaciones necesarias por el
m\'etodo para converger) ya se lleg\'o al mismo orden al que se llegar\'a al
converger (aunque con distintos puntajes). Si esto se pudiera realizar,
podr\'iamos tener una primera heur\'istica para mejorar el tiempo de
convergencia, aunque claramente estar\'iamos trabajando ya con orden \'aun m\'as
aproximado que el que devuelve el m\'etodo de la potencia implementado sobre
aritm\'etica finita.
