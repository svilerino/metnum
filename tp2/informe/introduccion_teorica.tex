\par \IEEEPARstart{E}{N} el siguiente trabajo se aborda una problem\'atica tan
com\'un a muchos \'ambitos distintos: La generaci\'on de \textit{Rankings} u
\'ordenes. El objeto de estudio en nuestro caso ser\'a el caso particular de la
elaboraci\'on de estos rankings para el caso de las p\'aginas web para luego
tratar de extrapolar los m\'etodos encontrados y aplicarlos a los participantes
de eventos deportivos.

\par Esta introducci\'on est\'a orientada a explicar muy brevemente: las
caracter\'isticas y motivaciones detr\'as de la necesidad de generar estos
rankings en el \'ambito de las p\'aginas web, el modelo propuesto orignalmente
por el motor de b\'usqueda \emph{Google} con una breve fundamentaci\'on
matem\'atica del mismo, para finalizar introduciendo los conceptos te\'oricos
utilizados para adaptar dicho model a los eventos deportivos.

%-------------------------------------------------------------------------------
\subsection{El Problema}
\par En su publicaci\'on original en 1998~\cite{Brin1998}, Brin y Page comentan
algunas de las caracter\'isticas principales de su motor de b\'usqueda de la
\emph{World Wide Web} que apuntan a solucionar las problem\'aticas existentes y
crecientes de la Internet de aquella \'epoca (que incluso al d\'ia de hoy siguen
siendo problemas vigentes y estudiados).

\par En dicho documento se hace hincapi\'e, entre otras cosa, en como el
crecimiento exponencial tanto en el uso de internet como de la cantidad de
informaci\'on\footnote{a.k.a: P\'aginas web.} han generado un verdadero problema
a la hora de encontrar el contenido buscado por los usuarios. Sin mencionar que
estos nuevos usuarios, en su gran mayor\'ia, no son programadores ni
administradores de red ni nada similar. Es decir, estos nuevos usuarios no
tienen un conocimiento (ni se espera que lo tengan, si uno espera que el uso de
internet sea masivo) para realizar \emph{queries} complejas en alg\'un lenguaje
de b\'usqueda\footnote{Como por ejemplo \emph{ANSI SQL}.}. E incluso para
usuarios con dicho conocimento, ser\'ia m ucho m\'as practico poder realizar la
b\'usqueda y obtener los resultados correctos sin tener que recurrir a dicha
sint\'axis.

\par Junto con el crecimiento de la informaci\'on, que a su vez se da desde
distintas partes del mundo (generando una problem\'atica geogr\'afica de acceso
a la informaci\'on), el problema de las b\'usquedas de contenido espec\'ifico se
a\'un m\'as d\'ificil: ahora buscar la informaci\'on correcta en un universo cada vez
m\'as grande y diverso requiere de m\'etods m\'as elaborados.

\par 


%-------------------------------------------------------------------------------
\subsection{El Modelo Matem\'atico}


%-------------------------------------------------------------------------------
\subsection{Adaptaci\'on a Eventos Deportivos}


%-------------------------------------------------------------------------------
