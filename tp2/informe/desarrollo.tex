\par \IEEEPARstart{E}{N} la secci\'on anterior se presento una descripci\'on
breve del problema a resolver, y se lleg\'o a un modelo matricial del mismo
sobre el cual la resoluci\'on del problema se reduc\'ia a calcular el autovector
$x$ asociado al autovalor $\lambda$ de valor 1. A su vez, se demostr\'o que como
consecuencia del modelo y su matriz resultante, este $x$ y $\lambda$ existir\'an
(y en el caso de $x$, que ser\'a \'unico).

\par A continuaci\'on presentamos un m\'etodo num\'erico iterativo el cual fue
utilizado para resolver este problema. Explicaremos brevemente el m\'etodo en sí
mismo, sus posibles desventajas y porque el mismo funcionar\'a en este contexto.

%---------------------------------------------------------------
\subsection{M\'etodo de la Potencia}
\par El m\'etodo de la potencia es una t\'ecnica iterativa usada para calcular
el autovalor dominante de una matriz (es decir, el autovalor de mayor valor
absoluto). A su vez este m\'etodo nos permite calcular no s\'olo el autovalor
dominante, sino que tambi\'en un autovector asociado (que es de hecho, el uso que
es relevante para este trabajo).

\par Para utilizar este m\'etodo se asume que se tiene una matriz cuadrada de
dimensi\'on $n$ que tiene $n$ autovectores \emph{li}\footnote{Linealmente
independientes} Y, evidentemente por lo ya comentado en el p\'arrafo anterior,
se asume que la matriz tiene un \'unico autovalor dominante:
\begin{equation*}
    |\lambda_1|>|\lambda_2|\geq|\lambda_3|\geq\dots\geq|\lambda_n|\geq 0
\end{equation*}

\par Estos requisitos son los necesarios para asegurar que el m\'etodo de la
potencia ser\'a satisfactorio. En el caso de no tener $n$ autovectores li el
m\'etodo a\'un podr\'ia terminar satisfactoriamente, pero no se podr\'ia
garantizar este resultado~\cite[p.576]{Burden2010}.

\par El m\'etodo en cuesti\'on, en su forma de pseudoc\'odigo, se encuentra en
el algoritmo \ref{alg:power_method}:
\begin{algorithm}
    \KwIn{Dimensi\'on $n$; matriz $M$; vector $x$; tolerancia $\epsilon$;
        m\'aximo n\'umero de iteraciones $N$}
    \KwOut{Autovalor aproximado $\mu$; autovector aproximado $x$ (con
    $||x||_\infty = 1$)}
    $k\gets 1$\;
    Sea $p$ el menor entero tal que $1\leq p\leq n$ y $\displaystyle |x_p| =
        ||x||_\infty$\;
    $x\gets x/x_p$\;
    \While{$k\leq N$}{
        $y\gets Mx$\;
        Sea $p$ el menor entero tal que $1\leq p\leq n$ y $\displaystyle |y_p| =
            ||y||_\infty$\;
        $\mu\gets y_p$\;
        \If{$y_p = 0$}{
            \KwRet{$M$ tiene un autovalor $0$, seleccione un nuevo vector $x$ y
                cominece de nuevo}\;
                \tcc{El proceso finaliz\'o insatisfactoriamente}
        }
        $\delta\gets ||x-(y/y_p)||_\infty$\;
        $x\gets y/y_p$\;
        \If{$\delta < \epsilon$}{
            \KwRet{$\mu,x$}\;
            \tcc{El m\'etodo finaliz\'o satisfactoriamente}
        }
        $k\gets k+1$\;
    }
    \KwRet{Se alcanz\'o el m\'aximo n\'umero de iteraciones}\;
    \tcc{El proceso finaliz\'o insatisfactoriamente}
    \caption{M\'etodo de la potencia - Pseudoc\'odigo~\cite[p.578]{Burden2010}}
    \label{alg:power_method}
\end{algorithm}

\par Ahora bien, el m\'etodo expuesto como se lo ve en el algoritmo
\ref{alg:power_method} es gen\'erico, y no toma en cuenta las caracter\'isticas
propias de nuestra matriz $M$ de entrada.

\par En primer lugar, para que este m\'etodo sea \'util a nuestro problema,
deber\'iamos poder asegurar que nuestro autovalor $\lambda = 1$ es el
\textbf{\'unico} autovalor dominante de la matriz $M$. Esto efectivamente
ocurre, como demostramos a continuaci\'on:

\smallskip
\begin{itemize}
    \item $M$ es est\'ocastica por columnas, positiva y cuadrada
        (secci\'on~\ref{itm:m_pos}) $\implies M^T$ es est\'ocastica por filas,
        positiva y cuadrada (trivial).\\

    \item $det(M) = det(M^T) \implies det(M-\lambda I) = det(M^T-\lambda I)
        \implies$ $M$ y $M^T$ tienen los mismos autovalores.\\

    \item $\nexists\lambda\in L(M^T) / \left|\lambda\right| > 1$ Para $L(M^T)$
        el conjunto de autovalores de $M^T$. Demostraci\'on:
        \begin{align}
            \intertext{Sea $\lambda$ autovalor de $M^T$ y $v$ su autovector
            asociado, tal que $|\lambda| > 1$.}
            &\iff M^Tv = \lambda v \implies |M^Tv| = |\lambda v|
            \overarrow[\implies][\downarrow]{$M^T > 0$} M^T |v| = |\lambda||v|\\
            %
            \intertext{Sea $m = M^T|v|$). Luego, podemos asegurar que cada
            componente del vector $m$ ser\'a menor a $v_{max}$, para $v_{max}
            = max |v_i|, \forall i$}
            &m_i = \sum_{j=1}^n M^T_{ij} |v_j|
            \overarrow[\leq][\downarrow]{$v_{max}\geq |v_i|$} \sum_{j=1}^n
            M^T_{ij} v_{max} = v_{max}\sum_{j=1}^n M^T_{ij}
            \overarrow[=][\downarrow]{Est\'ocastica} v_{max}\\
            %
            &\implies \forall i,\quad m_i \leq v_{max}\label{eq:abs1}\\
            %
            \intertext{Sin p\'erdida de generalidad, digamos que $j$ es el
            \'indice tal que $|v_j| = v_{max}$. Luego:}
            &m_j = |\lambda||v_j| = |\lambda|v_{max}
            \overarrow[>][\downarrow]{$|\lambda| > 1$} v_{max}\\
            %
            &\implies m_j > v_{max}\label{eq:abs2}\\
            %
            \intertext{Pero entonces llegamos a un absurdo: }
            &\left.\begin{aligned}
                \text{De \ref{eq:abs1}: } \forall i,\quad m_i \leq v_{max}
                    \implies m_j \leq v_{max}\\
                \text{De \ref{eq:abs2}: } m_j > v_{max}\label{eq:dom}
            \end{aligned}\right\} \implies\text{Absurdo}.
        \end{align}
        As\'i pues, llegamos a un absurdo por haber asumido que $|\lambda|>1$,
        ergo esta suposici\'on ha de ser falsa y tenemos que $|\lambda|\leq
        1$.\\

    \item Reci\'en vimos que $\forall\lambda\in L(M^T), |\lambda|\leq 1$. Como
        $M$ y $M^T$ tienen los mismos autovalores, entonces $\forall\lambda\in
        L(M), \lambda\leq\left|1\right|$\\

    \item Como $1$ es autovalor de $M$ (secci\'on \ref{subsubsec:dang_nodes}) y
        sabemos que ning\'un autovalor de $M$ puede tener magnitud mayor a 1
        (ecuaci\'on \ref{eq:dom})$\implies$ $1$ es un autovalor dominante de
        $M$.  Como corolario, tendr\'iamos que $\rho(M) = 1$.\\

    \item Por \'ultimo, para demostrar la unicidad de este autovalor dominante
        $1$, alcanza con remarcar que la multiplicidad de dicho autovalor ya se
        ha mostrado que es $1$ (secci\'on \ref{subsubsec:dang_nodes}). Pero
        sino, tambi\'en podemos enunciar que $M$ es positiva y cuadrada
        $\implies \rho(M)$ es el \'unico autovalor de $M$ en el radio
        espectral~\cite[p.664]{Meyer2000}, con lo cual $1$ es entonces
        \textbf{el \'unico autovalor dominante} de $M$.\\
\end{itemize}
\medskip

\par As\'i pues ya sabemos cual es el valor del autovalor dominante $M$: $1$.
Por lo tanto, el c\'alculo de este valor no nos hace falta.  M\'as a\'un, como
sabemos que $M$ es estoc\'astica y nuestro vector inicial $x$ (u $x^{(0)}$)
ser\'a un vector de probabilidad (es decir, estoc\'astico), entonces podemos
asegurar que $Mx^{(0)}$ ser\'a estoc\'astico\footnote{Trivial de demostrar.}.
As\'i pues, muchos de los pasos del m\'etodo expuesto en el algoritmo
\ref{alg:power_method} que tienen como objetivo normalizar vectores, son
innecesarios, pues $Mx^{(k)}$ ser\'a siempre estoc\'astico (al menos en la
teor\'ia, ya que despu\'es por cuestiones de los c\'alculos n\'umericos con
aritm\'etica finita en la computadora puede llegar a obligarnos a tener que
normalizar los vectores luego de cada c\'alculo).

\par Justamente esta es una de las conclusiones de Bryan y Leise al decirnos
como computar el autovector de PageRank (aunque en ning\'un momento mencionan
cadenas de Markov, o m\'etodo de la potencia), y adem\'as concluyen que el
m\'etodo convergir\'a para cualquier sea $x^{(0)}$ que sea un vector de
probabilidad~\cite[p.580]{Bryan2006}. En su teorema n\'umero 5, nos indican que
nuestro autovector $x$ puede ser calculado como:

\begin{align*}
    x^{(k)} &= \alpha Sx^{(k-1)} + (1-\alpha)\frac{1}{n}ee^T = Mx^{(k-1)}\\
    x &= \lim_{k\rightarrow\infty} x^{(k)}
\end{align*}

\par Claramente aqu\'i cuando se toma un \emph{l\'imite} para calcular $x$, no
es algo realmente computable. Justamente es por eso que el algoritmo
\ref{alg:power_method} implementa un criterio de parada basado en la distancia
del vector entre iteraciones, y a su vez tiene un n\'umero m\'aximo de
iteraciones para llegar a un resultado lo suficientemente aproximado.

\par Esta f\'ormula no es otra cosa que el m\'etodo de la potencia, y justamente
lo m\'as interesante a extraer del art\'iculo de Bryan y Leise (en lo que
respecta al m\'etodo de la potencia aplicado a nuestro problema) es que alcanza
conque el vector inicial $x^{(0)}$ sea un vector de probabilidad. Luego, el
m\'etodo convergir\'a al $x$ que buscamos (pues ya hemos demostrado que nuestra
matriz $M$ tiene un \'unico autovector dominante).

\par De esta manera llegamos a un pseudoc\'odigo del m\'etodo de la potencia
m\'as orientado a nuestro problema, e incluso m\'as claro quiz\'as. El mismo se
puede observar en el algoritmo \ref{alg:power_method2}.

\begin{algorithm}
    \KwIn{Matriz $M$; vector inicial $x^{(0)}$; n\'umero m\'aximo de iteraciones
        $N$; tolerancia $\epsilon$}
    \KwOut{Autovector aproximado $x$}
    $v\gets x^{(0)}$\;
    $k\gets 1$\;
    \Repeat{$\delta < \epsilon \land k\leq N$}{
        $x^{(k)} = Mx^{(k-1)}$\;
        $\delta\gets ||x^{(k)} - x^{(k-1)}||_1$\;
        $k\gets k+1$\;
    }
    \If{$k>N$}{
        \KwRet{Cantidad m\'axima de iteraciones alcanzada}
        \tcc{Finalizaci\'on insatisfactoria}
    }
    \KwRet{$x^{(k)}$}
    \tcc{Finalizaci\'on satisfactoria}

    \caption{M\'etodo de la Potencia en el contexto de PageRank y su matriz $M$
    - Pseudoc\'odigo~\cite[p.263]{Kamvar2003}}
    \label{alg:power_method2}
\end{algorithm}

\par Como comentario, se puede notar que en el algoritmo \ref{alg:power_method2} se
utiliza para el criterio de corte la norma 1 (o distancia \emph{Manhattan}).
Esto es as\'i pues Kamvar, Haveliwala et al. presentaron resultados emp\'iricos
que les permitieron demostrar que dicha medida era
adecuada~\cite[p.268]{Kamvar2003}.

\par As\'i pues hemos llegado a un simple m\'etodo n\'umerico, el cual calcula
un vector aproximado al autovector asociado al autovalor dominante de la matriz
$M$ del modelo explicado en la secci\'on \ref{subsec:intro_pagerank}. A su vez
se demostr\'o que dicho autovalor dominante no es otro que $1$, con lo cual el
autovector $\tilde{x}$ calculado cumplir\'a con $M\tilde{x}\approx\tilde{x}$. Se
prob\'o que dicho m\'etodo servir\'a (es decir, en la teor\'ia su resultado
convergir\'a \textbf{exactamente} al autovector buscado), pero al tratarse de un
m\'etodo iterativo no queda otra que conformarnos con una implementaci\'on que
aproxime al resultado buscado, pues el no podemos iterar hasta el
infinito\footnote{Es razonable esperar un resultado en tiempo finito.}. A\'un
as\'i, como lo que nos interesa en el contexto de este trabajo es obtener el
orden que nos da el autovector, m\'as que los valores de cada componente del
mismo en cuesti\'on, obtener una versi\'on aproximada deber\'ia ser suficiente
\footnote{De hecho obtener la versi\'on exacta es imposible, ya que la
computadora trabaja con aritm\'etica finita.}.
%---------------------------------------------------------------
