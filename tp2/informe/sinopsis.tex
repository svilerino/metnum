La elaboraci\'on de \'ordenes es uno de los problemas m\'as frecuentes que se
da en nuestra sociedad. Virtualmente todos los \'ambitos conocidos (deportes,
laboral, l\'udico, salarial, etc) enfrentan este problema, que ser\'a m\'as o
menos complejo seg\'un el contexto. A lo largo de este trabajo estudiamos este
problema en el \'ambito de las p\'aginas web: ¿cu\'ando una p\'agina deber\'ia
estar por encima de otra?

\par Presentamos un modelo de generaci\'on de rankings\footnote{Coloquialismo}
orientado al ordenamiento de p\'aginas web para resultados de motores de
b\'usqueda. Pasamos entonces a dar un m\'etodo num\'erico que nos asegura la
existencia de una soluc\'on para el modelo dado, para luego implementar el mismo
y analizarlo emp\'iricamente tanto en aspectos computacionales como del dominio
del problema. Por \'ultimo, analizamos le extensi\'on de este modelo al \'ambito de las
competiciones deportivas. Presentamos el modelo, lo implementamos y analizamos
sus comportamientos emp\'iricos respecto de los rankings reales utilizados por
las competencias, as\'i como con casos abstractos ideales para poder constratar
la calidad de los resultados contra algo m\'as determin\'istico que las
competencias deportivas.
