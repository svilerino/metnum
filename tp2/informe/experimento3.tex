\subsection{Convergencia de PageRank}
\label{subsec:exp3}
\begin{LaTeXdescription}
    \item[Objetivo] Estudiar como se comporta la convergencia en funci\'on de
        el factor de navegaci\'on $\alpha$.\\

    \item[Proposici\'on] PageRank propone una soluci\'on donde se observa a un
        conjunto de p\'aginas web como un grafo dirigido. Luego pasa este modelo
        a uno  matem\'atico, utilizando una matriz para computar una soluci\'on.
        Esta matriz, por construcci\'on, termina representando a un grafo
        completo (el grafo original no necesariamente lo era), y los valores del
        mismo dependen principalmente de 2 variables: $\epsilon$ y $\alpha$
        (algoritmo \ref{alg:power_method2}, p\'agina \pageref{alg:power_method2}
        y ecuaci\'on \ref{eq:M_def}, p\'agina \pageref{eq:M_def}). Si se observa
        el algoritmo, se ver\'a que claramente modificar el $\epsilon$ tendr\'a
        como resultado hacer que cada corrida converja en m\'as o menos
        iteraciones, ya que el criterio de parada depende de \'el. As\'i pues,
        no vemos como algo rico experimentar con este valor. En cambio, $\alpha$
        es una variable que modifica en gran medida a nuestra matriz $M$, con lo
        cual es dificil saber cual ser\'a su injerencia en la convergencia (en
        principio). El objetivo, entonces, es ver como $\alpha$ afecta al
        m\'etodo matem\'atico iterativo de la potencia en cuanto a su
        convergencia.\\

    \item[M\'etodo de Experimentaci\'on] Tomaremos 3 instancias de grafos de
        conectividad de p\'aginas web de tama\~no mediano-grande, obtenidas en
        \url{http://snap.stanford.edu/data/\#web}. Luego, correremos el
        algoritmo de PageRank para $\alpha=0.0$; $0.1$; $0.2$; $\dots$; $0.9$ y
        $\epsilon$ fijo en $0.00001$.\\

    \item[Resultados, an\'alisis y discusi\'on]
\end{LaTeXdescription}

\par En el gráfico \ref{subfig:exp3_comp} se exponen la cantidad de iteraciones
necesarias hasta converger a medida que el parámetro $\alpha$ crece. Para las 3
instancias el comportamiento observado es el mismo: a medida que el par\'ametro
$\alpha$ aumenta, la cantidad de iteraciones necesarias para converger crece
exponencialmente. Puede observarse que aunque con valores numéricos diferentes,
las curvas pertenecen a la misma familia de funciones, y por lo tanto tienen un
comportamiento id\'entico (emp\'iricamente hablando) en cuanto a la cantidad de
iteraciones en funci\'on de $\alpha$.

\begin{figure}[H]
    \centering
    \caption{An\'alisis de Convergencia en funci\'on de $\alpha$}
    \subfloat[][Iteraciones hasta Converger en funci\'on de $\alpha$]{
        \label{subfig:exp3_comp}
        \includegraphics[width=.5\textwidth]{exp3_iteraciones_funcion_alpha.png}
    }
    \subfloat[][Velocidad de convergencia en funci\'on de $\alpha$\\Escala
    logar\'itmica]{
        \includegraphics[width=.5\textwidth]{exp3_diff_funcion_iteraciones_standford.png}
        \label{subfig:exp3_diff}
    }
\end{figure}

\par Como se explic\'o en la secci\'on \ref{sec:introduccion}, el factor
$\alpha$ nos hacen variar la proporci\'on entre $S$ y la matriz equiprobable a
la hora de definir a $M$. Es decir, en el rango de posibles valores de $\alpha$,
la \'unica manera en la que se afecta a $M$ en en los valores de cada uno de sus
elementos. Pero para ning\'un $\alpha$ se puede pasar a tener un elemento nulo
en $M$ (justamente se quer\'ia tener la representaci\'on de un digrafo
completo).

\par Se nos ocurre conjeturar que en el contexto de una cadena de Markov y el
navegante aleatorio, un $\alpha$ pequeño nos genera una matriz de transici\'on
''m\'as'' equiprobable (recordar la definici\'on de $M$ en la ecuaci\'on
\ref{eq:M_def}, p\'agina \pageref{eq:M_def}) y dado que el método de la potencia
comienza inicialmente con el vector equiprobable, le toma pocas iteraciones
converger. A medida que $\alpha$ aumenta, el grafo generado denota más la
navegación estricta por los links entre los sitios, reduciendo la probabilidad
de teletransportación. Es decir, a mayor $\alpha$, el grafo se ''parece m\'as''
al grafo de conectividad original no adulterado (con los dangling nodes ya
resueltos), en el sentido de que $S$ predomina mucho m\'as en $M$ que
$(\rfrac{1}{n})ee^T$. Esta matriz $S$ no necesariamente representa un grafo
fuertemente conexo, una de las condiciones necesarias para asegurar convergencia
del método de la potencia en el contexto de este problema, con lo cual el método
de la potencia inicia sus iteraciones con una, si se quiere, \textbf{seguridad
de convergencia más débil}.

\par Respecto a la velocidad de convergencia, exponemos los resultados de una
sola instancia ya que para las 3 instancias los resultados son similares. Como
puede observarse en el gráfico \ref{subfig:exp3_diff}, a medida que el factor de
navegacion $\alpha$ crece, la velocidad de convergencia disminuye. Es decir, la
distancia \emph{Manhattan} entre el autovector calculado entre dos iteraciones
consecutivas disminuye cada vez m\'as, acerc\'andose cada vez \textbf{m\'as
rápido} al umbral de corte del algoritmo (basado en $\epsilon$ o una cantidad
fija m\'axima de iteraciones\footnote{Esto se implementa, ya que si bien la
teor\'ia nos indica que el m\'etodo convergir\'a, los errores n\'umericos de la
aritm\'etica finita puede hacer que esto no ocurra.}).
