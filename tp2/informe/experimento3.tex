\subsection{Convergencia}
\label{subsec:exp3}
\begin{LaTeXdescription}
    \item[Objetivo] Estudiar como se comporta la convergencia en funci\'on de
        el factor de navegaci\'on $\alpha$.

    \item[Proposici\'on] PageRank propone una soluci\'on donde se observa a un
        conjunto de p\'aginas web como un grafo dirigido. Luego pasa este modelo
        a uno  matem\'atico, utilizando una matriz para computar una soluci\'on.
        Esta matriz, por construcci\'on, termina representando a un grafo
        completo (el grafo original no necesariamente lo era), y los valores del
        mismo dependen principalmente de 2 variables: $\epsilon$ y $\alpha$
        (algoritmo \ref{alg:power_method2}, p\'agina \pageref{alg:power_method2}
        y ecuaci\'on \ref{eq:M_def}, p\'agina \pageref{eq:M_def}). Si se observa
        el algoritmo, se ver\'a que claramente modificar el $\epsilon$ tendr\'a
        como resultado hacer que cadda corrida converga en m\'as o menos
        iteraciones, ya que el criterio de parada depende de \'el. As\'i pues,
        no vemos como algo rico experimentar con este valor. En cambio, $\alpha$
        es una variable que modifica en gran medida a nuestra matriz $M$, con lo
        cual es dificil saber cual ser\'a su injerencia en la convergencia (en
        principio). El objetivo, entonces, es ver como $\alpha$ afecta al
        m\'etodo matem\'atico iterativo de la potencia en cuanto a su
        convergencia.

    \item[M\'etodo de Experimentaci\'on] Tomaremos 3 instancias de grafos de
        conectividad de p\'aginas web de tama\~no mediano-grande, obtenidas en
        \url{http://snap.stanford.edu/data/\#web}. Luego, correremos el
        algoritmo de PageRank para $\alpha=0 0.1 0.2 \dots 0.9$ y $\epsilon$
        fijo en $0.00001$

    \item[Resultados, an\'alisis y discusi\'on]
\end{LaTeXdescription}

\begin{figure}[!h]
    \centering
    \subfloat[][Iteraciones hasta Converger en funci\'on de $\alpha$]{
        \label{subfig:exp3_comp}
        \includegraphics[width=.5\textwidth]{exp3_iteraciones_funcion_alpha.png}
    }
    \subfloat[][Velocidad de convergencia en funci\'on de $\alpha$]{
        \includegraphics[width=.5\textwidth]{exp3_diff_funcion_iteraciones_standford.png}
        \label{subfig:exp3_diff}
    }
    \caption{An\'alisis de Convergencia en funci\'on de $\alpha$}
\end{figure}
