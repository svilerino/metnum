\par \IEEEPARstart{F}{i}nalizada la enumeraci\'on de ciertos rubros que motiven la investigaci\'on sobre el tema, con la adici\'on de una breve perspectiva del problema a trabajar y de la justificación teórica detrás del método de interpolación, nos dedicaremos a desarrollar con mayor precisi\'on cada m\'etodo n\'umerico en cuesti\'on. Recordemos que estos métodos intentan atacar el problema de construir, a partir de un video filmado en tiempo real, nuevos frames que den la sensaci\'on de que el video original ha sido ralentizado (efecto de \emph{slowmotion}), o, aún mejor, filmado con una cámara de alta frecuencia.

\subsection{Los m\'etodos propuestos}

Para la explicación de todos los métodos consideramos dado (como parámetros del problema) el video sobre el cual trabajar y la cantidad $n$ de frames a generar entre cada par de frames del original. Llamaremos \emph{frames artificiales} a los frames generados por los métodos, y \emph{frames originales} a los frames provenientes del video original.

\subsubsection{Cuadro m\'as cercano}

Este m\'etodo se basa en una idea de car\'acter simplista, muy sencilla de implementar, y que nos permitirá tener una base para establecer comparaciones con métodos más inteligentes. A pesar de esto, puede resultar precisa para ciertos tipos de videos, como por ejemplo, la filmaci\'on de un objeto inmóvil.

Como lo indica el subt\'itulo de la secci\'on, la idea del método es que cada frame artificial es una copia de su frame original más cercano en términos temporales. Para cada frame a generar el método calcula cuál es dicho ``vecino más cercano'', y luego copia la im\'agen de dicho frame al frame artificial a generar. 

De esta manera, habr\'a al menos $n/2$ copias de cada frame original. En caso que se desee agregar una cantidad impar, se opta por fragmentar en dos partes de $n/2$ y $n/2+1$ cuadros. En la partici\'on de mayor peso, es el usuario quien selecciona qu\'e extremo escoger de los originales. En la siguiente figura, nos encontramos con un ejemplo consico de lo explicado: 

( Ejemplo gra\'fico o dejar eso para el desarrollo )

( Breve explicaci\'on del ejemplo )

Volviendo al an\'alisis del video en su mera totalidad, se repite el anterior procedimiento en cada par de frames en el orden concebido por la secuencia del video. De tal forma, se obtienen los frames artificiales, consiguiendo una nueva filmaci\'on con el efecto de $slowmotion$ que este m\'etodo propone. 

( Alg\'un comentario final )

\subsubsection{Interpolaci\'on lineal}

Semejante al \emph{cuadro m\'as cercano}, en el sentido que comparten la ideolog\'ia de trabajar c\'iclicamente con cada par de frames para eventualmente, obtener el video deseado con su respectivo efecto. No obstante, contaremos lo particular y caracter\'istico de este m\'etodo, que puede propocionar ciertas mejoras en el movimiento de objetos durante su filmaci\'on.

Por un lado, no posee la misma intuici\'on que su antecesora, donde la im\'agen de cada frame artificial se basa en copiar alguno de los dos cuadros originales en evaluaci\'on. En cambio, se busca utilizar fundamentos matem\'aticos que ayuden a \emph{predecir y reflejar} lo sucedido entre cada par de frames del video original. 

Nuevamente, nos enfocamos en analizar el algoritmo propuesto para la creaci\'on de los $n$ frames artificiales que se desean adherir entre cada conjunto de pares. En primer lugar, debemos pensar a cada frame como una matriz de $m$ $x$ $n$ p\'ixeles (dependiendo la resoluci\'on en que se dispone el video), siendo $m$ el largo del cuadro y $n$ el ancho. En segundo lugar, el procedimiento destina a generar un polinomio de grado uno \footnote{ Tambi\'en definida como funci\'on lineal; \url{https://es.wikipedia.org/wiki/Lineal}.} entre ese par de frames originales. Aunque, ¿de qu\'e forma inventaremos dichos polinomios? Para ello, introducimos algunos conceptos que resolver\'an esta inc\'ognita.

Contamos previamente acerca de la representación de tales cuadros como matrices compuestas por p\'ixeles, y de dichos píxeles como números de 0 a 255.

Dicho esto, el m\'etodo se centra en crear un polinomio interpolador por cada posici\'on del frame cuyos puntos interpolados resultan ser los valores de dicha ubicaci\'on en el par de frames originales que se est\'a trabajando. Por ende, tendremos por cada par de cuadros reales, $m$ $x$ $n$ polinomios de grado uno. De lo anterior, nace una nueva duda : ¿C\'omo esto resuelve nuestro problema de crear m\'utiples frames artificiales?

El hecho esta en que ahora conocemos los valores intermedios entre los dos cuadros originales y en consecuencia, podemos particionar ese dominio en $p$ partes ($p$ siendo la cantidad de frames que se adiciona en cada par) y finalmente, evaluar el polinomio en dicho borde de cada fragmento. Como observaci\'on, este paso lo tendremos que aplicar para cada posici\'on de la im\'agen.

De misma forma, se realiza la tarea sobre cada par agrupado en el orden indicado con lo que eventualmente, el m\'etodo en cuesti\'on concluye con su labor, dejando el efecto de $slowmotion$ que la misma ofrece.

( Cosas para agregar : imagenes o ejemplo sencillo.)