%!TEX root = informe.tex
\IEEEPARstart{A}{} lo largo de este trabajo atacamos el problema de generar
frames artificiales para un video, intentando lograr que se ajusten al original
hasta el punto -ideal- de que el ojo humano no detecte la diferencia con un
video filmado en alta frecuencia.

\par Como primera conclusión al respecto podemos decir que no logramos dicho
objetivo ideal: de nuestras observaciones descubrimos fácilmente que, al menos
para los tres métodos implementados, los resultados ``no engañan a nadie''.
Todos los métodos presentan o bien \emph{lag} o bien \emph{fantasmeo},
perturbando la sensación de fluidez y alterando el realismo percibido del
video.

\par Más allá de ese análisis cualitativo, pudimos observar que atacar el
problema de forma \emph{naïve} (el método que llamamos ``Vecino más cercano'')
produce resultados con error matemático más reducido aunque visualmente
pareciera ser el peor método, pudiendo obtenerse resultados visualmente más
agradables con métodos más inteligentes como interpolación mediante
poliniomios. As\'i pues, en realidad, consideramos que la m\'etrica del ECM
no es necesariamente la ideal para tratar de identificar que m\'etodo comente
''menos error'' (pensando el error como el efecto final ante los espectadores
del video procesado). Probablementa se deban investigar otras posibles variables
que puedan tener alguna correlaci\'on para con la percepci\'on humana promedio.

\par También concluimos que, visto como método de compresión, los resultados
son pobres. En la actualidad existen métodos que, sin utilizar mucho más
espacio, generan prácticamente nulos \emph{artifacts} y no pierden información
de cuadros completos.

\par De los resultados obtenidos de la experimentaci\'on tambi\'en observamos
un \'unico caso (el de la c\'amara fija, im\'agen m\'ovil) donde el tama\~no
del bloque del m\'etodo de spline pareciera tener algun tipo de influencia.
De los resultados sobre esta variable nos queda la sensaci\'on de que no
la hemos explotado a fondo. La intuici\'on nos indica que en caso de haber
podido analizar casos donde los movimientos fueran abruptos/bruscos y suaves,
habr\'iamos visto diferencias notorias con los tama\~nos de bloque de este
m\'etodo. Claramente, a estas alturas, esto queda para futuros trabajos.

\par Pasando a cuestiones cuantitativas, observamos como es evidente que los
m\'etodos m\'as ''baratos'' (computacionalmente hablando) son los m\'as
simples: vecino m\'as cercano es m\'as barato que lineal, que a su vez es m\'as
barato que spline. Y de los resultados globales obtenidos en cuanto a la
percepci\'on, consideramos que a la hora de implementar un filtro de c\'amara
lenta, el m\'etodo por defecto (sin saber nada m\'as al respecto) ser\'ia el de
interpolaci\'on lineal, por obtener este resultados ''subjetivos'' (la
consideraci\'on de los autores de que m\'etodo logr\'o mejor el efecto)
similares a spline y mejores que vecino m\'as cercano, y por ser m\'as barato
que spline.

\par Para finalizar, era la intenci\'on de quienes les escriben realizar un
an\'alisis m\'as pormenorizado de los m\'etodos en funci\'on de los tipos de
videos evaluados. Es decir, tomar un m\'etodo y ver como var\'ia los ECM/PSNR,
desv\'io est\'andar, etc seg\'un el tipo de video con el que se trabaje. Por
cuestiones, nuevamente, de tiempo, no se alcanz\'o a realizar dicho an\'alisis
que consideramos que hubiera sido muy prol\'ifero.
