%!TEX root = informe.tex
\IEEEPARstart{A}{} lo largo de este trabajo atacamos el problema de generar frames artificiales para un video, intentando lograr que se ajusten al original hasta el punto -ideal- de que el ojo humano no detecte la diferencia con un video filmado en alta frecuencia.

Como primera conclusión al respecto podemos decir que no logramos dicho objetivo ideal: de nuestras observaciones descubrimos fácilmente que, al menos para los tres métodos implementados, los resultados ``no engañan a nadie''. Más allá de ese detalle cualitativo, sí pudimos observar que atacar el problema de forma \emph{naïve} (el método que llamamos ``Vecino más cercano'') produce resultados que pueden mejorarse significativamente, obteniendo soluciones más suaves y prolijas con métodos más inteligentes, como interpolación mediante poliniomios.

También concluimos que, visto como método de compresión, los resultados son también pobres. En la actualidad existen métodos que, sin utilizar mucho más espacio, generan prácticamente nulos \emph{artifacts} y no pierden información de cuadros completos.