\IEEEPARstart{E}{N} esta parte del trabajo se explican algunas de las hip\'otesis
que pudieron ser formuladas al pensar en el problema. Como ya se ha dicho, la
naturaleza de este trabajo ha sido m\'as exploratoria, pero a\'un as\'i se
comenz\'o el mismo con algunas conjeturas y nociones de lo que deb\'ia ocurrir.

\begin{enumerate}
    \item\begin{LaTeXdescription}
        \item[Hip\'otesis] A mayor cantidad de frames a interpolar, menor
            precisi\'on tendr\'a el video si se consideran los errores de todos
            los cuadros interpolados (respecto de lo que deber\'ian haber sido
            en caso de ser ''filmados'').

        \item[Motivaci\'on] La justificaci\'on de esta hip\'otesis es bastante
            simple e intuitiva. Uno debe utilizar dos frames para generar
            una cantidad fija de nuevos frames que se reproducir\'an entre
            estos dos frames iniciales. Luego, los frames constru\'idos que
            est\'an m\'as cerca de los frames dato (u originales) tiene sentido
            que sean tengan menor error que aquellos m\'as alejados (ya que
            los movimientos que se graban en los videos son continuos, si se los
            piensa matem\'aticamente). Es entonces m\'as probable que las
            funciones interpoladoras a utilizar en el trabajo tengan m\'as
            precisi\'on ''cerca'' de los puntos que interpolan (es decir, los
            frames originales). As\'i pues, de haber cada vez m\'as frames a
            ser generados, cada vez habr\'a m\'as frames con mayor error,
            incrementando el error promedio del video completo.
    \end{LaTeXdescription}\medskip

    \item\begin{LaTeXdescription}
        \item[Hip\'otesis] A mayor cantidad de frames a interpolar, mayor
            tiempo de c\'omputo ser\'a necesario.

        \item[Motivaci\'on] Nuevamente, esta es una idea intuitiva y simple. Si
            se deben generar m\'as frames, m\'as tiempo ser\'a necesario para
            construirlos (m\'as all\'a de que no se requiera m\'as tiempo para
            calcular el polinomio interpolador de cada frame\footnote{En realidad,
            como ya se ha explicado, habr\'a un polinomio interpolador por cada
            p\'ixel. Pero se puede pensar al conjunto de los polinomios de todos
            los p\'ixeles como el polinomio interpolador del frame. Es un
            peque\~no abuso de notaci\'on para facilitar la escritura y
            lectura.}).
    \end{LaTeXdescription}\medskip

    \item\begin{LaTeXdescription}
        \item[Hip\'otesis] A mayor resoluc\'on de los frames, mayor tiempo
            de c\'omputo ser\'a necesario.\footnote{Si bien pudimos realizar
            esta experimentaci\'on para un set de videos con mayor
            resoluci\'on, por limitantes de tiempo los resultados no pudieron
            ser expuestos en este trabajo.}.

        \item[Motivaci\'on] 
    \end{LaTeXdescription}\medskip

    \item\begin{LaTeXdescription}
        \item[Hip\'otesis] Al haber m\'as movimiento en el video, menos precisos
            ser\'an los frames interpolados (respecto de lo que deber\'ian haber
            sido en caso de ser filmados).

        \item[Motivaci\'on] Esta es quiz\'as nuestra hip\'otesis m\'as fuerte.
            B\'asicamente entendemos que el principal problema al tratar de
            generar nuevos frames que se comporten como lo har\'ia lo que sea
            que se haya filmado es estimar como ser\'an los movimientos. Los
            movimientos, desde el punto de vista de una filmaci\'on, son los
            cambios en las im\'agenes. Un p\'ixel movi\'endose por una pantalla
            no es otra cosa que un pixel con un color/tonalidad\footnote{En el
            caso de trabajar con filmaciones en blanco y negro.} en un frame, y
            que al siguiente dicho pixel cambia de tonalidad y alguno de sus
            p\'ixeles vecinos toma el color/tonalidad del p\'ixel mencionado
            para el frame previo (o una tonalidad similar, ya que tambi\'en
            podr\'iamos tener una filamci\'on afectada por la iluminaci\'on).
            As\'i pues, consideramos que por la forma de aproximar que tienen
            los m\'etodos implementados, el movimiento ser\'a justamente el
            aspecto m\'as problem\'atico, ya que los mismos no tienen en cuenta
            de ninguna manera este concepto, sino que simplemente se basan en
            los valores de los p\'ixeles (por ejemplo, un movimiento repentino,
            donde un pixel pasa de un color a otro en frames contiguos,
            probablemente al generarse frames intermedios estos muestren un
            cambio suave de tonalidad cuando en realidad, por la naturaleza de
            lo filmado, el cambio deber\'ia ser abrupto).
    \end{LaTeXdescription}\medskip

    \item\begin{LaTeXdescription}
        \item[Hip\'otesis] Los 3 m\'etodos a evaluar ser\'an m\'as similares en
            cuanto a la precisi\'on/error a menor cantidad de frames
            interpolados.

        \item[Motivaci\'on] A partir de la primera hip\'otesis, es razonable
            pensar que a menor cantidad de frames interpolados, menor ser\'a
            el error introducido en el video para cualquier m\'etodo de
            interpolaci\'on.
    \end{LaTeXdescription}\medskip

    \item\begin{LaTeXdescription}
        \item[Hip\'otesis] El m\'inimo error cometido (es decir, el error
            m\'inimo cometido al interpolar un frame respecto de lo que
            deber\'ia ser) ser\'a siempre el mismo para el m\'etodo del
            vecino m\'as cercano.

        \item[Motivaci\'on] La naturaleza del m\'etodo del vecino m\'as cercano
            nos asegura que, no importa la cantidad de frames que se
            interpolen, los frames generados inmediatos a los originales
            ser\'an una copia de estos \'ultimos (correspondientemente). Y esto
            no cambiar\'a aunque se aumente la cantidad de frames interpolados.
            As\'i pues, utilizando el mismo razonamiento con el cual se
            explic\'o que los frames artificiales m\'as cercanos a los
            originales son los que menor error deber\'ian tener, y teniendo que
            los frames m\'as cercanos a los originales ser\'an siempre los
            mismos (para este m\'etodo), tenemos que su error no deber\'ia
            variar significativamente si hay m\'as frames artificiales
            a continuaci\'on de los frames interpolados m\'as cercanos a los
            originales.
    \end{LaTeXdescription}\medskip
\end{enumerate}

\par B\'asicamente la pregunta que esperamos poder responder luego de
experimentar y analizar los resultados es: \textbf{¿Es alg\'un m\'etodo mejor
que los otros?}
