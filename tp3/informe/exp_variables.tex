A partir de las hip\'otesis enunciadas, se determinaron
ciertas caracter\'isticas y par\'ametros que posiblemente afecten a la calidad
de las interpolaciones aplicadas a los videos. Estas son:

\begin{LaTeXdescription}
    \item[Frames Interpolados] La cantidad de frames artificalmente generados
        que ser\'an insertados entre cada par de frames del video original. Se
        consideron 4 alternativas para esta variable: 1, 2, 5 o 10 frames.
        Agregar de a 1 frame cada virtualmente duplica la cantidad de frames
        del video original, con lo cual si se mantiene el \emph{ratio de
        reproducci\'on}\footnote{Frames per second.} se estar\'ia obteniendo el
        efecto de reproducir a la ''mitad'' de la velocidad original. Se
        consider\'o algunas alternativas m\'as, aunque es d\'ificil imaginar
        alg\'un contexto donde se busque reproducir artificalmente el efecto de
        \emph{c\'amara lenta} con algo m\'as de 5 frames.\medskip

    \item[M\'etodo de Interpolaci\'on] Tenemos 3 formas/m\'etodos de generar
        los frames interpolados: Vecino m\'as Cercano, Interpolaci\'on Lineal,
        Interpolaci\'on C\'ubica o Spline. Claramente cada uno de ellos funciona
        de forma distinta, como ya se ha explicado. As\'i pues, la elecci\'on
        del m\'etodo deber\'ia afectar a los frames generados artificialmente,
        ya que dificilmente en un caso de uso real los m\'etodos devuelvan un
        frame artificial id\'entico.\medskip

    \item[Tama\~no de Bloque (splines)] El m\'etodo de spline puede pensarse
        como aplicado a todo el video, o se podr\'ian aplicar a distintos
        bloques a lo largo del video (generando disintos splines por cada
        bloque del video).  Parte de los objetivos de la experimentaci\'on es
        ver si esto afecta a los resultados de alguna manera. Los valores
        considerados fueron 4, 8, 16 y 32 frames (m\'as el caso de aplicar el
        m\'etodo a todo el video).\medskip

    \item[Resoluci\'on del Video] ¿La resoluci\'on del video afecta a los
        m\'etodos de alguna manera?¿Es acaso alguno de ellos (o todos) m\'as
        preciso para resoluciones mayores?\footnote{Se experiment\'o con los
        mismos videos en 2 resoluciones diferentes, pero por cuestiones de
        tiempo los resultados de las resoluciones m\'as grandes no pudieron ser
        analizadas.}.\medskip

    \item[Tipo Movimiento grabado por la C\'amara] A partir de una c\'amara uno
        podr\'ia grabar algo est\'atico o con poco movimiento, como por ejemplo
        el crecimiento de la vida vegetal\footnote{O a la defensa de Platense
        jugando al \emph{offside}}, o algo con m\'as movimiento como por
        ejemplo una carrera de autos o un evento deportivo\footnote{No, no
        quedamos manija luego del TP2}.  As\'i pues, dados los tiempos de este
        trabajo, decidimos hacer una bipartici\'on de los posibles videos
        respecto de esta caracter\'istica: videos que graban cosas est\'aticas
        como videos que filman situaciones donde hay movimiento.  Hay
        distinciones que ser\'ian interesantes tener en cuenta, como por
        ejemplo si los movimientos son abruptos o suaves, pero nuevamente los
        tiempos para confeccionar esta trabajo no permitieron dicho
        an\'alisis.\medskip

    \item[Tipo de Movimiento de la C\'amara] Al igual que en el item anterior,
        bien se puede considerar los movimientos de la c\'amara que efect\'ua
        la filmaci\'on. De hecho, podr\'ia llegar a considerarse que si se
        analiza el movimiento de un video, no importa si el mismo proviene de
        mover la c\'amara o de filmar algo que efectivamente cambie de
        posici\'on. A\'un as\'i, dado que en este aspecto en realidad estamos
        explorando como se comportan los m\'etodos propuestos, se decidi\'o
        realizar una partici\'on similar al caso previo para los movimientos
        de la c\'amara: videos donde la c\'amara est\'e fija y videos donde
        la misma presente alg\'un tipo de movimiento. Nuevamente entra en juego
        otras distinciones respecto de como son estos movimientos que no
        pudieron ser evaluadas por las limintantes de tiempo ya
        mencionadas.\medskip
\end{LaTeXdescription}

\par Estas son algunas de las variables que se podr\'ian en tener en cuenta (en
particular las que se tuvieron en cuenta en este trabajo), pero no son las
\'unicas. Por dar un ejemplo, una variable con la que no se trabaj\'o fue con
los videos que cambi\'an de c\'amara, donde se pas\'a de un tipo de im\'agen a
otra de form\'a rotunda en frames contiguos (podr\'ia haber alg\'un tipo de
transici\'on tambi\'en, que ser\'ia otro caso a estudiar). Fue meramente por
cuestiones de tiempo que se decidi\'o limitar el enfoque de este trabajo a las
variables/par\'ametros enunciados.
