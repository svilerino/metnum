A partir de las hip\'otesis enunciadas, se determinaron
ciertas caracter\'isticas y par\'ametros que posiblemente afecten a la calidad
de las interpolaciones aplicadas a los videos. Estas son:

\begin{LaTeXdescription}
    \item[Frames Interpolados]

    \item[M\'etodo de Interpolaci\'on]

    \item[Tama\~no de Bloque (splines)]

    \item[Resoluci\'on del Video]

    \item[Duraci\'on del Video]

    \item[Tipo Movimiento grabado por la C\'amara]

    \item[Tipo de Movimiento de la C\'amara]
\end{LaTeXdescription}

\par Estas son algunas de las variables que se podr\'ian en tener en cuenta (en
particular las que se tuvieron en cuenta en este trabajo), pero no son las
\'unicas. Por dar un ejemplo, una variable con la que no se trabaj\'o fue con
los videos que cambi\'an de c\'amara, donde se pas\'a de un tipo de im\'agen a
otra de form\'a rotunda en frames contiguos (podr\'ia haber alg\'un tipo de
transici\'on tambi\'en, que ser\'ia otro caso a estudiar). Fue meramente por
cuestiones de tiempo que se decidi\'o limitar el enfoque de este trabajo a las
variables/par\'ametros enunciados.
