\IEEEPARstart{D}{urante} la experimentaci\'on, y dada la naturaleza de la misma,
se sigui\'o una misma metodolog\'ia para recabar la informaci\'on necesaria para
ser luego analizada.

\par En primer lugar se decidi\'o trabajar con im\'agenes en escala de grises o,
coloquialmente llamado \emph{blanco y negro}. Esto es as\'i ya que dentro del
alcance de nuestro trabajo, estamos estudiando (principalmente) la correctitud
de los frames interpolados. Si se tuvieran en cuenta los colores, se tendr\'ia
la complicaci\'on del error agregado de interpolar los canales de colores por
separado, lo cual escapa a los objetivos del trabajo (adem\'as de tener la
limitante del tiempo para realizar este trabajo.). Resumiendo, se tom\'o esta
decisi\'on para simplificar y enfocar m\'as el an\'alisis a realizar.

\par La forma en la que se presentar\'an los resultados m\'as adelante nada
tiene que ver con el orden en el que se realiz\'o la experimentaci\'on.

\par De la secci\'on anterior se puede observar que existen dos variables que
nos indican (o limitan) los tipos de videos con los que trabajaremos. Estas dos
variables no son otra que las diferentes combinaci\'on del tipo de movimiento
filmado y de la c\'amara que realiza la grabaci\'on. As\'i pues, se termin\'o
teniendo cuatro combinaci\'ones posibles: \emph{c\'amara fija-im\'agen fija},
\emph{c\'amara fija-im\'agen m\'ovil}, \emph{c\'amara m\'ovil-im\'agen fija} y
\emph{c\'amara m\'ovio-im\'agen m\'ovil}\footnote{Existen matices en estas
combinaciones. Por ejemplo, \emph{la suavidad} de los movimientos, que podr\'ian
variar en intensidad yendo de suaves a bruscos. Nuevamente, por cuestiones de
tiempo se decidi\'o no incursionar en estos aspectos.}.

\par Para cada una de estas categor\'ias de videos, se efectu\'o la
interpolaci\'on con los 3 m\'etodos expuestos y todas sus posibles combinaciones
de par\'ametros (cantidad de frames a interpolar entre frame y frame, y en el
caso de spline el tama\~no de bloque.). Para luego poder hacer un an\'alisis
objetivo de la calidad de los frames interpolados resultantes, lo que se hizo
fue remover de los videos originales una cantidad calculada\footnote{En base
a la cantidad de frames a interpolar.} de frames antes de interpolar el mismo.
De esta manera se puede comparar los frames interpolados resultantes con los
frames originales (los que fueron removidos previos al procesamiento), que no
son otra cosa que el resultado que uno quisiera obtener de la interpolaci\'on.

\par Luego se realizan los an\'alisis de los m\'etodos independientemente de los
dem\'as m\'etodos, es decir que analizamos los resultados obtenidos de cada
m\'etodo respecto de sus par\'ametros de entrada.

\par Continuamos realizando un an\'alisis de m\'etodo versus m\'etodo, para
los mismos par\'ametros (en el caso de splines, al tener 2 par\'ametros de
entrada\footnote{Cantidad de bloques a interpolar y tama\~no de bloque.} se
tuvieron que utilizar resultados obtenidos del an\'alisis del m\'etodo
realizado previamente).

\par Por \'ultimo, antes de terminar enunciando las conclusiones obtenidas para
el tipo de video estudiado, se generaron unos videos que realizan la
comparativa frame a frame de la diferencia entre el frame del video original y
su contraparte interpolada para luego ser estudiados (visualizado en forma de
\emph{heatmap}). La idea de esto es la de encontrar en que zonas de los frames,
respecto de lo que ocurre en el video, genera problemas para los m\'etodos
estudiados.

\par Las m\'etricas utilizadas para los an\'alisis de los m\'etodos no son otras
que las sugeridas por la c\'atedra: el \emph{Error Cuadr\'atico
Medio}~\cite{mse} y el \emph{Peak Signal to Noise Ratio}~\cite{psnr}. El primero
es una medida de error entre un valor (en nuestro caso, un frame) estimado y lo
que es estimado (el frame original). Mientras que el \emph{PSNR} es un ratio
que toma en cuenta el \emph{ECM} y el valor m\'aximo siendo
estimado\footnote{Oriundo del an\'alisis de se\~nales, donde indica la
relaci\'on entre la potencia m\'axima de una se\~nal y la potencia del ruido
que la afecta.}. Justamente el PSNR es utilizado para cuantificar la calidad de
la reconstrucci\'on de una se\~nal (o en nuestro caso, de un frame), un alto
valor del mismo del mismo indica mayor calidad de la reconstrucci\'on (aunque
se debe tener cierto cuidado en los casos donde existe compresi\'on de datos
involucrada, caso que no aplica a este trabajo). Tambi\'en se usaron los
valores medios, desv\'io est\'andard, m\'aximo y m\'inimo del ECM para realizar
comparaciones para el mismo m\'etodo y sus diferentes par\'ametros, como as\'i
tambi\'en entre los distintos m\'etodos.

\par Vale la pena aclarar que en las conclusiones generales es donde tambi\'en
se realiza un an\'alisis m\'as subjetivo, basado en la percepci\'on de los
autores de este trabajo. Claramente esta no es una muestra lo suficientemente
alta de las posibles audiencias de los videos, pero alcanza dado el objetivo
did\'actico del trabajo.

\par Para concluir con esta secci\'on, agregamos a su vez que se realizaron
otros experimentos m\'as puntuales para analizar los aspectos de correctitud de
los m\'etodos (m\'inimamente) y del tiempo de c\'omputo, como as\'i tambi\'en
una comparativa de como los m\'etodos se ven afectados seg\'un la variable
m\'as importante de todas: el video.

\par Los videos originales utilizados para la experimentaci\'on pueden
obtenerse en \url{https://drive.google.com/open?id=0B0RfkWV-4-XqMlhfa0Z2WUMtRTg}
